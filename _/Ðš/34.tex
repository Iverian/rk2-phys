\documentclass[12pt]{article}

\usepackage[a4paper,
            total={170mm,255mm},
            left=10mm,
            top=15mm]
            {geometry}

\usepackage{fontspec,
            polyglossia,
            graphicx}

\usepackage{amsmath,
            amsthm,
            amssymb}

\usepackage{unicode-math,
            tensor}

\usepackage{wrapfig,
            hyperref,
            multicol,
            multirow,
            tabularx,
            booktabs,
            subfiles}

\setdefaultlanguage{russian}
\setotherlanguage{english}
\setkeys{russian}{babelshorthands=true}

\defaultfontfeatures{Ligatures=TeX}
\setmainfont{STIX Two Text}
\setmathfont{STIX Two Math}
\DeclareSymbolFont{letters}{\encodingdefault}{\rmdefault}{m}{it}

\newfontfamily{\cyrillicfont}{STIX Two Text} 
\newfontfamily{\cyrillicfontrm}{STIX Two Text}
\newfontfamily{\cyrillicfonttt}{Courier New}
\newfontfamily{\cyrillicfontsf}{STIX Two Text}

\renewcommand{\thefigure}{\thesection.\arabic{figure}}
\renewcommand{\thetable}{\thesection.\arabic{table}}
\numberwithin{equation}{section}

\renewcommand{\qedsymbol}{$\blacksquare$}
\theoremstyle{definition}
\newtheorem{definition}{Опр.}[section]
\theoremstyle{remark}
\newtheorem{statement}{Утв.}[section]
\theoremstyle{plain}
\newtheorem{theorem}{Теор.}[section]

\graphicspath{{./img/}}
\everymath{\displaystyle}

\newcommand{\llabel}[1]{\label{\thesubsection:#1}}
\newcommand{\lref}[1]{\ref{\thesubsection:#1}}


\begin{document}
\paragraph{34}
Исходя из обобщённого соотношения неопределённостей Хайзенберга, получите соотношение неопределённостей «время-энергия». Объясните, почему название соотношения не вполне удачно.\\

\textit{Обобщенное соотношение неопределенностей Хайзенберга:}
$$\Delta A \Delta B \ge \left| \Big \langle \frac { \lbrack \widehat{A}, \widehat{B} \rbrack}{2} \Big \rangle \right| $$
В обобщенное соотношение вместо $\widehat{B}$ подставим гамильтониан:
 $$\Delta A \Delta E \ge \left| \Big \langle \frac { \lbrack \widehat{A}, \widehat{H} \rbrack}{2} \Big \rangle \right|$$

Если $\widehat{A}$ не зависит от времени, то $\lbrack \widehat{A}, \widehat{H} \rbrack$ полностью определяет скорость эволюции во времени
квантовомеханического среднего $A$:\\

$$\frac{d}{dt}\langle A \rangle = \frac{1}{2 \hbar} \langle \lbrack \widehat{A}, \widehat{H} \rbrack \rangle \Rightarrow \left| \langle \widehat{A}, \widehat{H}\rangle \right| = \hbar \left| \frac{d}{dt}\langle A \rangle \right|$$

Тогда: $\frac{\Delta A}{\left| \frac{d}{dt} \langle A \rangle \right| } \Delta E \ge \frac{\hbar}{2}$

где $\frac{\Delta A}{\left| \frac{d}{dt} \langle A \rangle \right| } = \Sigma_A $ - характеристическое время эволюции статистического распределения А. Это время, необходимое для того, чтобы центр $\langle A \rangle$ распределения сместился на ширину распределения $\Delta A$ 

Название соотношение не вполне удачно по причине того, что оно может запутать, так как под временем тут понимается характеристическое время эволюции. 

\end{document}