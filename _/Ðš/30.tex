\documentclass[12pt]{article}

\usepackage[a4paper,
            total={170mm,255mm},
            left=10mm,
            top=15mm]
            {geometry}

\usepackage{fontspec,
            polyglossia,
            graphicx}

\usepackage{amsmath,
            amsthm,
            amssymb}

\usepackage{unicode-math,
            tensor}

\usepackage{wrapfig,
            hyperref,
            multicol,
            multirow,
            tabularx,
            booktabs,
            subfiles}

\setdefaultlanguage{russian}
\setotherlanguage{english}
\setkeys{russian}{babelshorthands=true}

\defaultfontfeatures{Ligatures=TeX}
\setmainfont{STIX Two Text}
\setmathfont{STIX Two Math}
\DeclareSymbolFont{letters}{\encodingdefault}{\rmdefault}{m}{it}

\newfontfamily{\cyrillicfont}{STIX Two Text} 
\newfontfamily{\cyrillicfontrm}{STIX Two Text}
\newfontfamily{\cyrillicfonttt}{Courier New}
\newfontfamily{\cyrillicfontsf}{STIX Two Text}

\renewcommand{\thefigure}{\thesection.\arabic{figure}}
\renewcommand{\thetable}{\thesection.\arabic{table}}
\numberwithin{equation}{section}

\renewcommand{\qedsymbol}{$\blacksquare$}
\theoremstyle{definition}
\newtheorem{definition}{Опр.}[section]
\theoremstyle{remark}
\newtheorem{statement}{Утв.}[section]
\theoremstyle{plain}
\newtheorem{theorem}{Теор.}[section]

\graphicspath{{./img/}}
\everymath{\displaystyle}

\newcommand{\llabel}[1]{\label{\thesubsection:#1}}
\newcommand{\lref}[1]{\ref{\thesubsection:#1}}


\begin{document}

\paragraph{30}
Временная эволюция классической величины и временная эволюция квантовомеханического среднего. Интеграл движения в классической и квантовой механике.\\

Будем иметь дело только с временной функцией, нормированной на единицу, т.е.
 $\|\psi(t_{0}, \vec{r})\|^{2}=1
 \Rightarrow\langle A \rangle =(\psi, \hat{A}\psi)$

Исследуем как квантомеханическое среднее эволюционирует по времени:

\begin{flalign*}
\begin{split}
\frac{d}{dt} \langle A(t) \rangle
&=
\left.
\frac{d}{dt}(\psi(t,\vec{r}),\hat{A}\psi(t,\vec{r}))=(\partial_{t}\psi,\hat{A}\psi)+(\psi,(\partial_{t}\hat{A})\psi)+(\psi,(\hat{A})\partial_{t}\psi)
\right|_{\partial_{t}\psi=\frac{1}{i\hbar}\hat{H}\psi}
=\\
&=(\psi,(\partial_{t}\hat{A})\psi)+\frac{1}{i\hbar}\hat{H}\psi((\psi , \hat{A}\hat{H}\psi)-(\hat{H}\psi, \hat{A}\psi)).
\end{split}
\end{flalign*}
При этом $(\hat{H}\psi, \hat{A}\psi)=(\psi,\hat{H} \hat{A}\psi)$, не забываем, что $\hat{H}$ -- эрмитов оператор, можем переставлять в скобочках. Также $(\psi,(\hat{A}\hat{H}-\hat{H}\hat{A})\psi)=(\psi,[\hat{A},\hat{H}]\psi).$ $\langle A H \rangle = (\psi,[\hat{A},\hat{H}]\psi).$
Тогда  

\begin{gather}
\llabel{_30:final}
\frac{d}{dt}\langle A(t) \rangle = \langle \partial_{t}\hat{A} \rangle + \frac{1}{i\hbar}\langle [\hat{A},\hat{H}] \rangle
\end{gather}
В (\lref{_30:final}) собственно показана эволюция квантомеханического среднего. 

Если нет явной зависимости физической величины $A$ от времени, то :

$$
\frac{d}{dt}\langle A(t) \rangle =\frac{1}{i\hbar}\langle [\hat{A},\hat{H}] \rangle
$$

Если теперь выясняется, что $ [\hat{A},\hat{H}]=0$, то интеграл движения предтавляется в виде:

\begin{gather*}
\frac{d}{dt}\langle A(t) \rangle=0.
\end{gather*}

\end{document}