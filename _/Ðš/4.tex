\documentclass[12pt]{article}

\usepackage[a4paper,
            total={170mm,255mm},
            left=10mm,
            top=15mm]
            {geometry}

\usepackage{fontspec,
            polyglossia,
            graphicx}

\usepackage{amsmath,
            amsthm,
            amssymb}

\usepackage{unicode-math,
            tensor}

\usepackage{wrapfig,
            hyperref,
            multicol,
            multirow,
            tabularx,
            booktabs,
            subfiles}

\setdefaultlanguage{russian}
\setotherlanguage{english}
\setkeys{russian}{babelshorthands=true}

\defaultfontfeatures{Ligatures=TeX}
\setmainfont{STIX Two Text}
\setmathfont{STIX Two Math}
\DeclareSymbolFont{letters}{\encodingdefault}{\rmdefault}{m}{it}

\newfontfamily{\cyrillicfont}{STIX Two Text} 
\newfontfamily{\cyrillicfontrm}{STIX Two Text}
\newfontfamily{\cyrillicfonttt}{Courier New}
\newfontfamily{\cyrillicfontsf}{STIX Two Text}

\renewcommand{\thefigure}{\thesection.\arabic{figure}}
\renewcommand{\thetable}{\thesection.\arabic{table}}
\numberwithin{equation}{section}

\renewcommand{\qedsymbol}{$\blacksquare$}
\theoremstyle{definition}
\newtheorem{definition}{Опр.}[section]
\theoremstyle{remark}
\newtheorem{statement}{Утв.}[section]
\theoremstyle{plain}
\newtheorem{theorem}{Теор.}[section]

\graphicspath{{./img/}}
\everymath{\displaystyle}

\newcommand{\llabel}[1]{\label{\thesubsection:#1}}
\newcommand{\lref}[1]{\ref{\thesubsection:#1}}


\begin{document}
	
	\paragraph{4.}
	Энергетическая светимость, испускательная способность, поглощательная способность. Закон Кирхгофа.
	
	\begin{definition}
		Энергетическая светимость $R$ (интегральная плотность потока энергии излучения) — это энергия, испускаемая единицей площади поверхности тела в единицу времени.
		
		$$R = \int_{0}^{\infty} r_\omega d\omega$$
	\end{definition}
	
	\begin{definition}
		Испускательная способность тела $r_\omega$ (спектральная плотность потока энергии излучения) — это количество энергии, испускаемой единицей площади поверхности тела в единицу времени единичном интервале частот.
	\end{definition}
	
	\begin{definition}
		Поглощательная способность $α_ω$  (спектральный коэффициент поглощения) — это отношение энергии поглощенной поверхностью тела к энергии, падающей на поверхность тела. Обе энергии (падающая и поглощенная) берутся в расчете на единицу площади поверхности тела, единицу времени и единичный интервал частот.
	\end{definition}
	
	\textbf{Закон Кирхгофа}
	\begin{statement}
		Отношение излучательной способности любого тела к его поглощательной способности одинаково для всех тел при данной температуре для данной частоты и не зависит от их формы и химической природы.
	\end{statement}

	Т.е. отношение испускательной и поглощательной способностей не зависит от природы тела и является универсальной функцией частоты (длины волны) и температуры:
	$$\frac{1}{4} C U_\omega (T) = \frac{r_\omega}{a_\omega}$$
\end{document}