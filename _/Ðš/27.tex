\documentclass[12pt]{article}

\usepackage[a4paper,
            total={170mm,255mm},
            left=10mm,
            top=15mm]
            {geometry}

\usepackage{fontspec,
            polyglossia,
            graphicx}

\usepackage{amsmath,
            amsthm,
            amssymb}

\usepackage{unicode-math,
            tensor}

\usepackage{wrapfig,
            hyperref,
            multicol,
            multirow,
            tabularx,
            booktabs,
            subfiles}

\setdefaultlanguage{russian}
\setotherlanguage{english}
\setkeys{russian}{babelshorthands=true}

\defaultfontfeatures{Ligatures=TeX}
\setmainfont{STIX Two Text}
\setmathfont{STIX Two Math}
\DeclareSymbolFont{letters}{\encodingdefault}{\rmdefault}{m}{it}

\newfontfamily{\cyrillicfont}{STIX Two Text} 
\newfontfamily{\cyrillicfontrm}{STIX Two Text}
\newfontfamily{\cyrillicfonttt}{Courier New}
\newfontfamily{\cyrillicfontsf}{STIX Two Text}

\renewcommand{\thefigure}{\thesection.\arabic{figure}}
\renewcommand{\thetable}{\thesection.\arabic{table}}
\numberwithin{equation}{section}

\renewcommand{\qedsymbol}{$\blacksquare$}
\theoremstyle{definition}
\newtheorem{definition}{Опр.}[section]
\theoremstyle{remark}
\newtheorem{statement}{Утв.}[section]
\theoremstyle{plain}
\newtheorem{theorem}{Теор.}[section]

\graphicspath{{./img/}}
\everymath{\displaystyle}

\newcommand{\llabel}[1]{\label{\thesubsection:#1}}
\newcommand{\lref}[1]{\ref{\thesubsection:#1}}


\begin{document}
\paragraph{27} Запишите и прокомментируйте соотношения неопределённостей для $\hat{p}_x$ и $\hat{p}_y$, а также $\hat{L}_x$, $\hat{L}_y$\\
Для $\hat{p}_x$ и $\hat{p}_y$ соотношения неопределенности (которые я нашел в лекциях, да и в гугле) это соотношения неопределенности Гейзенберга, а именно:
$$\Delta x \Delta p_x\ge \hbar$$
$$\Delta y \Delta p_y\ge \hbar$$
Эти славные отношения по факту определяют границу классической механики...\\
Задачу можно рассматривать в классической механике, если неравенства Гейзенберга никак не влияют на результат\\
А если неравенства на результат влияют, то это уже квантмех.\\
Есть такая формулировка: чем точнее измеряется одна характеристика частицы, тем менее точно можно измерить вторую.\\
Не от Никифорова, зато понятно...\\
Для  $\hat{L}_x$, $\hat{L}_y$ это будет :\\
$$\Delta L_x \Delta L_y\ge \frac{\hbar}{2}\left<L_z\right>$$
Уникальность данного неравенства в том, что справа стоит не число, а среднее значение оператора.\\
Из неравенства следует, что нельзя одновременно уточнить (измерить точно) сразу три компоненты Момента Импульса. Это означает, что невозможно узнать Направление момента импульса!\\
\end{document}