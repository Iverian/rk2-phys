\documentclass[12pt]{article}

\usepackage[a4paper,
            total={170mm,255mm},
            left=10mm,
            top=15mm]
            {geometry}

\usepackage{fontspec,
            polyglossia,
            graphicx}

\usepackage{amsmath,
            amsthm,
            amssymb}

\usepackage{unicode-math,
            tensor}

\usepackage{wrapfig,
            hyperref,
            multicol,
            multirow,
            tabularx,
            booktabs,
            subfiles}

\setdefaultlanguage{russian}
\setotherlanguage{english}
\setkeys{russian}{babelshorthands=true}

\defaultfontfeatures{Ligatures=TeX}
\setmainfont{STIX Two Text}
\setmathfont{STIX Two Math}
\DeclareSymbolFont{letters}{\encodingdefault}{\rmdefault}{m}{it}

\newfontfamily{\cyrillicfont}{STIX Two Text} 
\newfontfamily{\cyrillicfontrm}{STIX Two Text}
\newfontfamily{\cyrillicfonttt}{Courier New}
\newfontfamily{\cyrillicfontsf}{STIX Two Text}

\renewcommand{\thefigure}{\thesection.\arabic{figure}}
\renewcommand{\thetable}{\thesection.\arabic{table}}
\numberwithin{equation}{section}

\renewcommand{\qedsymbol}{$\blacksquare$}
\theoremstyle{definition}
\newtheorem{definition}{Опр.}[section]
\theoremstyle{remark}
\newtheorem{statement}{Утв.}[section]
\theoremstyle{plain}
\newtheorem{theorem}{Теор.}[section]

\graphicspath{{./img/}}
\everymath{\displaystyle}

\newcommand{\llabel}[1]{\label{\thesubsection:#1}}
\newcommand{\lref}[1]{\ref{\thesubsection:#1}}


\begin{document}
	\paragraph{7.}
	Фотоэффект. Эффект Комптона. Интерференция электронов на двух щелях.\\
	
	
	Фотоэффект возникает при взаимодействии вещества с поглощаемым электромагнитным излучением.\\
	
	Различают внешний и внутренний фотоэффект.
	
	\begin{definition}
		Внешним фотоэффектом называется явление вырывания электронов из вещества под действием падающего на него света.
	\end{definition}

	\begin{definition}
		Внутренним фотоэффектом называется явление увеличения концентрации носителей заряда в веществе, а следовательно, и увеличения электропроводности вещества под действием света. Частным случаем внутреннего фотоэффекта является вентильный фотоэффект — явление возникновения под действием света электродвижущей силы в контакте двух различных полупроводников или полупроводника и металла.
	\end{definition}

	Законы фотоэффекта:
	
	\begin{enumerate}
		\item Число фотоэлектронов, вырываемых за 1 секунду с поверхности катода, пропорционально интенсивности света, падающего на это вещество.
		\item Кинетическая энергия фотоэлектронов не зависит от интенсивности падающего света, а зависит линейно от его частоты.
		\item Красная граница фотоэффекта зависит только от рода вещества катода.
		\item Фотоэффект практически безинерционен, так как с момента облучения металла светом до вылета электронов проходит время $\approx 10^{-9}$
	\end{enumerate}
	
	\begin{definition}
		Эффект Комптона – рассеяние электромагнитного излучения на свободном электроне, сопровождающееся уменьшением частоты излучения. В этом процессе электромагнитное излучение ведёт себя как поток отдельных частиц – корпускул (которыми в данном случае являются кванты электромагнитного поля - фотоны), что доказывает двойственную – корпускулярно-волновую природу электромагнитного излучения. С точки зрения классической электродинамики рассеяние излучения с изменением частоты невозможно.
	\end{definition}

	\begin{definition}
		    Комптоновское рассеяние – это рассеяние на свободном электроне отдельного фотона с энергией $E = h\nu = h\frac{c}{\lambda}$ ($h$ – постоянная Планка, $ν$ – частота электромагнитной волны, $λ$ – её длина, $с$ – скорость света) и импульсом $р = \frac{E}{c}$.
	\end{definition}

	Рассеиваясь на покоящемся электроне, фотон передаёт ему часть своей энергии и импульса и меняет направление своего движения. Электрон в результате рассеяния начинает двигаться. Фотон после рассеяния будет иметь энергию $Е' = hν'$ (и частоту) меньшую, чем его энергия (и частота) до рассеяния. Соответственно, после рассеяния длина волны фотона $λ'$ увеличится. Из законов сохранения энергии и импульса следует, что длина волны фотона после рассеяния увеличится на величину $\Delta \lambda = \lambda' - \lambda = \frac{h}{m_e с}(1-\cos\theta)$, где $θ$ – угол рассеяния фотона, а $m_e$ – масса электрона $\frac{h}{m_e c} = 0.024 Å$ называется комптоновской длиной волны электрона.\\
	
	Изменение длины волны при комптоновском рассеянии не зависит от $λ$ и определяется лишь углом $θ$ рассеяния $γ$-кванта. Кинетическая энергия электрона определяется соотношением
	$$E_e = \frac{E_γ}{1+\frac{m_e c^2}{2E_γ \sin^2\frac{\theta}{2}}}$$
	Эффективное сечение рассеяния γ-кванта на электроне не зависит от характеристик вещества поглотителя. Эффективное сечение этого же процесса,$ рассчитанное на один атом$, пропорционально атомному номеру (или числу электронов в атоме) Z.
	Сечение комптоновского рассеяния убывает с ростом энергии $γ$-кванта: $\sigma_k \approx \frac{1}{E_γ}$.
	\\\\\\
	\textbf{Обратный Комптон-эффект}
	
	Если электрон, на котором рассеивается фотон, является ультрарелятивистским $E_e >> E_γ$, то при таком столкновении электрон теряет энергию, а фотон приобретает энергию. Такой процесс рассеяния используется для получения моноэнергетических пучков $γ$-квантов высокой энергии. С этой целью поток фотонов от лазера рассеивают на большие углы на пучке ускоренных электронов высокой энергии, выведенных из ускорителя.\\
	
	Энергия рассеянного фотона $E_γ$ зависит от скорости $V$ ускоренного пучка электронов, энергии $E_{γ_0}$ и угла столкновения $θ$ фотонов лазерного излучения с пучком электронов, угла между $φ$ направлениями движения первичного и рассеянного фотона
	
	$$E_γ = E_{γ_0}\frac{1-\frac{V}{c}\cos\theta}{1-\frac{V}{c}\cos(\theta-\phi)+\frac{E_{γ_0}}{E_0}(1-\cos\phi)}$$
	
	При «лобовом» столкновении
	$$E_{γmax} = E_0 \frac{4E_{γ0} E_0}{4E_{γ0} E_0 + (mc^2)^2}$$
	
	
	$E_{0}$ − полная энергия электрона до взаимодействия, $ mc^2$ − энергия покоя электрона.
	Если направление скоростей начальных фотонов изотропно, то средняя энергия рассеянных фотонов  $\langle E_γ \rangle$  определяется соотношением
	
	$$\langle E_γ \rangle = (\frac{4E_γ}{3})*(\frac{E_e}{mc^2})$$
	
	При рассеянии релятивистских электронов на микроволновом реликтовом излучении образуется изотропное рентгеновское космическое излучение с энергией 
	$E_γ$ = 50–100 кэВ.
	\\\\\\
	\textbf{Интерференция электронов на двух щелях}
	
	Опыт, проведенный Йоннсоном, повторял технику фундаментального опыта Юнга по интерференции. Пучок электронов пропускался через две близлежащие щели, после чего на фотопластинке, установленной за ними, наблюдалась интерференционная картина.
	Электрон стал вести себя как классическая частица, пролетая либо через первую, либо через вторую щель, но не через две одновременно, — образуя на экране только две полосы напротив каждой из щелей. Как только детектор выключали, интерференционная картина восстанавливалась: экран снова становился покрытым семейством интерференционных полос.
	Результаты проведенного эксперимента говорят о следующем. Во-первых, даже отдельно взятый электрон обладает волновыми свойствами, поскольку способен пройти одновременно через две щели и образовать за ними интерференционную картину. Во-вторых, любая попытка определить, через какую из щелей прошел данный электрон, безнадежно нарушает когерентность щелей как источников вторичных электронных волн, в результате чего интерференционная картина исчезает.
	Данное явление свидетельствует о том, что обычное статистическое понятие вероятности неприменимо к микрочастицам. Электрон как бы одновременно, не делясь на части, проходит через обе щели — поэтому его состояние за щелями представляет собой суперпозицию двух состояний электрона, прошедшего через конкретную щель («суперпозиция» дословно означает наложение, одновременное существование). Явление же исчезновения интерференционной картины при установке детектора около одной из щелей называется декогеренцией, поскольку, с точки зрения волновой теории, две вторичных волны, исходящие от щелей, после этого действия теряют когерентность. С точки зрения квантовой теории данное явление также называют коллапсом волновой функции. Из двух одновременно существовавших потенциальностей реализуется только одна — причем так, будто остальных и не существовало.
	
	
\end{document}