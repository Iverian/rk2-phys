\documentclass[12pt]{article}

\usepackage[a4paper,
            total={170mm,255mm},
            left=10mm,
            top=15mm]
            {geometry}

\usepackage{fontspec,
            polyglossia,
            graphicx}

\usepackage{amsmath,
            amsthm,
            amssymb}

\usepackage{unicode-math,
            tensor}

\usepackage{wrapfig,
            hyperref,
            multicol,
            multirow,
            tabularx,
            booktabs,
            subfiles}

\setdefaultlanguage{russian}
\setotherlanguage{english}
\setkeys{russian}{babelshorthands=true}

\defaultfontfeatures{Ligatures=TeX}
\setmainfont{STIX Two Text}
\setmathfont{STIX Two Math}
\DeclareSymbolFont{letters}{\encodingdefault}{\rmdefault}{m}{it}

\newfontfamily{\cyrillicfont}{STIX Two Text} 
\newfontfamily{\cyrillicfontrm}{STIX Two Text}
\newfontfamily{\cyrillicfonttt}{Courier New}
\newfontfamily{\cyrillicfontsf}{STIX Two Text}

\renewcommand{\thefigure}{\thesection.\arabic{figure}}
\renewcommand{\thetable}{\thesection.\arabic{table}}
\numberwithin{equation}{section}

\renewcommand{\qedsymbol}{$\blacksquare$}
\theoremstyle{definition}
\newtheorem{definition}{Опр.}[section]
\theoremstyle{remark}
\newtheorem{statement}{Утв.}[section]
\theoremstyle{plain}
\newtheorem{theorem}{Теор.}[section]

\graphicspath{{./img/}}
\everymath{\displaystyle}

\newcommand{\llabel}[1]{\label{\thesubsection:#1}}
\newcommand{\lref}[1]{\ref{\thesubsection:#1}}


\begin{document}
\paragraph{44}
Принцип дополнительности Бора и соотношение неопределённости Хайзенберга. Постулат квантовой механики об измерениях.\\

Мдемс\\
По порядку: принцип дополнительности Бора гласит:\\
Существуют пары физических величин, называемых дополнительными, принципиальная точность измерения которых ограничена соотношением неопределенности Гейзенберга.\\
Соотношение неопределенности Гейзенберга, откуда и что такое:\\
Поскольку микрообъекты проявляют волновые свойства для них, как и для обычных волн справедливо следующее:
$$\Delta x \Delta k \ge 1$$
Где $\Delta x $ - протяженность цуга, а $\Delta k$ - диапазон волновых чисел.\\
Так как $\vec{p}=\hbar\vec{k}$, мы получаем Неравенства (кол-во неравенств равно размерности пространства) Гейзенберга:
$$\Delta x \Delta p_x\ge\hbar$$
Эти славные отношения по факту определяют границу классической механики...\\
Задачу можно рассматривать в классической механике, если неравенства Гейзенберга никак не влияют на результат\\
А если неравенства на результат влияют, то это уже квантмех.\\
Есть такая формулировка: чем точнее измеряется одна характеристика частицы, тем менее точно можно измерить вторую.\\
Не от Никифорова, зато понятно...\\
А вот сейчас -- огонь.\\
Постулат квантовой механики об измерениях:\\
Пусть измеряется некоторая величина А и ей в соответствие поставили оператор $\hat{A}(A\rightarrow \hat{A})$.\\ Если он Эрмитов ($ \hat{A}\phi_n=a_n\phi_n$), и $\psi$ - функция состояния $\psi(x)=\sum\limits_n C_n \phi_n(x)$\\
Тогда в момент измерения система (в результате неконтролируемого возмущения) неминуемо перейдет в состояние, описываемое собственными функциями оператора $\hat{A}$\\
Причем вероятность перейти в какое-то состояние $\phi_n$ пропорциональна $|C_n|^2$\\
В результате измерения будет получено соответствующее собственное значение\\
\end{document}