\documentclass[12pt]{article}

\usepackage[a4paper,
            total={170mm,255mm},
            left=10mm,
            top=15mm]
            {geometry}

\usepackage{fontspec,
            polyglossia,
            graphicx}

\usepackage{amsmath,
            amsthm,
            amssymb}

\usepackage{unicode-math,
            tensor}

\usepackage{wrapfig,
            hyperref,
            multicol,
            multirow,
            tabularx,
            booktabs,
            subfiles}

\setdefaultlanguage{russian}
\setotherlanguage{english}
\setkeys{russian}{babelshorthands=true}

\defaultfontfeatures{Ligatures=TeX}
\setmainfont{STIX Two Text}
\setmathfont{STIX Two Math}
\DeclareSymbolFont{letters}{\encodingdefault}{\rmdefault}{m}{it}

\newfontfamily{\cyrillicfont}{STIX Two Text} 
\newfontfamily{\cyrillicfontrm}{STIX Two Text}
\newfontfamily{\cyrillicfonttt}{Courier New}
\newfontfamily{\cyrillicfontsf}{STIX Two Text}

\renewcommand{\thefigure}{\thesection.\arabic{figure}}
\renewcommand{\thetable}{\thesection.\arabic{table}}
\numberwithin{equation}{section}

\renewcommand{\qedsymbol}{$\blacksquare$}
\theoremstyle{definition}
\newtheorem{definition}{Опр.}[section]
\theoremstyle{remark}
\newtheorem{statement}{Утв.}[section]
\theoremstyle{plain}
\newtheorem{theorem}{Теор.}[section]

\graphicspath{{./img/}}
\everymath{\displaystyle}

\newcommand{\llabel}[1]{\label{\thesubsection:#1}}
\newcommand{\lref}[1]{\ref{\thesubsection:#1}}


\begin{document}

\paragraph{23}
Покажите, что при унитарных преобразованиях векторов состояний квантовомеханические средние не меняются.\\

По опредедению \textit{квантовомеханическое среднее} наблюдаемой $\hat{A}$ в состоянии $|c\rangle$ выражается как скалярное произведение
\begin{gather*}
\langle A \rangle = \langle c|\hat{A}|c\rangle
\end{gather*}
Рассмотрим унитарное преобразование $\Omega\colon \Omega^\dagger=\Omega^{-1}$, тогда вычислим $\langle A\rangle$ в состоянии $|c'\rangle=\Omega|c\rangle$. Вспомним, что $\langle c'|=\left(|c'\rangle\right)^\dagger=\langle c|\Omega^\dagger$, тогда $\langle A\rangle$ примет вид:
\begin{gather*}
\langle A \rangle
=
\langle c'|\hat{A}'|c'\rangle
=
\langle c|\Omega^\dagger \hat{A}' \Omega|c\rangle
=
\langle c|\Omega^\dagger \Omega\hat{A}\Omega^\dagger \Omega|c\rangle
=
\langle c|\hat{A}|c\rangle
\end{gather*}
Вообще, унитарное преобразование на линейном пространстве над комплесным полем -- аналог ортогонального преобразования на пространстве над действительным полем. Можно показать, что унитарные преобразования сохраняют скалярное произведение (с учетом наших магических бра-кет обозначений это элементарно)
\begin{gather*}
\begin{cases}
|a'\rangle = \Omega|a\rangle\\
|b'\rangle = \Omega|b\rangle\\
\end{cases}
\Longrightarrow
\langle a'|b'\rangle
=
\langle a|\Omega^\dagger \Omega |b\rangle
=
\langle a|b\rangle
\end{gather*}

\end{document}