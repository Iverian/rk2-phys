\documentclass[12pt]{article}

\usepackage[a4paper,
            total={170mm,255mm},
            left=10mm,
            top=15mm]
            {geometry}

\usepackage{fontspec,
            polyglossia,
            graphicx}

\usepackage{amsmath,
            amsthm,
            amssymb}

\usepackage{unicode-math,
            tensor}

\usepackage{wrapfig,
            hyperref,
            multicol,
            multirow,
            tabularx,
            booktabs,
            subfiles}

\setdefaultlanguage{russian}
\setotherlanguage{english}
\setkeys{russian}{babelshorthands=true}

\defaultfontfeatures{Ligatures=TeX}
\setmainfont{STIX Two Text}
\setmathfont{STIX Two Math}
\DeclareSymbolFont{letters}{\encodingdefault}{\rmdefault}{m}{it}

\newfontfamily{\cyrillicfont}{STIX Two Text} 
\newfontfamily{\cyrillicfontrm}{STIX Two Text}
\newfontfamily{\cyrillicfonttt}{Courier New}
\newfontfamily{\cyrillicfontsf}{STIX Two Text}

\renewcommand{\thefigure}{\thesection.\arabic{figure}}
\renewcommand{\thetable}{\thesection.\arabic{table}}
\numberwithin{equation}{section}

\renewcommand{\qedsymbol}{$\blacksquare$}
\theoremstyle{definition}
\newtheorem{definition}{Опр.}[section]
\theoremstyle{remark}
\newtheorem{statement}{Утв.}[section]
\theoremstyle{plain}
\newtheorem{theorem}{Теор.}[section]

\graphicspath{{./img/}}
\everymath{\displaystyle}

\newcommand{\llabel}[1]{\label{\thesubsection:#1}}
\newcommand{\lref}[1]{\ref{\thesubsection:#1}}


\begin{document}
\paragraph{47}
Покажите, что переход от одного представления к другому производится с помощью унитарного преобразования.\\

\begin{definition}
Оператор $\hat{\Omega}$ называется унитарным, если $\hat{\Omega}^+=\hat{\Omega}^{-1}$
\end{definition}

Рассмотрим кет-векторы $\left|v\right>$ и $\left|w\right>$:

$$\left|v^{'}\right> = \hat{\Omega}\left|v\right>$$
$$\left|w^{'}\right> = \hat{\Omega}\left|w\right>$$
$$\left<w^{'}\right| = \left<w\right|\hat{\Omega}^{+}$$

Так как $\hat{\Omega}$ - унитарный оператор, то $\hat{\Omega}^{+}\hat{\Omega} = 1$.
Следовательно:

$$\left<w^{'}\vert v^{'}\right> = \left<w\vert\hat{\Omega}^{+}\hat{\Omega}\vert v\right> = \left<w\vert v\right>$$

\end{document}