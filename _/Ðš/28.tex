\documentclass[12pt]{article}

\usepackage[a4paper,
            total={170mm,255mm},
            left=10mm,
            top=15mm]
            {geometry}

\usepackage{fontspec,
            polyglossia,
            graphicx}

\usepackage{amsmath,
            amsthm,
            amssymb}

\usepackage{unicode-math,
            tensor}

\usepackage{wrapfig,
            hyperref,
            multicol,
            multirow,
            tabularx,
            booktabs,
            subfiles}

\setdefaultlanguage{russian}
\setotherlanguage{english}
\setkeys{russian}{babelshorthands=true}

\defaultfontfeatures{Ligatures=TeX}
\setmainfont{STIX Two Text}
\setmathfont{STIX Two Math}
\DeclareSymbolFont{letters}{\encodingdefault}{\rmdefault}{m}{it}

\newfontfamily{\cyrillicfont}{STIX Two Text} 
\newfontfamily{\cyrillicfontrm}{STIX Two Text}
\newfontfamily{\cyrillicfonttt}{Courier New}
\newfontfamily{\cyrillicfontsf}{STIX Two Text}

\renewcommand{\thefigure}{\thesection.\arabic{figure}}
\renewcommand{\thetable}{\thesection.\arabic{table}}
\numberwithin{equation}{section}

\renewcommand{\qedsymbol}{$\blacksquare$}
\theoremstyle{definition}
\newtheorem{definition}{Опр.}[section]
\theoremstyle{remark}
\newtheorem{statement}{Утв.}[section]
\theoremstyle{plain}
\newtheorem{theorem}{Теор.}[section]

\graphicspath{{./img/}}
\everymath{\displaystyle}

\newcommand{\llabel}[1]{\label{\thesubsection:#1}}
\newcommand{\lref}[1]{\ref{\thesubsection:#1}}

\begin{document}
\paragraph{28}
Обобщенное соотношение неопределенностей Хайзенберга. Примеры использоания.\\

 
В ряде случаев в квантовой механике оказывается невозможным одновременно охарактеризовать частицу ее положением в пространстве (координатами) и скоростью (или импульсом). Так например электрон не может иметь одновременно определенных точных значений координаты $x$ и вектора импульса $p_{x}$. Неопределенности значений  $x$ и $p_{x}$ имеют вид:

\begin{gather}
\llabel{_28:neop}
\Delta x\Delta p_{x}\geq h
\end{gather}
Из \lref{_28:neop} следует, что чем меньше неопределенность одной величины (x или   ), тем больше неопределенность другой.
Соотношение, аналогичное \lref{_28:neop}, имеет место для y и  $p_{y}$  , для z и  $p_{z}$  , а также для других пар величин (в классической механике такие пары называются канонически сопряженными). Обозначив канонически сопряженные величины буквами A и B, можно записать:


\begin{gather}
\llabel{_28:heiz}
\Delta A\Delta B\geq h
\end{gather}

Утверждение \lref{_28:heiz} о том, что произведение неопределенностей значений двух сопряженных переменных не может быть по порядку меньше постоянной Планка h, \textbf{называется соотношением неопределенностей Гейзенберга}.

\textbf{Пример использования 1:}

\textit{Энергия и время} являются канонически сопряженными величинами, поэтому для них верно:

\begin{gather}
\Delta E\Delta t\geq h
\end{gather}

Это соотношение означает, что определение энергии с точностью   должно занять интервал времени, равный, по меньшей мере,

$$
   \Delta t \sim \frac{h}{\Delta E}
$$

\textbf{Пример использования 2 :}

 Соотношение неопределенностей указывает, в какой мере возможно пользоваться понятиями классической механики применительно к микрочастицам, в частности с какой степенью точности можно говорить о траекториях микрочастиц. Движение по траектории характеризуется вполне определенными значениями координат и скорости в каждый момент времени. Подставив в \lref{_28:neop} вместо $p_{x}$  произведение m$v_x$  , получим соотношение:
 
 \begin{gather}
 \Delta x\Delta u_{x}\geq \frac{h}{m}
 \end{gather}
 
  Из этого соотношения следует, что чем больше масса частицы, тем меньше неопределенности ее координаты и скорости, следовательно тем с большей точностью можно применять к этой частице понятие траектории. 
 

\end{document}