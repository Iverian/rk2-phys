\documentclass[12pt]{article}

\usepackage[a4paper,
            total={170mm,255mm},
            left=10mm,
            top=15mm]
            {geometry}

\usepackage{fontspec,
            polyglossia,
            graphicx}

\usepackage{amsmath,
            amsthm,
            amssymb}

\usepackage{unicode-math,
            tensor}

\usepackage{wrapfig,
            hyperref,
            multicol,
            multirow,
            tabularx,
            booktabs,
            subfiles}

\setdefaultlanguage{russian}
\setotherlanguage{english}
\setkeys{russian}{babelshorthands=true}

\defaultfontfeatures{Ligatures=TeX}
\setmainfont{STIX Two Text}
\setmathfont{STIX Two Math}
\DeclareSymbolFont{letters}{\encodingdefault}{\rmdefault}{m}{it}

\newfontfamily{\cyrillicfont}{STIX Two Text} 
\newfontfamily{\cyrillicfontrm}{STIX Two Text}
\newfontfamily{\cyrillicfonttt}{Courier New}
\newfontfamily{\cyrillicfontsf}{STIX Two Text}

\renewcommand{\thefigure}{\thesection.\arabic{figure}}
\renewcommand{\thetable}{\thesection.\arabic{table}}
\numberwithin{equation}{section}

\renewcommand{\qedsymbol}{$\blacksquare$}
\theoremstyle{definition}
\newtheorem{definition}{Опр.}[section]
\theoremstyle{remark}
\newtheorem{statement}{Утв.}[section]
\theoremstyle{plain}
\newtheorem{theorem}{Теор.}[section]

\graphicspath{{./img/}}
\everymath{\displaystyle}

\newcommand{\llabel}[1]{\label{\thesubsection:#1}}
\newcommand{\lref}[1]{\ref{\thesubsection:#1}}


\begin{document}
	
\paragraph{5.}
Энергетический спектр квантомеханического гармонического осциллятора.\\

Рассмотрим классический гармонический осциллятор: его полная механическая энергия примет вид
\begin{gather*}
E = \frac{kx^2}{2}+\frac{p^2}{2m},
\end{gather*}
для перехода к квантовому аналогу заменим жесткость пружины $k$ (какие еще пружины в микромире?) на ее выражение через собственную частоту $\omega$: $k=m\omega^2$, теперь по аналогии с классическим выражением запишем квантовомеханический гамильтониан (оператор полной энергии системы):
\begin{gather*}
\hat{H} = \frac{m\omega^2\hat{x}^2}{2}+\frac{\hat{p}^2}{2m}
\end{gather*}
Теперь для нахождения энергетического спектра нужно найти собственные значения $\hat{H}$:
\begin{gather*}
\hat{H}|v\rangle = E|v\rangle
\end{gather*}
Для этого введем вспомогательный оператор (\textit{оператор числа элементарный возбуждений}) $\hat{n}=\hat{b}^\dagger\hat{b}$, где $\hat{b}$ такой, что $[\hat{b},\hat{b}^\dagger]=1$.

Заметим, что гамильтониан $\hat{H}$ можно выразить через оператор $\hat{n}$, положим:
\begin{gather*}
\hat{b} = \sqrt{\frac{m\omega}{2\hbar}}\hat{x} + i\sqrt{\frac{1}{2m\omega\hbar}}\hat{p}
\end{gather*}
Т.к. $\hat{x}$ и $\hat{p}$ эрмитово самосопряженные, то
\begin{gather*}
\hat{b} = \sqrt{\frac{m\omega}{2\hbar}}\hat{x} - i\sqrt{\frac{1}{2m\omega\hbar}}\hat{p}
\end{gather*}
Проверим $[\hat{b},\hat{b}^\dagger]=1$:
\begin{gather*}
[\hat{b},\hat{b}^\dagger]
=
\left.
-\frac{i}{2\hbar}[\hat{x},\hat{p}]+\frac{i}{2\hbar}[\hat{p},\hat{x}]
\right|_{[\hat{x},\hat{p}]=-[\hat{p},\hat{x}]}
=
\left.
-\frac{i}{\hbar}[\hat{x},\hat{p}]
\right|_{[\hat{x},\hat{p}]=i\hbar}
=
1
\end{gather*}
Оператор $\hat{n}$ примет вид:
\begin{gather*}
\hat{n}=\hat{b}^\dagger\hat{b}=\frac{m\omega\hat{x}^2}{2\hbar}+\frac{\hat{p}^2}{2m\omega\hbar}-\frac{1}{2}
\end{gather*}
Тогда получим:
\begin{gather*}
\hbar\omega\left(\hat{n}+\frac{1}{2}\right)
=
\frac{m\omega^2\hat{x}^2}{2}+\frac{\hat{p}^2}{2m}
=
\hat{H}
\end{gather*}
Ключевое свойство оператора $\hat{n}$: его собственные значения являются натуральными числами, обозначим $|n\rangle$ -- СФ оператора $\hat{n}$, отвечающая СЗ $n$. Следовательно:
\begin{gather*}
\hat{H}|n\rangle
=
\hbar\omega\left(\hat{n}+\frac{1}{2}\right)|n\rangle
=
\hbar\omega\left(n+\frac{1}{2}\right)|n\rangle
=
E_n|n\rangle
\end{gather*}
Т.е. собственные функции оператора $\hat{H}$ совпадают с СФ $\hat{n}$, а его собственные значения (стационарные значения энергии осциллятора)
$$
\forall n\in\mathbb{N}\cup\{0\}\colon E_n=\hbar\omega\left(n+\frac{1}{2}\right)
$$

\end{document}