\documentclass[12pt]{article}

\usepackage[a4paper,
            total={170mm,255mm},
            left=10mm,
            top=15mm]
            {geometry}

\usepackage{fontspec,
            polyglossia,
            graphicx}

\usepackage{amsmath,
            amsthm,
            amssymb}

\usepackage{unicode-math,
            tensor}

\usepackage{wrapfig,
            hyperref,
            multicol,
            multirow,
            tabularx,
            booktabs,
            subfiles}

\setdefaultlanguage{russian}
\setotherlanguage{english}
\setkeys{russian}{babelshorthands=true}

\defaultfontfeatures{Ligatures=TeX}
\setmainfont{STIX Two Text}
\setmathfont{STIX Two Math}
\DeclareSymbolFont{letters}{\encodingdefault}{\rmdefault}{m}{it}

\newfontfamily{\cyrillicfont}{STIX Two Text} 
\newfontfamily{\cyrillicfontrm}{STIX Two Text}
\newfontfamily{\cyrillicfonttt}{Courier New}
\newfontfamily{\cyrillicfontsf}{STIX Two Text}

\renewcommand{\thefigure}{\thesection.\arabic{figure}}
\renewcommand{\thetable}{\thesection.\arabic{table}}
\numberwithin{equation}{section}

\renewcommand{\qedsymbol}{$\blacksquare$}
\theoremstyle{definition}
\newtheorem{definition}{Опр.}[section]
\theoremstyle{remark}
\newtheorem{statement}{Утв.}[section]
\theoremstyle{plain}
\newtheorem{theorem}{Теор.}[section]

\graphicspath{{./img/}}
\everymath{\displaystyle}

\newcommand{\llabel}[1]{\label{\thesubsection:#1}}
\newcommand{\lref}[1]{\ref{\thesubsection:#1}}


\begin{document}
\paragraph{35}
Сформулируйте постулат квантовой механики об измерениях. Покажите, что собственные функции эрмитова оператора, отвечающие различным его собственным значениям, ортогональны.
\\

\textit{Постулат квантовой механики об измерениях:}
Пусть измеряется некоторая величина $A$ и ей в соответствие ставится оператор $\hat{A}$. Если он эрмитов ${\hat{A}^+=\hat{A}}$; $\phi_n(x)$ - его собственные функции, и функция состояния, в котором находится система, задана как:
$$\psi(x)=\sum_{n=0}c_n\phi_n(x)$$
Тогда в момент измерения система перейдет в состояние, описываемое собственной функцией оператора $\hat{A}$, причем вероятность перейти в состояние $\phi_n(x)$ пропорциональна ${\left|c_n\right|}^2$. В результате измерения получится соответствующие собственное значение оператора.

	\begin{definition}
		Эрмитовым оператором наз-ся линейный и самосопряженный оператор.
	\end{definition}
	
	СФ самосопряженного оператора $\hat F$ ортогональны, то есть:
	\begin{gather*}
	\int \psi^{\star}_m(x) \psi_n(x)dx = 0\text{ при }m\ne n
	\end{gather*}
	
	Запишем два уравнения двух различных собственных значений $f_n$ и $f_m$, взяв от второго комплексное сопряжение:
	\begin{gather*}
	\hat{F}\psi_n(x)=f_n\psi_n(x),\\
	\hat{F}^\star \psi^\star_m(x)=f_m\psi^\star_m(x).
	\end{gather*}
	
	Первое ур-е умножим слева на $\psi^\star_m(x)$, второе - на $\psi_n(x)$, затем проинтегрируем и вычтем из первого ур-я второе:
	\begin{gather*}
	\int \psi^\star_m(x)\hat{F}\psi_n(x)dx - 
	\int \psi_n(x)\hat{F}^\star \psi^\star_m(x)dx=
	(f_n-f_m)\int \psi^\star_m(x)\psi_n(x)dx.
	\end{gather*}
	
	Левая часть ур-я равна нулю, поскольку вычитаемое сводится к уменьшаемому после опреции транспонирования и использования св-ва самосопряжения оп-ра $\hat{F}$. Учитывая, что $f_n\ne f_m$, получаем:
	\begin{gather*}
	0=\int \psi^\star_m(x)\psi_n(x)dx=
	\langle\psi_m\mid\psi_n\rangle
	\end{gather*}
	
	Объединив результат с условием нормировки, можно написать так:
	\begin{gather*}
	\langle\psi_m\mid\psi_n\rangle=\delta_{mn}
	\end{gather*}

\end{document}