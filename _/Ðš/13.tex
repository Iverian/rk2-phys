\documentclass[12pt]{article}

\usepackage[a4paper,
            total={170mm,255mm},
            left=10mm,
            top=15mm]
            {geometry}

\usepackage{fontspec,
            polyglossia,
            graphicx}

\usepackage{amsmath,
            amsthm,
            amssymb}

\usepackage{unicode-math,
            tensor}

\usepackage{wrapfig,
            hyperref,
            multicol,
            multirow,
            tabularx,
            booktabs,
            subfiles}

\setdefaultlanguage{russian}
\setotherlanguage{english}
\setkeys{russian}{babelshorthands=true}

\defaultfontfeatures{Ligatures=TeX}
\setmainfont{STIX Two Text}
\setmathfont{STIX Two Math}
\DeclareSymbolFont{letters}{\encodingdefault}{\rmdefault}{m}{it}

\newfontfamily{\cyrillicfont}{STIX Two Text} 
\newfontfamily{\cyrillicfontrm}{STIX Two Text}
\newfontfamily{\cyrillicfonttt}{Courier New}
\newfontfamily{\cyrillicfontsf}{STIX Two Text}

\renewcommand{\thefigure}{\thesection.\arabic{figure}}
\renewcommand{\thetable}{\thesection.\arabic{table}}
\numberwithin{equation}{section}

\renewcommand{\qedsymbol}{$\blacksquare$}
\theoremstyle{definition}
\newtheorem{definition}{Опр.}[section]
\theoremstyle{remark}
\newtheorem{statement}{Утв.}[section]
\theoremstyle{plain}
\newtheorem{theorem}{Теор.}[section]

\graphicspath{{./img/}}
\everymath{\displaystyle}

\newcommand{\llabel}[1]{\label{\thesubsection:#1}}
\newcommand{\lref}[1]{\ref{\thesubsection:#1}}


\begin{document}
\paragraph{13} Получите и прокомментируйте обобщенное соотношение неопределённостей Хайзенберга величин $A$ и $B$: $\Delta A\Delta B \ge \frac{\left|\langle\hat{A} | \hat{B}\rangle\right|}{2}$.\\

Квантовая механика позволяет нам определить вероятность того или иного результата эксперимента и средние значения физических величин. Пусть $\hat{A}|j\rangle=a_j|j\rangle$. Тогда если состояние системы определяется вектором $|c\rangle$, то:
$$
\langle A \rangle =\langle c|\hat{A}|c \rangle = \sum_j \langle c | \hat{A}|j\rangle \langle j|c\rangle = \sum_j \langle c| a_j|j\rangle \langle j|c\rangle = \sum_j\langle c|j\rangle \langle | c \rangle a_j=\sum_j |c_j|^2a_j
$$
$|c_j|^2$ --- вероятность того, что если до измерения система находится в состоянии $|c\rangle$, то сразу после измерения она окажется в состоянии $|j\rangle$.\\

Определим неопределенности: $\Delta A \equiv \sigma[A]\equiv \sqrt{\langle (A-\langle A\rangle)^2\rangle} = \sqrt{\langle A^2-2A\langle A \rangle+\langle A \rangle^2\rangle}=\sqrt{\langle A^2\rangle-\langle A\rangle^2}$, $\sigma [A]$ --- среднее квадратичное отклонение случайной величины $A$.\\
$$
(\Delta A)^2=\langle (A-\langle A\rangle )^2\rangle = \langle c|(A-\langle A\rangle)(A-\langle A\rangle )|c=\langle c|(A^+-\langle A\rangle)|d\rangle = \langle d|d\rangle 
$$
Аналогично, $(\Delta B)^2=\langle f| f \rangle$. Неравенство Шварца:
$$
(\Delta A)^2 (\Delta B)^2 = \langle d|d\rangle \langle f|f \rangle \geqslant |\langle d|f\rangle|^2 = \langle d|f\rangle \langle f|d\rangle
$$
$$
\langle d|f \rangle = \langle c|(\hat{A}-\langle A \rangle)(\hat{B}-\langle B) \rangle | c \langle =\langle AB \rangle -\langle A \rangle \langle B \rangle
$$
$$
\langle f|d \rangle = \langle BA \rangle -\langle B \rangle \langle A \rangle
$$
$$
|\langle d|f \rangle |^2 = Re^2 + Im^2 \geqslant Im^2 = \left \langle \frac{[A, B]}{2i}\right \rangle^2 
$$
Мораль: $\Delta A \Delta B \geqslant \left|\left \langle \frac{[\hat{A}, \hat{B}]}{2}\right \rangle \right|$
\end{document}