\documentclass[12pt]{article}

\usepackage[a4paper,
            total={170mm,255mm},
            left=10mm,
            top=15mm]
            {geometry}

\usepackage{fontspec,
            polyglossia,
            graphicx}

\usepackage{amsmath,
            amsthm,
            amssymb}

\usepackage{unicode-math,
            tensor}

\usepackage{wrapfig,
            hyperref,
            multicol,
            multirow,
            tabularx,
            booktabs,
            subfiles}

\setdefaultlanguage{russian}
\setotherlanguage{english}
\setkeys{russian}{babelshorthands=true}

\defaultfontfeatures{Ligatures=TeX}
\setmainfont{STIX Two Text}
\setmathfont{STIX Two Math}
\DeclareSymbolFont{letters}{\encodingdefault}{\rmdefault}{m}{it}

\newfontfamily{\cyrillicfont}{STIX Two Text} 
\newfontfamily{\cyrillicfontrm}{STIX Two Text}
\newfontfamily{\cyrillicfonttt}{Courier New}
\newfontfamily{\cyrillicfontsf}{STIX Two Text}

\renewcommand{\thefigure}{\thesection.\arabic{figure}}
\renewcommand{\thetable}{\thesection.\arabic{table}}
\numberwithin{equation}{section}

\renewcommand{\qedsymbol}{$\blacksquare$}
\theoremstyle{definition}
\newtheorem{definition}{Опр.}[section]
\theoremstyle{remark}
\newtheorem{statement}{Утв.}[section]
\theoremstyle{plain}
\newtheorem{theorem}{Теор.}[section]

\graphicspath{{./img/}}
\everymath{\displaystyle}

\newcommand{\llabel}[1]{\label{\thesubsection:#1}}
\newcommand{\lref}[1]{\ref{\thesubsection:#1}}


\begin{document}
\paragraph{12}
Составить выражение для величины, имеющей размерность длины, используя скорость света $c$, массу частицы $m$, постоянную Планка $h$. Что это за величина?\\

Величина о которой идет речь в вопросе называется комптоновской длиной волны $\lambda_\text{комп}$.\\
Итак, $h$ -- постоянная планка. $\hbar = \frac{h}{2\pi}$. $c$ -- скорость света, $m$ -- масса частицы.\\
 Комптоновская длины волны определяет расстояние, на которое может удалиться виртуальная частица массы $m$ от точки своего рождения. Положение отдельной частицы можно определить с точностью до комптоновской длины волны этой частицы. Для локализации положения частицы массы $m$ её нужно облучать фотонами, имеющими длину волны $\lambda$ меньшую, чем область локализации частицы, что соответствует энергии фотона $Е$.
 $$ \lambda = \frac{h}{p} = \frac{hc}{E} = \frac{hc}{mc^2} = \frac{h}{mc} = \lambda_\text{комп}$$
 $p$ -- импульс.\\
 Фотоны могут рождать в области локализации $\lambda_\text{комп}$ частицы с энергией $E = mc^2$. В области локализации $\lambda_\text{комп}$ частица не может рассматриваться как точечный объект, потому что часть времени она находится в состоянии частицы + пара «частица-античастица». Таким образом, комптоновская длина волны определяет минимальную погрешность, с которой может быть измерена координата частицы. То есть область локализации частицы определена с точностью до её комптоновской длины волны.

\end{document}