\documentclass[12pt]{article}

\usepackage[a4paper,
            total={170mm,255mm},
            left=10mm,
            top=15mm]
            {geometry}

\usepackage{fontspec,
            polyglossia,
            graphicx}

\usepackage{amsmath,
            amsthm,
            amssymb}

\usepackage{unicode-math,
            tensor}

\usepackage{wrapfig,
            hyperref,
            multicol,
            multirow,
            tabularx,
            booktabs,
            subfiles}

\setdefaultlanguage{russian}
\setotherlanguage{english}
\setkeys{russian}{babelshorthands=true}

\defaultfontfeatures{Ligatures=TeX}
\setmainfont{STIX Two Text}
\setmathfont{STIX Two Math}
\DeclareSymbolFont{letters}{\encodingdefault}{\rmdefault}{m}{it}

\newfontfamily{\cyrillicfont}{STIX Two Text} 
\newfontfamily{\cyrillicfontrm}{STIX Two Text}
\newfontfamily{\cyrillicfonttt}{Courier New}
\newfontfamily{\cyrillicfontsf}{STIX Two Text}

\renewcommand{\thefigure}{\thesection.\arabic{figure}}
\renewcommand{\thetable}{\thesection.\arabic{table}}
\numberwithin{equation}{section}

\renewcommand{\qedsymbol}{$\blacksquare$}
\theoremstyle{definition}
\newtheorem{definition}{Опр.}[section]
\theoremstyle{remark}
\newtheorem{statement}{Утв.}[section]
\theoremstyle{plain}
\newtheorem{theorem}{Теор.}[section]

\graphicspath{{./img/}}
\everymath{\displaystyle}

\newcommand{\llabel}[1]{\label{\thesubsection:#1}}
\newcommand{\lref}[1]{\ref{\thesubsection:#1}}


\begin{document}
\paragraph{36}
Вычислите коммутаторы $\left[\hat{x},\hat{p}_x\right]$, $\left[\hat{L}_z,\hat{L}^2\right]$. Прокомментируйте результаты.\\

\begin{definition}
Коммутатором называется оператор $\left[\hat{A},\hat{B}\right]=\hat{A}\hat{B}-\hat{B}\hat{A}$\\
\end{definition}

Вычисление коммутатора  $\left[\hat{x},\hat{p}_x\right]$:\\

\begin{definition}
$\hat{p}_x$ - оператор компоненты $x$ вектора импульса 
\end{definition}

\begin{definition}
$\hat{x}$ - оператор координаты $x$
\end{definition}

$$\left[\hat{x},\hat{p}_x\right]\psi(x) = \left(\hat{x}\hat{p}_x - \hat{p}_x\hat{x}\right)\psi(x) = \frac{\hbar}{i}\left(x\frac{\partial}{\partial x}-\frac{\partial}{\partial x}x\right)\psi(x)=-\frac{\hbar}{i}\psi(x) = i\hbar\psi(x)$$

Таким образом:  $\left[\hat{x},\hat{p}_x\right] = i\hbar$\\

\begin{definition}
$\hat{L}^2$ - оператор квадрата момента импульса
\end{definition}

\begin{definition}
$\hat{L}_z$ - оператор компоненты z вектора момента импульса 
\end{definition}

Вычисление коммутатора $\left[\hat{L}_z,\hat{L}^2\right]$:

$$\hat{L}^2 = \hat{L}_x^2+\hat{L}_y^2+\hat{L}_z^2$$

$$\left[ \hat{L}_z,\hat{L}_x^2\right]=\left[\hat{L}_z,\hat{L}_x\right]\hat{L}_x+\hat{L}_x\left[\hat{L}_z,\hat{L}_x\right]$$

$$\left[ \hat{L}_z,\hat{L}_y^2\right]=\left[\hat{L}_z,\hat{L}_y\right]\hat{L}_y+\hat{L}_y\left[\hat{L}_z,\hat{L}_y\right]$$

$$\left[ \hat{L}_z,\hat{L}_z^2\right]=0$$\\

Так как $\left[\hat{L}_z,\hat{L}_x\right]=i\hbar\hat{L}_y$ и $\left[\hat{L}_z,\hat{L}_y\right] = -i\hbar\hat{L}_x$, то получаем что $\left[\hat{L}_z,\hat{L}^2\right]=0$\\

Следствием вышеперечисленного является то, что невозможно одновременно точно измерить $x$ и $p_x$, но при этом возможно измерить одновременно величину вектора момента импульса и какую-то одну его проекцию.

\end{document}