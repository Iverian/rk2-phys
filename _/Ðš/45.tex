\documentclass[12pt]{article}

\usepackage[a4paper,
            total={170mm,255mm},
            left=10mm,
            top=15mm]
            {geometry}

\usepackage{fontspec,
            polyglossia,
            graphicx}

\usepackage{amsmath,
            amsthm,
            amssymb}

\usepackage{unicode-math,
            tensor}

\usepackage{wrapfig,
            hyperref,
            multicol,
            multirow,
            tabularx,
            booktabs,
            subfiles}

\setdefaultlanguage{russian}
\setotherlanguage{english}
\setkeys{russian}{babelshorthands=true}

\defaultfontfeatures{Ligatures=TeX}
\setmainfont{STIX Two Text}
\setmathfont{STIX Two Math}
\DeclareSymbolFont{letters}{\encodingdefault}{\rmdefault}{m}{it}

\newfontfamily{\cyrillicfont}{STIX Two Text} 
\newfontfamily{\cyrillicfontrm}{STIX Two Text}
\newfontfamily{\cyrillicfonttt}{Courier New}
\newfontfamily{\cyrillicfontsf}{STIX Two Text}

\renewcommand{\thefigure}{\thesection.\arabic{figure}}
\renewcommand{\thetable}{\thesection.\arabic{table}}
\numberwithin{equation}{section}

\renewcommand{\qedsymbol}{$\blacksquare$}
\theoremstyle{definition}
\newtheorem{definition}{Опр.}[section]
\theoremstyle{remark}
\newtheorem{statement}{Утв.}[section]
\theoremstyle{plain}
\newtheorem{theorem}{Теор.}[section]

\graphicspath{{./img/}}
\everymath{\displaystyle}

\newcommand{\llabel}[1]{\label{\thesubsection:#1}}
\newcommand{\lref}[1]{\ref{\thesubsection:#1}}


\begin{document}

\paragraph{45}
Сформулируйте постулат квантовой механики о физических величинах. Покажите, что все собственные значения эрмитова оператора суть вещественные числа.\\

\textbf{Постулат квантовой механики о физических величинах}\\
Каждой физической величине ставится в соответствие эрмитов оператор обладающий полной системой собственных функций.
\begin{definition}
$A$-эрмитов, если $(Ax,y)=(x,Ay)$	
\end{definition}

\begin{statement}
	Все собственные значения эрмитова оператора $\hat{A}$ - вещественные числа.
\end{statement}
\begin{proof}
	Пусть $\psi_a$ - собственная функция оператора $\hat{A}$, тогда при воздействии оператора $\hat{A}$ на эту функцию получим эту же функцию домноженную на какое-то число $a$:
	\begin{gather*}
		\hat{A}\psi_a = a\psi_a
	\end{gather*}
	Необходимо доказать, что $a$ - вещественное. Для этого воспользуемся утверждением того, что если $\hat{A}$ - эрмитов оператор, то скалярное произведение $(\psi_a,\hat{A}\psi_a)$ - вещественное.
	\begin{gather*}
		(\psi_a,\hat{A}\psi_a) = (\psi_a,a\psi_a) = a(\psi_a,\psi_a)
	\end{gather*}
	Ввиду того, скалярное произведение по определению $(\psi_a,\psi_a) = \vert\vert\psi_a\vert\vert^2$ - вещественное число, то "$a$" ничего не остается как быть только вещественным т.к левая часть уравнения по утверждению выше  - вещественная.
\end{proof}
\end{document}