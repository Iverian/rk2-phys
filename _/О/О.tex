\include{__header__}

\begin{document}

\paragraph{О-01}
Какова интенсивность света в центре дифракционной картины от круглого экрана, если он закрывает всю первую зону Френеля? Интенсивность света в отсутствие экрана равна $I_0$.

\paragraph{О-02}
Общая схема наблюдения интерференции. Когерентные источники. Временная когерентность электромагнитных волн. Длина когерентности. Связь временной когерентности с степенью монохроматичности. Пространственная когерентность электромагнитных волн. Радиус когерентности. Интерференция в тонких пленках.

\paragraph{О-03}
Законы геометрической оптики. Электромагнитные волны. Поперечность электромагнитных волн. Интерференция света: определение, общая схема и условия наблюдения интерференции. Интерференция в тонких пленках.

\paragraph{О-04}
Найти угловое распределение дифракционных минимумов при дифракции на решетке, период которой равен $d$, а ширина щели равна $b$.

\paragraph{О-05}
Точечный источник монохроматического света помещен на расстоянии $a$ от круглой диафрагмы, а экран с противоположной стороны - на расстоянии $b$ от нее. При каких радиусах диафрагмы $r$ центр дифракционных колец, наблюдаемых на экране, будет темным и при каких - светлым, если перпендикуляр, опущенный из источника на плоскость диафрагмы, проходит через ее центр?

\paragraph{О-06}
Объясните возникновение пятна Пуассона-Араго за диском. Почему Вы не видите звезд, расположенных за диском Луны на его фоне?

\paragraph{О-07}
Дифракционная решетка. Основная формула. Главные и второстепенные максимумы. Ширина $m$-го главного максимума. Спектр $n$-го порядка. Угловая дисперсия, критерий Рэлея, разрешающая способность.

\paragraph{О-08}
Законы геометрической оптики. Дифракция света. Принцип Гюйгенса-Френеля. Метод зон Френеля. Классификация дифракционных явлений. Дифракция Фраунгофера на длинной прямоугольной щели: схема наблюдения, влияние ширины щели на дифракционную картину, условия наблюдения дифракции.

\paragraph{О-09}
Цуг и монохроматическая волна. Длина волны, волновой вектор, волновое число. Волновая поверхность, фронт волны, фазовая скорость. Кинематические уравнения плоской и сферической монохроматических волн.

\end{document}
