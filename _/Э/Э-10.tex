\documentclass[12pt]{article}

\usepackage{fontspec}
\usepackage{polyglossia}
\usepackage{geometry}
\usepackage{graphicx}

\usepackage{amsmath,
            amsthm,
            amssymb
}

\usepackage{unicode-math,
            tensor
}

\usepackage{wrapfig, 
            hyperref,
            multicol,
            multirow,
            tabularx,
            booktabs,
            makecell
}

\usepackage{}

\geometry{a4paper,
          total={170mm,255mm},
          left=10mm,
          top=15mm,
}

\setdefaultlanguage{russian}
\setotherlanguage{english}
\setkeys{russian}{babelshorthands=true}

\defaultfontfeatures{Ligatures=TeX}
\setmainfont{STIX Two Text}
\setmathfont{STIX Two Math}
\DeclareSymbolFont{letters}{\encodingdefault}{\rmdefault}{m}{it}

\newfontfamily{\cyrillicfont}{STIX Two Text} 
\newfontfamily{\cyrillicfontrm}{STIX Two Text}
\newfontfamily{\cyrillicfonttt}{Courier New}
\newfontfamily{\cyrillicfontsf}{STIX Two Text}

\renewcommand{\thefigure}{\thesection.\arabic{figure}}
\renewcommand{\thetable}{\thesection.\arabic{table}}
\numberwithin{equation}{section}

\renewcommand{\qedsymbol}{$\blacksquare$}
\theoremstyle{definition}
\newtheorem{definition}{Опр.}[section]
\theoremstyle{remark}
\newtheorem{statement}{Утв.}[section]
\theoremstyle{plain}
\newtheorem{theorem}{Теор.}[section]

\addto\captionsrussian{
  \renewcommand{\figurename}{Рис.}
  \renewcommand{\tablename}{Табл.}
  \renewcommand{\proofname}{Док-во}
}

\graphicspath{{./img/}}
\everymath{\displaystyle}

\newcommand{\RNumb}[1]{\uppercase\expandafter{\romannumeral#1\relax}}
\newcommand{\llabel}[1]{\label{\thesubsection:#1}}
\newcommand{\lref}[1]{\ref{\thesubsection:#1}}
\begin{document}
\paragraph{Э-10}
Тензор Максвелла: определение и свойства. Уравнения Максвелла <<без источников>>.\\

Здесь полагаем
$$
\partial_x f = \frac{\partial f}{\partial x}, \qquad
\partial_\alpha f = \frac{\partial f}{\partial x^\alpha}, \qquad
F_{\alpha\beta,\gamma} = \frac{\partial F_{\alpha\beta}}{\partial x^\gamma}
$$

\begin{definition}
	Тензором Максвелла называется: 
$$F_{\alpha\beta}=
\begin{pmatrix}
0 & -E_x & -E_y & -E_z\\
E_x & 0 & B_z & -B_y\\
E_y & -B_z & 0 & B_x\\
E_z & B_y & -B_y & 0
\end{pmatrix}$$
\end{definition}

где $\bar E$ - вектор напряжения электрического поля, а $\bar B$ - магнитного.
Каждая компонента считается как:
$$
F_{\alpha\beta}=\frac{\partial A_\beta}{\partial x^{\alpha}}
               -\frac{\partial A_\alpha}{\partial x^{\beta}},
$$
где $A$ - действие.
\begin{gather}
\begin{cases}
	-E_x=F_{01}=\frac{\partial A_1}{\partial x^{0}}-\frac{\partial A_0}{\partial x^{1}}\\
	-E_y=F_{02}=\frac{\partial A_2}{\partial x^{0}}-\frac{\partial A_0}{\partial x^{2}}\\
	-E_z=F_{03}=\frac{\partial A_3}{\partial x^{0}}-\frac{\partial A_0}{\partial x^{3}}
\end{cases}
\end{gather}

откуда $\bar{E}=-\frac{\partial\bar{ A}}{\partial t}-\nabla\varphi$, где $\varphi= A^0$ - нулевая компонента 4-вектора, $\nabla$- оператор Гамильтона.
\begin{gather}
	\begin{cases}
		B_x=F_{23}=\frac{\partial A_3}{\partial x^{2}}-\frac{\partial A_2}{\partial x^{3}}\\
		-B_y=F_{13}=\frac{\partial A_3}{\partial x^{1}}-\frac{\partial A_1}{\partial x^{3}}\\
		B_z=F_{12}=\frac{\partial A_2}{\partial x^{1}}-\frac{\partial A_1}{\partial x^{2}}
	\end{cases}
\end{gather}
откуда $\bar{B}=\nabla\times\bar{ A}$.

\textbf{Свойства тензора Максвелла}
\begin{enumerate}
\item Антисимметричность: $F_{\alpha\beta}=-F_{\beta\alpha}$
\item При калибровочном преобразовании $A_0$ , компоненты тензора Максвелла остаются неизменными , $\bar{B}=const, \bar{E}=const$.
\begin{proof}
Калибровочное преобразование имеет вид:
\begin{gather*}
	 A_\alpha\to A'_\alpha= A_\alpha+\frac{\partial f}{\partial x^\alpha}
\end{gather*}
преобразуем тензор Максвелла:
\begin{gather*}
F_{\alpha\beta}\to F'_{\alpha\beta} =
\partial_\alpha A'_\beta - \partial_\beta A'_\alpha =
\partial_\alpha A_\beta + \partial_\alpha\partial_\beta f - \partial_\beta A_\alpha-\partial_\beta\partial_\alpha f=
F_{\alpha\beta}.
\end{gather*}
\end{proof}

\end{enumerate}

\begin{statement}
Тензор $\Phi_{\alpha\beta\gamma}=F_{\alpha\beta,\gamma}+F_{\gamma\alpha,\beta}+F_{\beta\gamma,\alpha}$ = 0.
\end{statement}
\begin{proof}
\begin{gather}
\begin{cases}
\Phi_{012}=F_{01,2}+F_{20,1}+F_{12,0}=-\partial_y E_x+\partial_x E_y + \partial_t B_z=0\\
\Phi_{013}=F_{01,3}+F_{30,1}+F_{13,0}=-\partial_z E_x+\partial_x E_z - \partial_t B_y=0\\
\Phi_{023}=F_{02,3}+F_{30,2}+F_{23,0}=-\partial_z E_y+\partial_y E_z + \partial_t B_x=0\\
\Phi_{123}=F_{12,3}+F_{31,2}+F_{23,1}=\partial_z E_z+\partial_y E_y + \partial_x B_x=0
\end{cases}
\end{gather}
\end{proof}
Откуда получаем уравнения Максвелла <<без источников>>:
\begin{gather}
\begin{cases}
	\nabla\times\bar{E}=-\partial_t\bar{B}\\
	\nabla\bar{B}=0
\end{cases}
\end{gather}

\end{document}