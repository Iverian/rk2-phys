% возьмите __header__ из папки вопросы для преднастройки документа
\input{__header__}

\begin{document}
\paragraph{Э-14}
Выпишите действие для заряженной частицы в электромагнитном поле. Обсудите следствия его лоренцевой и калибровочной симметрий. Воспользуйтесь принципом экстремального действия, чтобы получить выражение, определяющее 4-силу Лоренца.\\

Действие для заряда в электромагнитном поле имеет вид
\begin{gather*}
	S = \int_{a}^{b}(-mcds - \frac{e}{c}A_idx^i).
\end{gather*}
\textbf{Лоренцева сила}
Уравнения движения заряда в заданном электромагнитном поле получаются варьированием действия, т.е. даются уравнениями Лагранжа
\begin{gather}
	\label{e-14-ur}
	\frac{d}{dt}\frac{\partial L}{\partial v} = \frac{\partial L}{\partial \tau},
\end{gather}
где L определяется формулой
\begin{gather*}
	L = -mc^2\sqrt{1 - \frac{v^2}{c^2}} + \frac{e}{c}A\textnormal{v} - e\varphi.
\end{gather*}
Производная $\partial L/\partial\textnormal{v}$ есть обобщённый импульс частицы
\begin{gather*}
	P = \frac{m\textnormal{v}}{\sqrt{1-\frac{v^2}{c^2}}} + \frac{e}{c}A = p + \frac{e}{c}A.
\end{gather*}
Далее пишем:\\
\begin{gather*}
 \frac{\partial L}{\partial \tau} \equiv \triangledown L = \frac{e}{c}grad A\textnormal{v} - e grad \varphi.
\end{gather*}
Но по известной формуле векторного анализа
\begin{gather*}
	grad ab = (a\triangledown)b + (b\triangledown)a + [brot a] + [arot b],
\end{gather*}
где $a$ и $b$ - любые два вектора. Применяя эту формулу к $A\textnormal{v}$ и помня, что дифференцирование по $r$ производится при постоянном $\textnormal{v}$, находим:
\begin{gather*}
	\frac{\partial L}{\partial\tau} = \frac{e}{c}(\textnormal{v}\triangledown)A + \frac{e}{c}[\textnormal{v}rotA] - egrad\varphi.
\end{gather*}
Уравнения Лагранжа, следовательно, имеют вид
\begin{gather*}
	\frac{d}{dt}(p + \frac{e}{c}A) = \frac{e}{c}(\textnormal{v}\triangledown)A + \frac{e}{c}[\textnormal{v}rotA] - egrad\varphi.
\end{gather*}
Но полный дифференциал $\frac{dA}{dt}dt$ складывается из двух частей: из изменения $\frac{\partial A}{\partial t}dt$ векторного потенциала со временем в данной точке пространства и из изменения при переходе от одной точки пространства к другой на расстояние $dr$. Эта вторая часть равна $(dr\triangledown)A$. Таким образом,
\begin{gather*}
	\frac{dA}{dt} = \frac{\partial A}{\partial t} + (\textnormal{v}\triangledown)A.
\end{gather*}
Подставляя это в предыдущее уравнение, получаем:
\begin{gather}
	\label{e-14-ur1}
	\frac{dp}{dt} = -\frac{e}{c}\frac{\partial A}{\partial t} - egrad\varphi + \frac{e}{c}[\textnormal{v}rotA].
\end{gather}
Это и есть уравнение движения частицы в электромагнитном поле. Слева стоит производная от импульса частицы по времени. Следовательно, выражение в правой части $\ref{e-14-ur1}$ есть сила, действующая на заряд в электромагнитном поле. Мы видим, что эта сила состоит из двух частей. Первая часть(первый и второй члены в правой части) не зависит от скорости частицы. Вторая часть (третий член) зависит от этой скорости: пропорциональна величине скорости и перпендикулярна к ней.
Силу первого рода, отнесённую к заряду, равному единице, называют  напряжённостью электрического поля; обозначим её посредством $E$. Итак, по определению
\begin{gather}
	\label{e-14-ur2}
	E = -\frac{1}{c}\frac{\partial A}{\partial t} - grad \varphi.
\end{gather}
Множитель при скорости, точнее при $\textnormal{v}/c$, в силе второго рода, действующей на единичный заряд, называют напряжённостью магнитного поля; обозначим её через $H$. Итак, по определению,
\begin{gather*}
	\label{e-14-ur3}
	H = rot A.
\end{gather*}
Если в электомагнитном поле $E\ne0$, а $H=0$, то говорят об электрическом поле; если же $E=0$, а $H\ne0$, то поле называют магнитным. 
Уравнение движения заряда в электромагнитном поле можно теперь написать в виде
\begin{gather*}
	\frac{dp}{dt} = eE + \frac{e}{c}[\textnormal{v}H].
\end{gather*}
стоящее справа выражение носит название лоренцевой силы\\

Получение лоренцевой силы по Никифорову\\
\begin{gather*}
	0 = \delta \quad^{\underset{int}{prt}}\mathcal{A}=
	-m\int\delta dr-q\int\delta\mathcal{A}
_\alpha dx^\alpha - q\int\mathcal{A}_\alpha\delta dx^\alpha =
\int(mu_\alpha-q\mathcal{A}_\alpha)\delta x^\alpha|^f_i - \int\delta x^\alpha d(mu_\alpha-q\mathcal{A}_\alpha)-q\int\delta\mathcal{A}_\alpha dx^\alpha = -\int\delta x^\alpha m\frac{du_\alpha}{dr}dr+\int\delta x^\alpha q\partial_\beta\mathcal{A}_\alpha dx^\beta\frac{dr}{dr}-\int q\partial\mathcal{A}_\alpha\delta x^\beta dx^\alpha=\int dr\delta x^\alpha[-m\frac{du_\alpha}{dr}+q(\partial_\alpha\mathcal{A}_\beta-\partial_\beta\mathcal{A}_\alpha)u^\beta] => 
\end{gather*}
Уравнения движения заряженной массивной частицы в электромагнитном поле:\\
$m\frac{du_\alpha}{dr}=q(\partial_\alpha\mathcal{A}_\beta-\partial_\beta\mathcal{A}_\alpha)u^\beta$ или $F_\alpha=q\mathcal{F}_{\alpha\beta}u^\beta$ - 4-сила Лоренца.
% теоремы и определения
%\begin{definition}
%\label{m-01-def-work}
%	метка, для последующих ссылок, сама ссылка \ref{m-01-def-work}
%Элементарной работой силы $\vec F$ называется такая величина $A$, что:
%\( \displaystyle \delta A = \vec F d\vec x \).
%\end{definition}

%\begin{statement}[Свойства работы силы на участке кривой]
%	{statement} - тип теоремы (можно statement или theorem)
%	[...] - название
%\label{m-01-cv-va}
%авдпоывшщ
%\end{statement}
%\begin{proof}/
%	доказательство
%и не поспоришь.
%\end{proof}

% пример вставки картинки
%\begin{wrapfigure}[20]{L}{0.3\linewidth}
%	20 - кол-во строк, занимаемых картинкой
%	0.3\linewidth - щирина картинки в долях длины строки
%	L - выровнять слева (нестрого) (l - строго)
%\includegraphics[width=1\linewidth]{img/m-18.eps}
%	img/m-18.eps - путь к рисунку
%\caption{Двойной маятник}
%	подпись
%\label{m-18-ris}
%	метка для ссылок, поставить ссылку - \ref{m-18-ris}
%\end{wrapfigure}

%копия вопроса из phys questions.tex \\

%ответ

\end{document}