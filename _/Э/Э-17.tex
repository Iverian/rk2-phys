\input{__header__}
\begin{document}
\paragraph{Э-17}
Движущийся со скоростью $\vec v$ электрон попадает в однородное магнитное поле $\vec B$, перпендикулярное его скорости. Охарактеризуйте траекторию, по которой будет двигаться электрон. Найдите работу силы, действующей на электрон.\\

Мы не черти, мы решаем нерелятивистский случай. А значит $\nu \ll 1$.\\
$\vec H || \vec (OZ)$ и $\vec E = 0$.\\
Тогда уравнения движения будут иметь вид:
$$
m\dot{\vec\nu} = e\vec E+e\vec\nu\times\vec H = e\vec\nu\times\vec H,
$$
Или, в нашем случае:
\begin{gather*}
\begin{cases}
m\ddot x= e\dot y H \\
m\ddot y = eE_y-e\dot x H=-e\dot x H \\
m\ddot z = eE_z=0
\end{cases},
\end{gather*}


У множим второе уравнение (для $\ddot y$) на $i$ и сложим с первым, положив $\dot z = \dot x + i \dot y$:
$$
\frac{d}{dt}\dot z+i\omega\dot z =0
$$
Где $\omega = \frac{eH}{m}$. Решение этого дифференциального уравнения примет вид:
$$
\dot x + i\dot y=\alpha e ^{-i\omega t}$$

В общем случае $\alpha$ - комплексное число, однако правильно выбрав начало отсчета времени мы можем сделать так, что умножение на $e^{-i\omega t}$ избавит нас от хлопот, и, как следствие, сделает $\alpha$ действительным.\\

Отделяя комплексные переменные от действительных получим пару уравнений:
\begin{flalign*}
\begin{split}
\dot x &= \alpha \cos(\omega t) \\
\dot y &= -\alpha \sin(\omega t)
\end{split},
\end{flalign*}

Очевидно , что
\begin{flalign*}
\begin{split}
x &= \frac{\alpha}{\omega} \sin(\omega t)\\
y &= \frac{\alpha}{\omega} \cos(\omega t-1)
\end{split}
\end{flalign*}

Пределы интегрирования выставлялись так, что $x(0)=y(0)=0$.\\
В ваших глазах сейчас может появиться вопрос, однако не будем позорить Гордина, это параметрически заданная окружность. 

Ах да, нам дан вектор $\vec B=\vec H..... $
\end{document}