\include{__header__}

\begin{document}
\paragraph{Э-04}
Найдите все независимые релятивистские инварианты электромагнитного поля. Докажите, что представленный список инвариантов исчерпывающий.\\

Т.к. электромагниное поле однозначно описывается тензором электромагнитного поля $F^{\alpha\beta}$, то достаточно будет описать все инварианты этого тензора: Запишем его контравариантную форму в матричном виде:
$$
F^{\alpha\beta} =
\left(
\begin{matrix}
0   & -E_x & -E_y & -E_z \\
E_x &    0 & -B_z &  B_y \\
E_y &  B_z &    0 & -B_x \\
E_z & -B_y &  B_x &    0 
\end{matrix}
\right),
$$
Тогда запишем характеристический многочлен этого тензора:
\begin{flalign*}
&\det\left(F^{\alpha\beta} - \lambda\eta^{\alpha\beta}\right) = 
\left|
\begin{matrix}
\lambda & -E_x     &     -E_y & -E_z \\
    E_x & -\lambda &     -B_z &  B_y \\
    E_y &      B_z & -\lambda & -B_x \\
    E_z &     -B_y &      B_x & -\lambda \\
\end{matrix}
\right| = \\
=& -\lambda^4+
\lambda^2\left(E_x^2+E_y^2+E_z^2-B_x^2-B_y^2-B_z^2\right) + \\
+&
\left(
E_x^2B_x^2 + E_y^2B_y^2 + B_z^2E_z^2
+ 2E_x E_y B_x B_y + 2B_x B_z E_x E_z + 2 E_y E_z B_y B_z
\right) = \\
=& -\lambda^4+
\lambda^2\left(\vec E^2 - \vec B^2\right) +
\left(\vec E,\vec B\right)^2,
\end{flalign*}
где $\eta^{\alpha\beta}$ - обратный метирический тензор пространства Минковского, в ортонормированной СО имеющий вид:
$$
\eta^{\alpha\beta} =
\left(
\begin{matrix}
-1 & 0 & 0 & 0 \\
 0 & 1 & 0 & 0 \\
 0 & 0 & 1 & 0 \\
 0 & 0 & 0 & 1 \\
\end{matrix}
\right).
$$
Из теоремы об инвариантности характеристического многочлена следует, что полученный многочлен не меняет свой вид при смене СО. Тогда инварианты примут вид:
\begin{gather*}
I_1 = \vec{E}^2 - \vec{B}^2,\qquad
I_2 = \left(\vec E, \vec B\right)^2
\end{gather*}
То, что список этих инвариантов исчерпывающий следует из отстутствия других коэффициентов у характеристического многочлена
\footnote{
	В общем случае при произвольном тензорном преобразовании не меняется только характеристический многочлен (наверное).
}.

\end{document}