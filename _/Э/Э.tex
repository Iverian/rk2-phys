\documentclass[12pt]{article}

\usepackage{fontspec}
\usepackage{polyglossia}
\usepackage{geometry}
\usepackage{graphicx}

\usepackage{amsmath,
            amsthm,
            amssymb
}

\usepackage{unicode-math,
            tensor
}

\usepackage{wrapfig, 
            hyperref,
            multicol,
            multirow,
            tabularx,
            booktabs,
            makecell
}

\usepackage{}

\geometry{a4paper,
          total={170mm,255mm},
          left=10mm,
          top=15mm,
}

\setdefaultlanguage{russian}
\setotherlanguage{english}
\setkeys{russian}{babelshorthands=true}

\defaultfontfeatures{Ligatures=TeX}
\setmainfont{STIX Two Text}
\setmathfont{STIX Two Math}
\DeclareSymbolFont{letters}{\encodingdefault}{\rmdefault}{m}{it}

\newfontfamily{\cyrillicfont}{STIX Two Text} 
\newfontfamily{\cyrillicfontrm}{STIX Two Text}
\newfontfamily{\cyrillicfonttt}{Courier New}
\newfontfamily{\cyrillicfontsf}{STIX Two Text}

\renewcommand{\thefigure}{\thesection.\arabic{figure}}
\renewcommand{\thetable}{\thesection.\arabic{table}}
\numberwithin{equation}{section}

\renewcommand{\qedsymbol}{$\blacksquare$}
\theoremstyle{definition}
\newtheorem{definition}{Опр.}[section]
\theoremstyle{remark}
\newtheorem{statement}{Утв.}[section]
\theoremstyle{plain}
\newtheorem{theorem}{Теор.}[section]

\addto\captionsrussian{
  \renewcommand{\figurename}{Рис.}
  \renewcommand{\tablename}{Табл.}
  \renewcommand{\proofname}{Док-во}
}

\graphicspath{{./img/}}
\everymath{\displaystyle}

\newcommand{\RNumb}[1]{\uppercase\expandafter{\romannumeral#1\relax}}
\newcommand{\llabel}[1]{\label{\thesubsection:#1}}
\newcommand{\lref}[1]{\ref{\thesubsection:#1}}

\begin{document}

\paragraph{Э-01}
Исходя из требования калибровочной симметрии, получите закон сохранения электрического заряда.

\paragraph{Э-02}
Система уравнений Максвелла-Лоренца в вакууме. Работа электрической и магнитной составляющей силы Лоренца. Закон сохранения электрического заряда.

\paragraph{Э-03}
Найдите правила преобразования компонент электрического и магнитного полей при переходе от одной ИСО к другой (относительная скорость направлена вдоль оси $OY$). Какие комбинации электрического и магнитного полей при бусте вдоль $OY$ не изменяются?

\paragraph{Э-04}
Найдите все независимые релятивистские инварианты электромагнитного поля. Докажите, что представленный список инвариантов исчерпывающий.

\paragraph{Э-05}
Найдите магнитное поле, порождаемое медленно (по сравнению со скоростью света) равномерно движущимся зарядом $q$, производя преобразование полей от системы отсчёта, в которой заряд покоится.

\paragraph{Э-06}
Найдите электрическое и магнитное поля равномерно и прямолинейно движущегося заряда $q$, производя преобразование полей от системы отсчёта, в которой заряд покоится.

\paragraph{Э-07}
Система уравнений Максвелла в пространстве, свободном от зарядов и токов. Электромагнитные волны. Поперечность электромагнитных волн, релятивистские инварианты электромагнитного поля.

\paragraph{Э-08}
Сформулируйте полевую версию теоремы Нётер. Получите выражение, определяющее сохраняющийся нётеровский ток. Прокомментируйте её использование на примере канонического тензора энергии-импульса электромагнитного поля.

\paragraph{Э-09}
Полевые уравнения Эйлера-Лагранжа. Действие для электромагнитного поля в присутствии зарядов и токов. Уравнения Максвелла «с источниками». Закон сохранения электрического заряда.

\paragraph{Э-10}
Тензор Максвелла: определение и свойства. Уравнения Максвелла «без источников».

\paragraph{Э-11}
Тензор Максвелла: определение и свойства. Преобразование $\vec E$ и $\vec B$ при переходе из одной инерциальной системы отсчета в другую: на примере буста вдоль $OZ$ и в общей форме.

\paragraph{Э-12}
Калибровочные преобразования. Инвариантность тензора Максвелла относительно калибровочных преобразований. Связь полей $\vec E$ и $\vec B$ с 4-векторным потенциалом $A^\mu$.

\paragraph{Э-13}
Убедитесь, что симметризованный тензор энергии-импульса электромагнитного поля калибровочно инвариантен; интерпретируйте его компоненты.

\paragraph{Э-14}
Выпишите действие для заряженной частицы в электромагнитном поле. Обсудите следствия его лоренцевой и калибровочной симметрий. Воспользуйтесь принципом экстремального действия, чтобы получить выражение, определяющее 4-силу Лоренца.

\paragraph{Э-15}
Получите связь между тензором энергии импульса системы заряженных частиц и плотностью 4-силы Лоренца.

\paragraph{Э-16}
Движущийся со скоростью $\vec v$ электрон, попадает в однородные и взаимно перпендикулярные электрическое $\vec E$ и магнитное $\vec B$ поля. Скорость электрона перпендикулярна обоим полям. Найдите траекторию движения электрона.

\paragraph{Э-17}
Движущийся со скоростью $\vec v$ электрон попадает в однородное магнитное поле $\vec B$, перпендикулярное его скорости. Охарактеризуйте траекторию, по которой будет двигаться электрон. Найдите работу силы, действующей на электрон.

\paragraph{Э-18}
Уравнения движения заряженной массивной частицы в электромагнитном поле. 4-сила Лоренца. Калибровочные преобразования. Инвариантность тензора Максвелла относительно калибровочных преобразований.

\paragraph{Э-19}
4-вектор плотности тока. Уравнение Максвелла «с источниками». Закон сохранения электрического заряда.

\end{document}
