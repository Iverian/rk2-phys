\input{__header__}

\begin{document}
	\paragraph{Э-01}
	Исходя из требования калибровочной симметрии, получите закон сохранения электрического заряда.
	
	Из теоремы Нётер и калибровочной симметрии получаем:
	$$
	\partial_\mu\mathcal{J}^\mu = 0 ,
	$$
	где $\mathcal{J}^\mu$ - Нётеровский ток. Распишем покомпонентно и проинтегрируем по объёму:
	$j^0$ - плотность заряда, $\vec j$ - плотность потока заряда.
	\begin{flalign*}
	\partial_t j^0+\nabla\vec j&=0
	\Rightarrow
	\partial_t j^0=-\nabla\vec j
	\Rightarrow\\
	\int\partial_t j^0~dV&=-\int\nabla\vec j~dV
	\Rightarrow\\
	\partial_t\int j^0~dV&=-\oint\vec j\vec n~dS
	\Rightarrow\\
	\partial_t q&=-\oint\vec j\vec n~dS.
	\end{flalign*}
	
	Это означает, что заключенный в объёме зарядможет меняться только за счёт потока через охваченную этот объём поверхность. Если же он не выходит то $\partial_t q=0 \Rightarrow q=const.$
\end{document}