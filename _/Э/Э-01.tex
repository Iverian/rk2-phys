\documentclass[12pt]{article}

\usepackage{fontspec}
\usepackage{polyglossia}
\usepackage{geometry}
\usepackage{graphicx}

\usepackage{amsmath,
            amsthm,
            amssymb
}

\usepackage{unicode-math,
            tensor
}

\usepackage{wrapfig, 
            hyperref,
            multicol,
            multirow,
            tabularx,
            booktabs,
            makecell
}

\usepackage{}

\geometry{a4paper,
          total={170mm,255mm},
          left=10mm,
          top=15mm,
}

\setdefaultlanguage{russian}
\setotherlanguage{english}
\setkeys{russian}{babelshorthands=true}

\defaultfontfeatures{Ligatures=TeX}
\setmainfont{STIX Two Text}
\setmathfont{STIX Two Math}
\DeclareSymbolFont{letters}{\encodingdefault}{\rmdefault}{m}{it}

\newfontfamily{\cyrillicfont}{STIX Two Text} 
\newfontfamily{\cyrillicfontrm}{STIX Two Text}
\newfontfamily{\cyrillicfonttt}{Courier New}
\newfontfamily{\cyrillicfontsf}{STIX Two Text}

\renewcommand{\thefigure}{\thesection.\arabic{figure}}
\renewcommand{\thetable}{\thesection.\arabic{table}}
\numberwithin{equation}{section}

\renewcommand{\qedsymbol}{$\blacksquare$}
\theoremstyle{definition}
\newtheorem{definition}{Опр.}[section]
\theoremstyle{remark}
\newtheorem{statement}{Утв.}[section]
\theoremstyle{plain}
\newtheorem{theorem}{Теор.}[section]

\addto\captionsrussian{
  \renewcommand{\figurename}{Рис.}
  \renewcommand{\tablename}{Табл.}
  \renewcommand{\proofname}{Док-во}
}

\graphicspath{{./img/}}
\everymath{\displaystyle}

\newcommand{\RNumb}[1]{\uppercase\expandafter{\romannumeral#1\relax}}
\newcommand{\llabel}[1]{\label{\thesubsection:#1}}
\newcommand{\lref}[1]{\ref{\thesubsection:#1}}

\begin{document}
	\paragraph{Э-01}
	Исходя из требования калибровочной симметрии, получите закон сохранения электрического заряда.
	
	Из теоремы Нётер и калибровочной симметрии получаем:
	$$
	\partial_\mu\mathcal{J}^\mu = 0 ,
	$$
	где $\mathcal{J}^\mu$ - Нётеровский ток. Распишем покомпонентно и проинтегрируем по объёму:
	$j^0$ - плотность заряда, $\vec j$ - плотность потока заряда.
	\begin{flalign*}
	\partial_t j^0+\nabla\vec j&=0
	\Rightarrow
	\partial_t j^0=-\nabla\vec j
	\Rightarrow\\
	\int\partial_t j^0~dV&=-\int\nabla\vec j~dV
	\Rightarrow\\
	\partial_t\int j^0~dV&=-\oint\vec j\vec n~dS
	\Rightarrow\\
	\partial_t q&=-\oint\vec j\vec n~dS.
	\end{flalign*}
	
	Это означает, что заключенный в объёме зарядможет меняться только за счёт потока через охваченную этот объём поверхность. Если же он не выходит то $\partial_t q=0 \Rightarrow q=const.$
\end{document}