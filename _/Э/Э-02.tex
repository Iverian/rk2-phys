% возьмите __header__ из папки вопросы для преднастройки документа
\documentclass[12pt]{article}

\usepackage{fontspec}
\usepackage{polyglossia}
\usepackage{geometry}
\usepackage{graphicx}

\usepackage{amsmath,
            amsthm,
            amssymb
}

\usepackage{unicode-math,
            tensor
}

\usepackage{wrapfig, 
            hyperref,
            multicol,
            multirow,
            tabularx,
            booktabs,
            makecell
}

\usepackage{}

\geometry{a4paper,
          total={170mm,255mm},
          left=10mm,
          top=15mm,
}

\setdefaultlanguage{russian}
\setotherlanguage{english}
\setkeys{russian}{babelshorthands=true}

\defaultfontfeatures{Ligatures=TeX}
\setmainfont{STIX Two Text}
\setmathfont{STIX Two Math}
\DeclareSymbolFont{letters}{\encodingdefault}{\rmdefault}{m}{it}

\newfontfamily{\cyrillicfont}{STIX Two Text} 
\newfontfamily{\cyrillicfontrm}{STIX Two Text}
\newfontfamily{\cyrillicfonttt}{Courier New}
\newfontfamily{\cyrillicfontsf}{STIX Two Text}

\renewcommand{\thefigure}{\thesection.\arabic{figure}}
\renewcommand{\thetable}{\thesection.\arabic{table}}
\numberwithin{equation}{section}

\renewcommand{\qedsymbol}{$\blacksquare$}
\theoremstyle{definition}
\newtheorem{definition}{Опр.}[section]
\theoremstyle{remark}
\newtheorem{statement}{Утв.}[section]
\theoremstyle{plain}
\newtheorem{theorem}{Теор.}[section]

\addto\captionsrussian{
  \renewcommand{\figurename}{Рис.}
  \renewcommand{\tablename}{Табл.}
  \renewcommand{\proofname}{Док-во}
}

\graphicspath{{./img/}}
\everymath{\displaystyle}

\newcommand{\RNumb}[1]{\uppercase\expandafter{\romannumeral#1\relax}}
\newcommand{\llabel}[1]{\label{\thesubsection:#1}}
\newcommand{\lref}[1]{\ref{\thesubsection:#1}}

\begin{document}
\paragraph{Э-02}
Система уравнений Максвелла-Лоренца в вакууме. Работа электрической и магнитной составляющей силы Лоренца. Закон сохранения электрического заряда.

\textbf{Система уравнений Максвелла-Лоренца в вакууме}\\

\begin{gather*}
	H = rot A, \quad E = -\frac{1}{c}\frac{\partial A}{\partial t} - grad \varphi,
\end{gather*}
где $H$ - напряжённость магнитного поля, $E$ - напряжённость электрического поля,
$A$ - векторный потенциал поля,
$\varphi$ - скалярный потенциал.

Определим $rot E$:
\begin{gather*}
	rot E = -\frac{1}{c}\frac{\partial}{\partial t}rot A - rot grad \varphi.
\end{gather*}

Ротор градиента равен нулю.
\begin{gather}
	\label{e-02-rotE}
	rot E = -\frac{1}{c}\frac{\partial H}{\partial t}.
\end{gather}

Возьмём дивергенцию от обеих частей уравнения $rot A = H$ (дивергенция ротора равна нулю):
\begin{gather}
	\label{e-02-divH}
	div H = 0
\end{gather}

Уравнения $\ref{e-02-rotE}$ и  $\ref{e-02-divH}$ составляют первую пару уравнений Максвелла.\\
В интегральной форме:\\
\begin{gather*}
	\oint Edl = \frac{1}{c}\frac{\partial}{\partial t}\int Hdf,\\
	\oint H df = 0.
\end{gather*}

Для вывода второй пары уравнений Максвелла нам нужно знать, что действие:
\begin{gather}
	\label{e-02-S}
	S = -\sum\int mcds - \frac{1}{c^2}\int A_ij^id\Omega - \frac{1}{16\pi c}\int F_{ik}F^{ik}d\Omega.
\end{gather}

При нахождении уравнений поля их принципа наименьшего действия мы должны считать заданным движения зарядов и должны варьировать только потенциалы поля(играющие здесь роль "координат" системы); при нахождении уравнений движения мы, наоборот, считали поле заданным и варьировали траекторию частицы.
Поэтому вариация первого члена в $\ref{e-02-S}$ равна теперь нулю, а во втором не должен варьироваться ток $j^i$. Таким образом, 
\begin{gather*}
	\delta S = -\frac{1}{c}\big[\frac{1}{c}j^i\delta A_i + \frac{1}{8\pi}F^{ik}\delta F_{ik}\big]d\Omega = 0
\end{gather*}
(при варьировании во втором члене учтено, что $F^{ik}\delta F_{ik} \equiv F_{ik}\delta F^{ik}$). Подставляя
\begin{gather*}
	F_{ik} = \frac{\partial A_k}{\partial x^i} - \frac{\partial A_i}{\partial x^k},
\end{gather*}
имеем:
\begin{gather*}
	\delta S = -\frac{1}{c}\int \big\{\frac{1}{c}j^i\delta A_i + \frac{1}{8\pi}F^{ik}\frac{\partial}{\partial x^i}\delta A_k - \frac{1}{8\pi}F^{ik}\frac{\partial}{\partial x^k}\delta A_i\big\}d\Omega.
\end{gather*}
Во втором члене меняем местами индексы $i$ и $k$, по которым производится суммирование, и, кроме того, заменяем $F_{ki}$ на $-F_{ik}$.\\
Тогда мы получим:
\begin{gather*}
	\delta S = -\frac{1}{c}\int\big\{\frac{1}{c}j^i\delta A_i - \frac{1}{4\pi}F^{ik}\frac{\partial}{\partial x^k}\delta A_i\big\}d\Omega.
\end{gather*}
Второй из этих интегралов берём по частям, т.е. применяем теорему Гаусса:
\begin{gather}
	\label{e-02-deltaS}
	\delta S = -\frac{1}{c}\int\big\{\frac{1}{c}j^i + \frac{1}{4\pi}\frac{\partial F^{ik}}{\partial x^k}\big\}\delta A_id\Omega - \frac{1}{4\pi c}\int F^{ik}\delta A_idS_k\big|.
\end{gather}
Во втором члене  мы должны взять его значение на пределах интегрирования. Пределами интегрирования по координатам является бесконечность, где поле исчезает. На пределах же интегрирования по времени, т.е. в заданные начальный и конечный моменты времени, вариация потенциалов равна нулю, так как по смыслу принципа наименьшего действия потенциалы в эти моменты заданы. Таким образом, второй член в $\ref{e-02-deltaS}$ равен нулю, и мы находим:
\begin{gather*}
	\int(\frac{1}{c}j^i + \frac{1}{4\pi}\frac{\partial F^{ik}}{\partial x^k})\delta A_id\Omega = 0.
\end{gather*}
Ввиду того, что по смыслу принципа наименьшего действия вариации $\delta A_i$ произвольны, нулю должен равняться коэффициент при $\delta A_i$, т.е.
\begin{gather}
	\label{e-02-deltaF}
	\frac{\partial F^{ik}}{\partial x^k} = -\frac{4\pi}{c}j^i.
\end{gather}
Перепишем эти четыре ($i = 0, 1, 2, 3$) уравнения в трёхмерной форме. При $i=1$ имеем:
\begin{gather*}
	\frac{1}{c}\frac{\partial F^{10}}{\partial t} + \frac{\partial F^{11}}{\partial x} + \frac{\partial F^{12}}{\partial y} + \frac{\partial F^{13}}{\partial z} = -\frac{4\pi}{c}j^1.
\end{gather*}
Подставляя значения составляющих тензора $F^{ik}$, находим:
\begin{gather*}
	\frac{1}{c}\frac{\partial E_x}{\partial t} - \frac{\partial H_z}{\partial y} + \frac{\partial H_y}{\partial z} = -\frac{4\pi}{c}j_x.
\end{gather*}
Вместе с двумя следующими $(i = 2, 3)$ уравнениями они могут быть записаны как одно векторное:
\begin{gather}
	\label{e-02-rotH}
	rot H = \frac{1}{c}\frac{\partial E}{\partial t} + \frac{4\pi}{c}j.
\end{gather}
Наконец, уравнение с $i=0$ даёт:
\begin{gather}
\label{e-02-divE}
	div E = 4\pi\rho.
\end{gather}
Уравнения $\ref{e-02-rotH}$ и $\ref{e-02-divE}$ и составляют вторую пару уравнения Максвелла.\\
В интегральной форме:\\
\begin{gather*}
	\oint Hdl = \frac{1}{c}\frac{\partial}{\partial t}\int Edf + \frac{4\pi}{c}\int jdf,\\
	\oint Edf = 4\pi\int \rho dV.
\end{gather*}
 Уравнения Максвелла являются основными уравнениями электродинамики.\\

\textbf{Закон сохранения электрического заряда}\\
Зако́н сохране́ния электри́ческого заря́да гласит, что алгебраическая сумма зарядов электрически замкнутой системы сохраняется.
\begin{gather*}
 q_{1}+q_{2}+q_{3}+......+q_{n}=const
 \end{gather*}
 Закон сохранения заряда в интегральной форме\\
 Вспомним, что плотность потока электрического заряда есть просто плотность тока. Тот факт, что изменение заряда в объёме равно полному току через поверхность, можно записать в математической форме:
 \begin{gather*}
 	{\displaystyle {\frac {\partial }{\partial t}}\int \limits _{\Omega }\rho dV=-\oint \limits _{\partial \Omega }{\vec {j}}\cdot d{\vec {S\ }}.}
 \end{gather*}
 Здесь $\Omega$  — некоторая произвольная область в трёхмерном пространстве, $\partial \Omega$  — граница этой области, $\rho$  — плотность заряда, $\vec{j}$ — плотность тока (плотность потока электрического заряда) через границу.
% теоремы и определения
%\begin{definition}
%\label{m-01-def-work}
%	метка, для последующих ссылок, сама ссылка \ref{m-01-def-work}
%Элементарной работой силы $\vec F$ называется такая величина $A$, что:
%\( \displaystyle \delta A = \vec F d\vec x \).
%\end{definition}

%\begin{statement}[Свойства работы силы на участке кривой]
%	{statement} - тип теоремы (можно statement или theorem)
%	[...] - название
%\label{m-01-cv-va}
%авдпоывшщ
%\end{statement}
%\begin{proof}/
%	доказательство
%и не поспоришь.
%\end{proof}

% пример вставки картинки
%\begin{wrapfigure}[20]{L}{0.3\linewidth}
%	20 - кол-во строк, занимаемых картинкой
%	0.3\linewidth - щирина картинки в долях длины строки
%	L - выровнять слева (нестрого) (l - строго)
%\includegraphics[width=1\linewidth]{img/m-18.eps}
%	img/m-18.eps - путь к рисунку
%\caption{Двойной маятник}
%	подпись
%\label{m-18-ris}
%	метка для ссылок, поставить ссылку - \ref{m-18-ris}
%\end{wrapfigure}

%копия вопроса из phys questions.tex \\

%ответ

\end{document}