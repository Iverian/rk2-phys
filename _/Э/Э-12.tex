\documentclass[12pt]{article}

\usepackage{fontspec}
\usepackage{polyglossia}
\usepackage{geometry}
\usepackage{graphicx}

\usepackage{amsmath,
            amsthm,
            amssymb
}

\usepackage{unicode-math,
            tensor
}

\usepackage{wrapfig, 
            hyperref,
            multicol,
            multirow,
            tabularx,
            booktabs,
            makecell
}

\usepackage{}

\geometry{a4paper,
          total={170mm,255mm},
          left=10mm,
          top=15mm,
}

\setdefaultlanguage{russian}
\setotherlanguage{english}
\setkeys{russian}{babelshorthands=true}

\defaultfontfeatures{Ligatures=TeX}
\setmainfont{STIX Two Text}
\setmathfont{STIX Two Math}
\DeclareSymbolFont{letters}{\encodingdefault}{\rmdefault}{m}{it}

\newfontfamily{\cyrillicfont}{STIX Two Text} 
\newfontfamily{\cyrillicfontrm}{STIX Two Text}
\newfontfamily{\cyrillicfonttt}{Courier New}
\newfontfamily{\cyrillicfontsf}{STIX Two Text}

\renewcommand{\thefigure}{\thesection.\arabic{figure}}
\renewcommand{\thetable}{\thesection.\arabic{table}}
\numberwithin{equation}{section}

\renewcommand{\qedsymbol}{$\blacksquare$}
\theoremstyle{definition}
\newtheorem{definition}{Опр.}[section]
\theoremstyle{remark}
\newtheorem{statement}{Утв.}[section]
\theoremstyle{plain}
\newtheorem{theorem}{Теор.}[section]

\addto\captionsrussian{
  \renewcommand{\figurename}{Рис.}
  \renewcommand{\tablename}{Табл.}
  \renewcommand{\proofname}{Док-во}
}

\graphicspath{{./img/}}
\everymath{\displaystyle}

\newcommand{\RNumb}[1]{\uppercase\expandafter{\romannumeral#1\relax}}
\newcommand{\llabel}[1]{\label{\thesubsection:#1}}
\newcommand{\lref}[1]{\ref{\thesubsection:#1}}

\begin{document}
\pdfbookmark[2]{Э-12}{} \paragraph{Э-12}
Калибровочные преобразования. Инвариантность тензора Максвелла относительно калибровочных преобразований. Связь полей $\vec E$ и $\vec B$ с 4-векторным потенциалом $A^\mu$.\\

\textbf{Калибровочные преобразования}\\
$$\mathcal A_\mu \longrightarrow \mathcal A'_\mu \equiv \mathcal A_\mu +\partial_\mu f(x)$$
\textbf{Инвариантность тензора Максвелла относительно калибровочных преобразований}\\
$$F_{\mu\nu} \longrightarrow F'_{\mu\nu} \equiv \partial_\mu \mathcal A'_\nu - \partial_\nu \mathcal A'_\mu = \partial_\mu \mathcal A_\nu + \partial_\mu \partial_\nu f(x) - \partial_\nu \mathcal A_\mu - \partial_\nu\partial_\mu f(x) = F_{\mu\nu}$$
\textbf{Связь полей $\vec E$ и $\vec B$ с 4-векторным потенциалом $A^\mu$}\\
\begin{flalign*}
\left. \begin{matrix}
-E_x = F_{01} = \partial_0\mathcal A_1 - \partial_1 \mathcal A_0\\
-E_y = F_{02} = \partial_0\mathcal A_2 - \partial_2 \mathcal A_0\\
-E_z = F_{03} = \partial_0\mathcal A_3 - \partial_3 \mathcal A_0\\
\end{matrix}\right\}& \qquad \vec E = -\nabla\phi - \partial_t \vec \mathcal A\\
\left. \begin{matrix}
B_x = F_{23} = \partial_2\mathcal A_3 - \partial_3 \mathcal A_2\\
-B_y = F_{13} = \partial_1\mathcal A_3 - \partial_3 \mathcal A_1\\
B_z = F_{12} = \partial_1\mathcal A_2 - \partial_2 \mathcal A_1\\
\end{matrix}\right\}& \qquad \vec B = \nabla\times \vec \mathcal A
\end{flalign*}
\begin{gather*}
p_\alpha = (- \mathcal E, \vec p)
\frac{dp_\alpha}{dr} = qF_{\alpha\beta}\frac{dx^\beta}{dr};\\
dr = \sqrt{1-v^2}dt = \frac{dt}{\gamma} \rightarrow \gamma\frac{dp_\alpha}{dt} = qF_{\alpha\beta}\gamma\frac{dx^\beta}{dt} \rightarrow \frac{dp_\alpha}{dt} = qF_{\alpha\beta}\frac{dx^\beta}{dt};\\
\frac{d}{dt}\left(\begin{matrix}-\mathcal E\\ p_x\\p_y\\p_z\end{matrix}\right) = 
q\left(\begin{matrix}
0&-E_x&-E_y&-E_z\\
E_x&0&B_z&-B_y\\
E_y&-B_z&0&B_x\\
E_z&B_y&-B_x&0
\end{matrix}\right)\left(\begin{matrix}1\\v_x\\v_y\\v_z\end{matrix}\right)
\end{gather*}
или:
\begin{gather*}
\frac{d\mathcal E}{dt} = q\vec E\cdot \vec v\\
\frac{d\vec p}{dt} = q\vec E + q\vec v \times \vec B (\text{Сила Лоренца})
\end{gather*}
\end{document} 
