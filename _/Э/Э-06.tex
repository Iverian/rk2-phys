\include{__header__}

\begin{document}

\paragraph{Э-06}
Найдите электрическое и магнитное поля равномерно и прямолинейно движущегося заряда $q$, производя преобразование полей от системы отсчёта, в которой заряд покоится.\\

Рассмотрим заряд $q$, помещенный в начало координат ИСО $K_1$, движущейся относительно ИСО $K$ вдоль оси $Ox$ с быстротой $\nu=const$, тогда координаты заряда в $K$ примут вид:
\begin{gather*}
x^1 = \nu x^0, \quad x^2 = 0, \quad x^3 = 0,
\end{gather*}
где $\nu = const$ -- быстрота $K_1$ относительно $K$.
Из закона Кулона в ИСО $K_1$ потенциал поля, порождаемого зарядом $q$:
\begin{gather}
\label{e-06-potez}
A^{(1)}_{\alpha} = (\varphi_{(1)}, \vec{A}_{(1)}) = (\frac{q}{x_{(1)}},\vec{0}),
\end{gather}
где $x_{(1)}=\sqrt{\left(x^1_{(1)}\right)^2+\left(x^2_{(1)}\right)^2+\left(x^3_{(1)}\right)^2}$, тогда напряженности электрического и магнитного полей примут вид:
\begin{gather}
\label{e-06-field}
\vec{E}_{(1)} = -\frac{\partial \vec A_{(1)}}{\partial x^0_{(1)}} - \nabla \varphi_{(1)} = -\nabla \varphi_{(1)},
\qquad
\vec{H}_{(1)} = \nabla \times \vec A_{(1)} = \vec{0},
\end{gather}
из векторных форм полевых преобразований Лоренца при $\vec{\nu} = \nu\vec{e}_1$ - вектор быстроты $K_1$ относительно $K$:
\begin{gather}
\label{e-06-trans}
\vec{E} = \vec{E}_{(1)} + \vec{H}_{(1)} \times \vec{\nu},
\qquad
\vec{H} = \vec{H}_{(1)} - \vec{E}_{(1)} \times \vec{\nu},
\end{gather}
тогда из (\ref{e-06-potez}), (\ref{e-06-field}) и (\ref{e-06-trans}):
\begin{flalign}
\label{e-06-newfield}
\begin{split}
\vec{E} &= -\nabla\varphi_{(1)} = \frac{q}{\left(x_{(1)} \right)^3}\vec x_{(1)} , \\
\vec{H} &= \nabla\varphi_{(1)} \times\vec\nu =
-\frac{q\nu}{\left(x_{(1)} \right)^3}\left(x_{(1)}^3\vec{e}^{(1)}_2 - x_{(1)}^2\vec{e}^{(1)}_3\right),
\end{split}
\end{flalign}
Теперь из преобразований Лоренца найдем координаты $\vec{x}_1$ в ИСО $K$:
\begin{gather}
\label{e-06-lorenz}
\left(\begin{matrix}
x^0_{(1)} \\ x^1_{(1)} \\x^2_{(1)} \\ x^3_{(1)}
\end{matrix}\right) = 
\left(\begin{matrix}
    \gamma & -\gamma\nu & 0 & 0 \\
-\gamma\nu &     \gamma & 0 & 0 \\
         0 &          0 & 1 & 0 \\
         0 &          0 & 0 & 1
\end{matrix}\right)
\left(\begin{matrix}
x^0 \\ x^1 \\ x^2 \\ x^3
\end{matrix}\right)
\qquad \Rightarrow \qquad
\left(\begin{matrix}
x^0_{(1)} \\ x^1_{(1)} \\x^2_{(1)} \\ x^3_{(1)}
\end{matrix}\right) =
\left(\begin{matrix}
\gamma x^0 - \gamma\nu x^1 \\
-\gamma\nu x^0 + \gamma x^1 \\
x^2 \\
x^3
\end{matrix}\right),
\end{gather}
тогда из (\ref{e-06-newfield}) и (\ref{e-06-lorenz}):
\begin{gather}
\label{e-06-module}
x_{(1)} = \sqrt{\left(x^1_{(1)}\right)^2+\left(x^2_{(1)}\right)^2+\left(x^3_{(1)}\right)^2} =
\sqrt{\gamma^2\left(x^1-\nu x^0\right)^2+\left(x^2\right)^2+\left(x^3\right)^2},
\end{gather}
окончательно из (\ref{e-06-lorenz}) и (\ref{e-06-module}) получим:
\begin{flalign}
\begin{split}
\vec{E} &=
\frac{q\left(\gamma(x^1-\nu x^0)\vec{e}_1 + x^2\vec{e}_2 + x^3\vec{e}_3\right)}
{\sqrt[2/3]{\gamma^2\left(x^1-\nu x^0\right)^2+\left(x^2\right)^2+\left(x^3\right)^2}}
,\\
\vec{H} &= -\frac{q\nu \left(x^3\vec{e}_2 - x^2\vec{e}_3\right)}
{\sqrt[2/3]{\gamma^2\left(x^1-\nu x^0\right)^2+\left(x^2\right)^2+\left(x^3\right)^2}},
\end{split}
\end{flalign}

\end{document}