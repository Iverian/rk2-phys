\documentclass[12pt]{article}

\usepackage{fontspec}
\usepackage{polyglossia}
\usepackage[a4paper,
            total={170mm,255mm},
            left=10mm,
            top=15mm]
            {geometry}
\usepackage{graphicx}

\usepackage{amsmath,
            amsthm,
            amssymb
}

\usepackage{unicode-math,
            tensor
}

\usepackage{wrapfig, 
            hyperref,
            multicol,
            multirow,
            tabularx,
            booktabs,
            makecell,
            subfiles
}

\setdefaultlanguage{russian}
\setotherlanguage{english}
\setkeys{russian}{babelshorthands=true}

\defaultfontfeatures{Ligatures=TeX}
\setmainfont{STIX Two Text}
\setmathfont{STIX Two Math}
\DeclareSymbolFont{letters}{\encodingdefault}{\rmdefault}{m}{it}

\newfontfamily{\cyrillicfont}{STIX Two Text} 
\newfontfamily{\cyrillicfontrm}{STIX Two Text}
\newfontfamily{\cyrillicfonttt}{Courier New}
\newfontfamily{\cyrillicfontsf}{STIX Two Text}

\renewcommand{\thefigure}{\thesection.\arabic{figure}}
\renewcommand{\thetable}{\thesection.\arabic{table}}
\numberwithin{equation}{section}

\renewcommand{\qedsymbol}{$\blacksquare$}
\theoremstyle{definition}
\newtheorem{definition}{Опр.}[section]
\theoremstyle{remark}
\newtheorem{statement}{Утв.}[section]
\theoremstyle{plain}
\newtheorem{theorem}{Теор.}[section]

\addto\captionsrussian{
  \renewcommand{\figurename}{Рис.}
  \renewcommand{\tablename}{Табл.}
  \renewcommand{\proofname}{Док-во}
}

\graphicspath{{./img/}}
\everymath{\displaystyle}

\newcommand{\RNumb}[1]{\uppercase\expandafter{\romannumeral#1\relax}}

\newcommand{\llabel}[1]{\label{\thesubsection:#1}}
\newcommand{\lref}[1]{\ref{\thesubsection:#1}}


\begin{document}
\paragraph{24} 
Стационарное уравнение Шрёдингера. Электрон в бесконечно глубокой одномерной потенциальной яме.\\

Запишем уравнение Шрёдингера квантовой системы формально:
$$i\hbar\frac{\partial}{\partial t}\Psi=H\Psi$$
Предположим, что гамильтониан (H) не зависит от времени явно. Это случай консервативным системам, соответствующих классическим системам, для которых энергия есть интеграл движения. Образуем решение $\Psi$, представляющее динамическое состояние с определенной энергией E.\\
Такая волновая функция  $\Psi$ должна обладать вполне определенной круговой частотой, соответствующей соотношению Эйнштейна $E=\hbar \omega$. 
Отсюда получим, что:
$$\Psi=\psi e ^{-i\frac{Et}{\hbar}}$$
Где $\psi$ функция от координат в конфигурационном пространстве, но не от времени.\\
Подставляя вышенаписанное в формальное ур-е Шрёдингера получаем:
$$H\psi=E\psi$$
Это уравнение называется уранением стационарным уравнением Шрёдингера.\\
Для электрона в при движении в потенциале верно:
$$\frac{\hat{p}}{2m}+V(\vec{r})\sim \hat{H} $$
Для одного измерения также верно:
$$\hat{p}=\frac{\hbar}{i}\partial_x\Rightarrow\hat{H}=-\frac{\hbar^2}{2m}\frac{\partial^2}{\partial x^2}+V(x)$$
Подставляя в уранение Шрёдингера для стационарных состояний получим:
$$\frac{d^2\psi(x)}{dx^2}+\frac{2m}{\hbar}(E-V)\psi(x)=0 \Leftrightarrow \psi''+k^2\psi=0\Leftrightarrow \psi(x)=Ae^{ikx}+Be^{-ikx}$$
$$
\left\{
\begin{gathered}
\psi(0)=0\\
\psi(l)=0\\
\end{gathered}
\right. 
\Rightarrow
\left\{
\begin{gathered}
A+B=0\\
Ae^{ikx}+Be^{-ikx}=0\\
\end{gathered}
\right. 
\Rightarrow
\sin{kl}=0
\Rightarrow
k_n=\frac{\pi n}{l}
$$
$$\psi_(x)=2i A \sin{k_n x}$$
Энергия частицы будет иметь вид:
$$T_n=\frac{\hbar^2 k_n^2}{2m}=\frac{\pi^2\hbar^2 }{2ml^2}n^2$$
Найдем коэффициент А:
$$\int \limits^{\infty}_{-\infty}dx|\psi_n|^2=\int \limits^{l}_{0}4A^2\sin^2{(k_nx)} dx=4A^2\frac{l}{2}$$
Так как $||\psi_n||^2=1$ получаем, что $A=\frac{1}{\sqrt{2l}}$\\
В итоге : $\psi_n(x)=\sqrt{\frac{2}{l}}\sin{(\frac{\pi n x}{l})}$\\
Я хз что тут еще надо, если честно.\\
\end{document}