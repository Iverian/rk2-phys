\documentclass[12pt]{article}

\usepackage{fontspec}
\usepackage{polyglossia}
\usepackage[a4paper,
            total={170mm,255mm},
            left=10mm,
            top=15mm]
            {geometry}
\usepackage{graphicx}

\usepackage{amsmath,
            amsthm,
            amssymb
}

\usepackage{unicode-math,
            tensor
}

\usepackage{wrapfig, 
            hyperref,
            multicol,
            multirow,
            tabularx,
            booktabs,
            makecell,
            subfiles
}

\setdefaultlanguage{russian}
\setotherlanguage{english}
\setkeys{russian}{babelshorthands=true}

\defaultfontfeatures{Ligatures=TeX}
\setmainfont{STIX Two Text}
\setmathfont{STIX Two Math}
\DeclareSymbolFont{letters}{\encodingdefault}{\rmdefault}{m}{it}

\newfontfamily{\cyrillicfont}{STIX Two Text} 
\newfontfamily{\cyrillicfontrm}{STIX Two Text}
\newfontfamily{\cyrillicfonttt}{Courier New}
\newfontfamily{\cyrillicfontsf}{STIX Two Text}

\renewcommand{\thefigure}{\thesection.\arabic{figure}}
\renewcommand{\thetable}{\thesection.\arabic{table}}
\numberwithin{equation}{section}

\renewcommand{\qedsymbol}{$\blacksquare$}
\theoremstyle{definition}
\newtheorem{definition}{Опр.}[section]
\theoremstyle{remark}
\newtheorem{statement}{Утв.}[section]
\theoremstyle{plain}
\newtheorem{theorem}{Теор.}[section]

\addto\captionsrussian{
  \renewcommand{\figurename}{Рис.}
  \renewcommand{\tablename}{Табл.}
  \renewcommand{\proofname}{Док-во}
}

\graphicspath{{./img/}}
\everymath{\displaystyle}

\newcommand{\RNumb}[1]{\uppercase\expandafter{\romannumeral#1\relax}}

\newcommand{\llabel}[1]{\label{\thesubsection:#1}}
\newcommand{\lref}[1]{\ref{\thesubsection:#1}}


\begin{document}

\paragraph{40}
Временное и стационарное уравнения Шрёдингера.\\

\textbf{Временное уравнение Шредингера}:
\begin{gather*}
	\hat{H}\psi(t,\vec{r}) = ih\frac{\partial}{\partial t}\psi(t,\vec{r})
\end{gather*}
где $\hat{H} = \frac{\hat{p}^2}{2m}+U(t,\vec{r})$ - Гамильтониан или полная энергия системы.
Уравнение Шредингера позволяет найти $\psi$ функцию в любой момент времени, если известно ее значение в начальный момент времени, т.е уравнение Шредингера выражает принцип причинности в квантовой механике.\\\\
Пусть $U(\vec{r})$, т.е потенциальная энергия не зависит явно от времени тогда существует решение вида $\psi(t,\vec{r})=X(t)\psi(\vec{r})$ - процедура разделения переменных.
\begin{gather*}
	\hat{H}\psi(t,\vec{r})=-\frac{h^2}{2m}\Delta\psi(t,\vec{r})+U(t,\vec{r})\psi(t,\vec{r})
\end{gather*}
Используя процедуру разделения переменных получаем:
\begin{gather*}
\left[-\frac{h^2}{2m}\Delta\psi(\vec{r})+U(\vec{r})\psi(\vec{r})\right]X(t)=ih\psi(\vec{r})\frac{dX(t)}{dt}
\end{gather*}
Поделив обе части равенства на $\psi(\vec{r})X(t)$ получим:
\begin{gather*}
-\frac{h^2}{2m}\frac{\Delta\psi(\vec{r})}{\psi(\vec{r})}+U(\vec{r})=ih\frac{\frac{dX(t)}{dt}}{X(t)} = \varepsilon = const
\end{gather*}
Как видим левая часть равенства зависит от $\vec{r}$ и равна $\hat{H}$, а правая зависит от $t$\\\\
Отсюда получаем \textbf{стационарное уравнение Шредингера}
\begin{gather*}
	\hat{H}\psi(\vec{r}) = \varepsilon\psi(\vec{r})
\end{gather*}


\end{document}