\documentclass[12pt]{article}

\usepackage{fontspec}
\usepackage{polyglossia}
\usepackage[a4paper,
            total={170mm,255mm},
            left=10mm,
            top=15mm]
            {geometry}
\usepackage{graphicx}

\usepackage{amsmath,
            amsthm,
            amssymb
}

\usepackage{unicode-math,
            tensor
}

\usepackage{wrapfig, 
            hyperref,
            multicol,
            multirow,
            tabularx,
            booktabs,
            makecell,
            subfiles
}

\setdefaultlanguage{russian}
\setotherlanguage{english}
\setkeys{russian}{babelshorthands=true}

\defaultfontfeatures{Ligatures=TeX}
\setmainfont{STIX Two Text}
\setmathfont{STIX Two Math}
\DeclareSymbolFont{letters}{\encodingdefault}{\rmdefault}{m}{it}

\newfontfamily{\cyrillicfont}{STIX Two Text} 
\newfontfamily{\cyrillicfontrm}{STIX Two Text}
\newfontfamily{\cyrillicfonttt}{Courier New}
\newfontfamily{\cyrillicfontsf}{STIX Two Text}

\renewcommand{\thefigure}{\thesection.\arabic{figure}}
\renewcommand{\thetable}{\thesection.\arabic{table}}
\numberwithin{equation}{section}

\renewcommand{\qedsymbol}{$\blacksquare$}
\theoremstyle{definition}
\newtheorem{definition}{Опр.}[section]
\theoremstyle{remark}
\newtheorem{statement}{Утв.}[section]
\theoremstyle{plain}
\newtheorem{theorem}{Теор.}[section]

\addto\captionsrussian{
  \renewcommand{\figurename}{Рис.}
  \renewcommand{\tablename}{Табл.}
  \renewcommand{\proofname}{Док-во}
}

\graphicspath{{./img/}}
\everymath{\displaystyle}

\newcommand{\RNumb}[1]{\uppercase\expandafter{\romannumeral#1\relax}}

\newcommand{\llabel}[1]{\label{\thesubsection:#1}}
\newcommand{\lref}[1]{\ref{\thesubsection:#1}}


\begin{document}
	
	\paragraph{16.}
	Получите и прокомментируйте <<соотношение неопределённостей энергия-время>>.
	
	Квантовая механика в общем случае не даёт с достоверностью результатов того или иного измерения в отдельной системе. Что квантовая механика даёт, так это статистическое распределение результатов, получаемых при повторении одного и того же измерения на совокупность очень большого числа таких систем, состояния которых представлены одинаковыми векторными функциями.
	
	Итак, статистическое описание явлений микромира неизбежно, и квантовая механика позволяет нам опеределить вероятности того или иного результата эксперимента и средние значения физических величин.
	
	Пусть $\hat{A} |j\rangle = a_j |j\rangle$, тогда, если состояние системы определяется вектором $|c\rangle$ (condition), то 
	$$\langle A \rangle = \langle c | \hat{A} | c \rangle = \sum_{j}^{} \langle c | \hat{A} | j \rangle \langle j | c \rangle = \sum_{j} \langle c | j \rangle \langle j | c \rangle a_j = \sum_{j} |c_j|^2 a_j$$
	
	Определим неопределённости: $\Delta A \equiv \sigma[A] \equiv \sqrt{\langle(A-\langle A \rangle)^2\rangle} $
	
	$$(\Delta A)^2 = \langle(A-\langle A \rangle)^2\rangle = \langle c | (A - \langle A \rangle)(A - \langle A \rangle) | c \rangle = \langle d | d \rangle $$
	
	Аналогично, $(\Delta B)^2 = \langle f | f \rangle$
	
	$$(\Delta A)^2 (\Delta B)^2 = \langle d | d \rangle \langle f | f \rangle \geqslant | \langle d | f \rangle |^2 $$
	
	Это \textbf{неравенство Шварца}.
	
	$$\langle d | f \rangle = \langle c | (\hat{A} - \langle A \rangle)(\hat{B} - \langle B \rangle) | c \rangle = \langle AB \rangle - \langle A \rangle \langle B \rangle$$	
	
	$$\langle f | d \rangle = \langle BA \rangle - \langle B \rangle \langle A \rangle$$
	
	$$ | \langle d | f \rangle |^2 = Re^2 +Im^2 \geqslant Im^2 = \langle \frac{[A,B]}{2i} \rangle$$
	
	\textbf{\textit{Мораль:}} 
	
	$$\Delta A \Delta B \geqslant | \langle \frac{[\hat{A},\hat{B}]}{2} \rangle |$$
	
	Получается, что соотношение неопределённостей - это некое утверждение о средних.
	
	$$\Delta x \Delta p_x \geqslant | \langle \frac{[\hat{x},\hat{p_x}]}{2} \rangle | \rightarrow \Delta x \Delta p_x \geqslant \frac{\hbar}{2}$$
	
	Возьмём в качестве $\hat{B}$ гамильтониан, тогда $\Delta A \Delta \xi \geqslant | \langle \frac{[\hat{A},\hat{\mathcal{H}}]}{2} \rangle |$.
	
	Если $\hat{A}$ не зависит явно от времени, то $[\hat{A},\hat{\mathcal{H}}]$ полностью определяет скорость эволюции во времени квантовой механики среднего $A$: 
	
	$$\frac{d}{dt} \langle A \rangle = \frac{1}{i \hbar} \langle [\hat{A},\hat{H}] \rangle \rightarrow | \langle [\hat{A},\hat{H}] \rangle | = \hbar | \frac{d}{dt} \langle A \rangle |$$
	
	Тогда 
	
	$$\frac{\Delta A}{\frac{d}{dt}\langle A \rangle} \Delta \xi \geqslant \frac{\hbar}{2},$$
	
	где $\frac{\Delta A}{\frac{d}{dt}\langle A \rangle} \equiv \Sigma_A$ - характеристическое время эволюции статистического распределения А.
	
	Стационарное состояние $\frac{d}{dt} \langle A \rangle = 0 \Rightarrow \Sigma_A = \infty \Rightarrow \Delta \xi = 0$
\end{document}