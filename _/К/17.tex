\documentclass[12pt]{article}

\usepackage{fontspec}
\usepackage{polyglossia}
\usepackage[a4paper,
            total={170mm,255mm},
            left=10mm,
            top=15mm]
            {geometry}
\usepackage{graphicx}

\usepackage{amsmath,
            amsthm,
            amssymb
}

\usepackage{unicode-math,
            tensor
}

\usepackage{wrapfig, 
            hyperref,
            multicol,
            multirow,
            tabularx,
            booktabs,
            makecell,
            subfiles
}

\setdefaultlanguage{russian}
\setotherlanguage{english}
\setkeys{russian}{babelshorthands=true}

\defaultfontfeatures{Ligatures=TeX}
\setmainfont{STIX Two Text}
\setmathfont{STIX Two Math}
\DeclareSymbolFont{letters}{\encodingdefault}{\rmdefault}{m}{it}

\newfontfamily{\cyrillicfont}{STIX Two Text} 
\newfontfamily{\cyrillicfontrm}{STIX Two Text}
\newfontfamily{\cyrillicfonttt}{Courier New}
\newfontfamily{\cyrillicfontsf}{STIX Two Text}

\renewcommand{\thefigure}{\thesection.\arabic{figure}}
\renewcommand{\thetable}{\thesection.\arabic{table}}
\numberwithin{equation}{section}

\renewcommand{\qedsymbol}{$\blacksquare$}
\theoremstyle{definition}
\newtheorem{definition}{Опр.}[section]
\theoremstyle{remark}
\newtheorem{statement}{Утв.}[section]
\theoremstyle{plain}
\newtheorem{theorem}{Теор.}[section]

\addto\captionsrussian{
  \renewcommand{\figurename}{Рис.}
  \renewcommand{\tablename}{Табл.}
  \renewcommand{\proofname}{Док-во}
}

\graphicspath{{./img/}}
\everymath{\displaystyle}

\newcommand{\RNumb}[1]{\uppercase\expandafter{\romannumeral#1\relax}}

\newcommand{\llabel}[1]{\label{\thesubsection:#1}}
\newcommand{\lref}[1]{\ref{\thesubsection:#1}}


\begin{document}

\paragraph{17}
Квантовомеханическое среднее и его временная эволюция.\\

По собственным функциям наблюдаемого оператора Вы можете разложить любую функцию из Гильбертова пространства, описывающую текущее состояние квантовой системы:

\begin{gather*}
	\psi (x) = \sum_nc_n\varphi_n(x)
\end{gather*}

Рассмотрим следующие выражения:
\begin{enumerate}
	\item $(\psi,\hat A \psi) = \left(\sum_mc_m\varphi_m,\sum_nc_na_n\varphi_n\right) = \sum_{mn}c_m^*c_na_n(\varphi_m,\varphi_n) = \sum_{mn}c_m^*c_na_n\delta_{mn} = \sum_m|c_m|^2a_m $
	\item $(\psi,\psi) = \left(\sum_mc_m\varphi_m,\sum_nc_n\varphi_n\right) = \sum_{mn}c_m^*c_n(\varphi_m,\varphi_n) = \sum_{mn}c_m^*c_n\delta_{mn} = \sum_m|c_m|^2 $
\end{enumerate}

Квантовомеханическое среднее значение наблюдаемой A находится по формуле:
\begin{gather*}
	<A> = \sum_jP_{a_j}a_j = \frac{\sum_j|c_j|^2a_j}{\sum_p|c_p|^2} = \frac{(\psi,\hat{A}\psi)}{(\psi,\psi)}
\end{gather*}
Исследуем, как кв.мех. среднее эволюционирует во времени. Возьмем такую вектор функцию, которая нормирована на единицу,т.е
\begin{gather*}
	\vert\vert \psi(t_0,\vec{r})\vert\vert ^2 = 1 \Rightarrow\;  <A> = \frac{(\psi,\hat{A}\psi)}{(\psi,\psi)} = \frac{(\psi,\hat{A}\psi)}{\vert\vert\psi\vert\vert^2} = (\psi,\hat{A}\psi)
\end{gather*}
Теперь посмотрим как она эволюционирует во времени:
\begin{gather*}
	\frac{d}{dt}<A(t)> = \frac{d}{dt}\left(\psi(t,\vec{r}), \hat{A} \psi(t,\vec{r})\right) = \left(\frac{d}{dt}\psi,\hat{A}\psi\right)+\left(\psi,\frac{d}{dt}(\hat{A})\psi\right)+\left(\psi,\hat{A}\frac{d}{dt}\psi\right) = 
\end{gather*}

Вспомним, что  $\frac{d}{dt}\psi = \frac{1}{ih}\widehat{H}\psi$
\begin{gather*}
	= \left(\psi,\frac{d}{dt}(\hat{A})\psi\right)+\frac{1}{ih}\left[(\psi,\widehat{A}\widehat{H}\psi) - (\widehat{H}\psi,\widehat{A}\psi)\right] = 
\end{gather*}
Так как $\widehat{H}$ - эрмитов, то его можно переносить из ячейки в ячейку
\begin{gather*}
	= \left(\psi,\frac{d}{dt}(\hat{A})\psi\right)+\frac{1}{ih}\left[(\psi,\widehat{A}\widehat{H}\psi) - (\psi,\widehat{H}\widehat{A}\psi)\right] = \left(\psi,\frac{d}{dt}(\hat{A})\psi\right)+\frac{1}{ih}\left[(\psi,(\hat{A}\hat{H}-\hat{H}\hat{A})\psi)\right] =\\ = \left(\psi,\frac{d}{dt}(\hat{A})\psi\right)+\frac{1}{ih}(\psi,[\hat{A},\hat{H}]\psi)
\end{gather*}
Таким образом получаем:
\begin{gather*}
	\frac{d}{dt}<A(t)> = <\frac{d}{dt}\hat{A}>+\frac{1}{ih}<[\hat{A},\hat{H}]>
\end{gather*}
В случае если нет явной зависимости от времени, тогда
\begin{gather*}
	\frac{d}{dt}<A>=\frac{1}{ih}<[\hat{A},\hat{H}]>
\end{gather*}
Вдобавок заметим, если $[\hat{A},\hat{H}] = 0$ , то $\frac{d}{dt}<A> = 0 \Rightarrow \;\;<A>$ - интеграл движения
\end{document}