\documentclass[12pt]{article}

\usepackage{fontspec}
\usepackage{polyglossia}
\usepackage[a4paper,
            total={170mm,255mm},
            left=10mm,
            top=15mm]
            {geometry}
\usepackage{graphicx}

\usepackage{amsmath,
            amsthm,
            amssymb
}

\usepackage{unicode-math,
            tensor
}

\usepackage{wrapfig, 
            hyperref,
            multicol,
            multirow,
            tabularx,
            booktabs,
            makecell,
            subfiles
}

\setdefaultlanguage{russian}
\setotherlanguage{english}
\setkeys{russian}{babelshorthands=true}

\defaultfontfeatures{Ligatures=TeX}
\setmainfont{STIX Two Text}
\setmathfont{STIX Two Math}
\DeclareSymbolFont{letters}{\encodingdefault}{\rmdefault}{m}{it}

\newfontfamily{\cyrillicfont}{STIX Two Text} 
\newfontfamily{\cyrillicfontrm}{STIX Two Text}
\newfontfamily{\cyrillicfonttt}{Courier New}
\newfontfamily{\cyrillicfontsf}{STIX Two Text}

\renewcommand{\thefigure}{\thesection.\arabic{figure}}
\renewcommand{\thetable}{\thesection.\arabic{table}}
\numberwithin{equation}{section}

\renewcommand{\qedsymbol}{$\blacksquare$}
\theoremstyle{definition}
\newtheorem{definition}{Опр.}[section]
\theoremstyle{remark}
\newtheorem{statement}{Утв.}[section]
\theoremstyle{plain}
\newtheorem{theorem}{Теор.}[section]

\addto\captionsrussian{
  \renewcommand{\figurename}{Рис.}
  \renewcommand{\tablename}{Табл.}
  \renewcommand{\proofname}{Док-во}
}

\graphicspath{{./img/}}
\everymath{\displaystyle}

\newcommand{\RNumb}[1]{\uppercase\expandafter{\romannumeral#1\relax}}

\newcommand{\llabel}[1]{\label{\thesubsection:#1}}
\newcommand{\lref}[1]{\ref{\thesubsection:#1}}


\begin{document}

\paragraph{26}Воспользуйтесь постулатами Бора, чтобы получить комбинационное правило Ритберга-Ритца.\\
\begin{definition}
	Комбинацио́нный при́нцип Ри́тца — основной закон спектроскопии, установленный эмпирически Вальтером Ритцем в 1908 году. Согласно этому принципу всё многообразие спектральных линий какого-либо элемента может быть представлено через комбинации неких величин, получивших название термы.Спектроскопическое волновое число (не путать с волновым вектором k) каждой спектральной линии можно выразить через разность двух термов:
\end{definition}
\begin{flalign}
\overbrace{\nu}=\frac{1}{\lambda}=T_{n_1} -T_{n_2}
\end{flalign}
\begin{flalign}
	T_n = \frac{R_n}{n^2}
\end{flalign}
$R_n$ - постоянная Ритберга\\


Определим энергию, отвечающую боровским стационарным состояниям:

\begin{flalign}
\varepsilon_n = \frac{mv^2}{2}-\frac{kZe^2}{r}=-\frac{kZe^2}{2r}= -\frac{Z^2}{n^2} \frac{k^2*m*e^4}{2\hbar^2}
\end{flalign}
Используя правило частот Бора:
\begin{flalign}
	\hbar\omega_(n_2 - n_1)=\varepsilon_{n_2}-\varepsilon_{n_1}=\frac{k^2 m e^4}{2 \hbar^2}Z^2(\frac{1}{n_1^2}-\frac{1}{n_2^2})
\end{flalign}
Подставим следующее вместо $\omega_(n_2 - n_1)$:
\begin{flalign}
	\frac{2\pi c}{\lambda_(n_2 n_1)} 
\end{flalign}
Следовательно:
\begin{flalign}
	\frac{1}{\lambda_(n_2 n_1)} = \frac{k^2 m e^4}{4\pi c \hbar^3}Z^2(\frac{1}{n_1^2}-\frac{1}{n_2^2})
\end{flalign}
\begin{flalign}
	\frac{k^2 m e^4}{4\pi c \hbar^3}=R_n \text{ -- константа Ритберга}
\end{flalign}
А принимая $Z$ за 1, получим комбинационное правило Ритберга-Ритца.



\end{document}