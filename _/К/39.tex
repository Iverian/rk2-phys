\documentclass[12pt]{article}

\usepackage{fontspec}
\usepackage{polyglossia}
\usepackage[a4paper,
            total={170mm,255mm},
            left=10mm,
            top=15mm]
            {geometry}
\usepackage{graphicx}

\usepackage{amsmath,
            amsthm,
            amssymb
}

\usepackage{unicode-math,
            tensor
}

\usepackage{wrapfig, 
            hyperref,
            multicol,
            multirow,
            tabularx,
            booktabs,
            makecell,
            subfiles
}

\setdefaultlanguage{russian}
\setotherlanguage{english}
\setkeys{russian}{babelshorthands=true}

\defaultfontfeatures{Ligatures=TeX}
\setmainfont{STIX Two Text}
\setmathfont{STIX Two Math}
\DeclareSymbolFont{letters}{\encodingdefault}{\rmdefault}{m}{it}

\newfontfamily{\cyrillicfont}{STIX Two Text} 
\newfontfamily{\cyrillicfontrm}{STIX Two Text}
\newfontfamily{\cyrillicfonttt}{Courier New}
\newfontfamily{\cyrillicfontsf}{STIX Two Text}

\renewcommand{\thefigure}{\thesection.\arabic{figure}}
\renewcommand{\thetable}{\thesection.\arabic{table}}
\numberwithin{equation}{section}

\renewcommand{\qedsymbol}{$\blacksquare$}
\theoremstyle{definition}
\newtheorem{definition}{Опр.}[section]
\theoremstyle{remark}
\newtheorem{statement}{Утв.}[section]
\theoremstyle{plain}
\newtheorem{theorem}{Теор.}[section]

\addto\captionsrussian{
  \renewcommand{\figurename}{Рис.}
  \renewcommand{\tablename}{Табл.}
  \renewcommand{\proofname}{Док-во}
}

\graphicspath{{./img/}}
\everymath{\displaystyle}

\newcommand{\RNumb}[1]{\uppercase\expandafter{\romannumeral#1\relax}}

\newcommand{\llabel}[1]{\label{\thesubsection:#1}}
\newcommand{\lref}[1]{\ref{\thesubsection:#1}}


\begin{document}
\paragraph{39} Тепловое излучение и люминесценция. Законы теплового излучения.\\

Свет, испускаемый источником света, уносит с собой энергию. Ответ на вопрос о происхождении этой энергии лежит в основе классификации типов излучения. \\
\begin{definition}
Тепловым излучением называется испускание ЭВМ за счет внутренней энергии  тел.
\end{definition}
Энергия, излучаемая телом, равна убыли внутренней (тепловой) энергии.\\
Все остальные виды излучения относятся к люминесценции.\\
Примеры люминесценции:
\begin{itemize}
	\item Хемолюминесценция --- светящийся в темноте гнилой пень. Излучение сопровождается химической реакцией (вещество окисляется на воздухе).
	\item Катодолюминесценция --- зеленое пятно на экране ЭЛТ осциллографа. Необходимая энергия сообщается материалу, нанесенному на экран, бомбардирующими его электронами.
	\item Люминесцентные лампы --- разряд возбуждает содержащиеся в лампе атомы газа, они высвечиваются УФ, УФ поглощается люминесцентным слоем, покрывающим внутреннюю поверхность лампы.
\end{itemize}

Законы теплового излучения.\\
1) Закон Кирхгофа.\\
Пусть $r_{\omega}$ --- испускающая способность. Тогда энергия, которую площадка $ds$ испускает за единицу времени на интервале частот $[\omega; \omega+d\omega]$ равна $r_\omega d\omega ds$. \\
Поглощающая способность $a_\omega$ определяется как отношение поглощенной площадкой $ds$ на интервале $[\omega;\omega+d\omega]$ энергии к падающей на эту же площадку на том же интервале энергии ($E_{\text{пад}}$).\\
Тогда в состоянии равновесия:
$$
a_\omega \cdot E_{\text{пад}} = r_\omega d\omega ds \Longrightarrow a_\omega\cdot \frac{1}{4}cu_\omega(\tau)d\omega ds = r_\omega d\omega ds,
$$
где $cu_\omega(\tau)d\omega$ --- плотность потока энергии в частотном интервале $[\omega; \omega+d\omega]$.\\
Из этого и следует закон Кирхгофа:
$$
\frac{1}{4}cu_\omega(\tau)=\frac{r_\omega}{a_\omega}
$$
2) Закон смещения Вина. \\
Из термодинамических соображений:
$$
u_\omega(\tau)=\omega^3f\left(\frac{\omega}{\tau}\right)
$$
$$
f\left(\frac{\omega}{\tau}\right)\longrightarrow 0 \text{  при  } \omega \longrightarrow \infty
$$
Пусть $x\equiv \frac{\omega}{\tau}$. Найдем частоту $\omega_m$, доставляющую $max(u_\omega (\tau)$.
$$
3\omega_m^2f(x_m)+\frac{\omega^3}{\tau}f'(x_m)=0 \Longrightarrow x_m=C, 
$$
где $C$ --- число, не зависящее от $\tau$ и $\omega$.\\
Следовательно,
$$
\lambda_m\tau=\sigma=2.90\cdot 10^{-3}\text{м}\cdot K
$$
3) Закон Стефана-Больцмана.\\
Энергетическая светимость АЧТ
$$
R^*=\int\limits^{\infty}\limits_0 r^*_\omega d\omega \sim \int\limits^{\infty}\limits_0 u_\omega(\tau)d\omega =\tau^4 \int\limits^{\infty}\limits_0 \frac{\omega^3}{\tau^3}f\left(\frac{\omega}{\tau}\right)d\frac{\omega}{\tau}
$$
$$
R^*=\sigma \tau^4
$$
$$
\sigma = 5.67\cdot 10^{-8} \frac{\text{Вт}}{\text{м}^2\cdot \text{К}^4}
$$
\end{document}