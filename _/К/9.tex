\documentclass[12pt]{article}

\usepackage{fontspec}
\usepackage{polyglossia}
\usepackage[a4paper,
            total={170mm,255mm},
            left=10mm,
            top=15mm]
            {geometry}
\usepackage{graphicx}

\usepackage{amsmath,
            amsthm,
            amssymb
}

\usepackage{unicode-math,
            tensor
}

\usepackage{wrapfig, 
            hyperref,
            multicol,
            multirow,
            tabularx,
            booktabs,
            makecell,
            subfiles
}

\setdefaultlanguage{russian}
\setotherlanguage{english}
\setkeys{russian}{babelshorthands=true}

\defaultfontfeatures{Ligatures=TeX}
\setmainfont{STIX Two Text}
\setmathfont{STIX Two Math}
\DeclareSymbolFont{letters}{\encodingdefault}{\rmdefault}{m}{it}

\newfontfamily{\cyrillicfont}{STIX Two Text} 
\newfontfamily{\cyrillicfontrm}{STIX Two Text}
\newfontfamily{\cyrillicfonttt}{Courier New}
\newfontfamily{\cyrillicfontsf}{STIX Two Text}

\renewcommand{\thefigure}{\thesection.\arabic{figure}}
\renewcommand{\thetable}{\thesection.\arabic{table}}
\numberwithin{equation}{section}

\renewcommand{\qedsymbol}{$\blacksquare$}
\theoremstyle{definition}
\newtheorem{definition}{Опр.}[section]
\theoremstyle{remark}
\newtheorem{statement}{Утв.}[section]
\theoremstyle{plain}
\newtheorem{theorem}{Теор.}[section]

\addto\captionsrussian{
  \renewcommand{\figurename}{Рис.}
  \renewcommand{\tablename}{Табл.}
  \renewcommand{\proofname}{Док-во}
}

\graphicspath{{./img/}}
\everymath{\displaystyle}

\newcommand{\RNumb}[1]{\uppercase\expandafter{\romannumeral#1\relax}}

\newcommand{\llabel}[1]{\label{\thesubsection:#1}}
\newcommand{\lref}[1]{\ref{\thesubsection:#1}}


\begin{document}
	\paragraph{9}
	Вычислить постоянную Стефана-Больцмана, воспользовавшись формулой Планка.\\
	
	%Энергетическую светимость абсолютно чёрного тела определим, как:
	
	%\begin{gather}
%		ε_T = \int\limits_{0}^{\infty} ε_{ν,T} dν
%	\end{gather}

%	Используем формулу Планка для излучательной способности абсолютно чёрного %тела:

%	\begin{gather}
%		ε_{ν,T} = \frac{2πν^3}{c^2} \frac{h}{exp(\frac{hν}{kT}-1)}
%	\end{gather}
%	Подставим (0.2) в (0.1), получим интеграл:
	
%	\begin{gather}
%		ε_T = \int\limits_{0}^{\infty}\frac{2πν^3}{c^2} %\frac{h}{exp(\frac{hν}{kT}-1)}dν = \frac{2\pi h}{c^2} %\int\limits_{0}^{\infty}\frac{ν^3}{exp(\frac{hν}{kT}-1)}dν
%	\end{gather}

%	Проведём замену переменных, подставим $x = \frac{h\nu}{kT} \rightarrow \nu %= \frac{xkT}{h}$, тогда $dν=\frac{kTdx}{h}$ и интеграл в (0.3) преобразуется %к виду:
	
%	\begin{gather}
%		ε_T=\frac
%		{2\pi h}{c^2}
%		\int\limits_{0}^{\infty}
%		\frac{(\frac{xkT}{h})^3 \frac{kTdx}{h}}{exp(x)-1} = 
%		\frac{2\pi (kT)^4}{c^2h^3}
%		\int\limits_{0}^{\infty}
%		\frac{x^3 σ exp(-x)dx}{1-exp(-x)}		
%	\end{gather}

%	Разложим значение %

Попробуем вычислить постоянную С-Б при помощи формулы Планка:
	\begin{gather*}
	U_\omega(T)=\frac{\hbar\omega^3}{\pi^2c^3}\cdot\frac{1}{e^{\frac{\hbar\omega}{K_BT}}-1}
	\end{gather*}
	Запишем энергетическую светимость:
	\begin{gather*}
	R^\ast(T)
	=
	\int_{0}^{\infty}d\omega r^\ast_\omega(T)
	=
	\int_{0}^{\infty}d\omega\frac{c}{4}U_\omega(T)
	=\\
	=
	\left.
	\int_{0}^{\infty}\frac{\hbar\omega^3}{4\pi^2c^2}\cdot\frac{d\omega}{e^{\frac{\hbar\omega}{K_BT}}-1}
	\right|_{x=\frac{\hbar\omega}{K_BT}}
	=
	\frac{\hbar}{4\pi^2c^2}\left(\frac{K_BT}{\hbar}\right)^4\int_{0}^{\infty}\frac{x^3\cdot dx}{e^x-1}
	\end{gather*}
	Посчитаем интеграл отдельно:
	\begin{gather*}
	\int_{0}^{\infty}\frac{x^3dx}{e^x-1}
	=
	\int_{0}^{\infty}dx\cdot x^3\frac{e^{-x}}{1-e^{-x}}
	=
	\int_{0}^{\infty}dx\cdot x^3\sum_{p=1}^{\infty}e^{-px}
	=\\
	\left.
	\sum_{p=1}^{\infty}\int_{0}^{\infty}dx\cdot x^3e^{-px}
	\right|_{y=-px}
	=
	\sum_{p=1}^{\infty}\left(\frac{1}{p^4}\right)\int_{0}^{\infty}dy\cdot y^3e^{-y}
	=
	\zeta(4)\cdot\Gamma(4)
	=
	\frac{\pi^4}{15},
	\end{gather*}
	где $\zeta(y)$ -- дзета-функция Римана.
	Получаем $\sigma=\frac{\pi^2K_B^4}{60c^2\hbar^3}$, т.к. з-н Стефана-Больцмана: $R^\ast(T)=\sigma T^4$
	
	
\end{document}