\documentclass[12pt]{article}

\usepackage{fontspec}
\usepackage{polyglossia}
\usepackage[a4paper,
            total={170mm,255mm},
            left=10mm,
            top=15mm]
            {geometry}
\usepackage{graphicx}

\usepackage{amsmath,
            amsthm,
            amssymb
}

\usepackage{unicode-math,
            tensor
}

\usepackage{wrapfig, 
            hyperref,
            multicol,
            multirow,
            tabularx,
            booktabs,
            makecell,
            subfiles
}

\setdefaultlanguage{russian}
\setotherlanguage{english}
\setkeys{russian}{babelshorthands=true}

\defaultfontfeatures{Ligatures=TeX}
\setmainfont{STIX Two Text}
\setmathfont{STIX Two Math}
\DeclareSymbolFont{letters}{\encodingdefault}{\rmdefault}{m}{it}

\newfontfamily{\cyrillicfont}{STIX Two Text} 
\newfontfamily{\cyrillicfontrm}{STIX Two Text}
\newfontfamily{\cyrillicfonttt}{Courier New}
\newfontfamily{\cyrillicfontsf}{STIX Two Text}

\renewcommand{\thefigure}{\thesection.\arabic{figure}}
\renewcommand{\thetable}{\thesection.\arabic{table}}
\numberwithin{equation}{section}

\renewcommand{\qedsymbol}{$\blacksquare$}
\theoremstyle{definition}
\newtheorem{definition}{Опр.}[section]
\theoremstyle{remark}
\newtheorem{statement}{Утв.}[section]
\theoremstyle{plain}
\newtheorem{theorem}{Теор.}[section]

\addto\captionsrussian{
  \renewcommand{\figurename}{Рис.}
  \renewcommand{\tablename}{Табл.}
  \renewcommand{\proofname}{Док-во}
}

\graphicspath{{./img/}}
\everymath{\displaystyle}

\newcommand{\RNumb}[1]{\uppercase\expandafter{\romannumeral#1\relax}}

\newcommand{\llabel}[1]{\label{\thesubsection:#1}}
\newcommand{\lref}[1]{\ref{\thesubsection:#1}}


\begin{document}
	\paragraph{32}
	Спектр операторов $\hat{J}^2$ и $\hat{J}_z$ ($\hat{J}$ -- оператор полного момента импульса). Вычислите постоянную Стефана-Больцмана, использую формулу Планка\\
	
	Запишем уравнение $\hat{J}_z\Psi(\bar r)=J_z\Psi(\bar{r})$ в сферических координатах $\left(\bar{r}=\{r,\theta,\varphi\}\right):$
	\begin{gather*}
	-ih\frac{\partial\Psi}{\partial\varphi}=J_z\Psi\Rightarrow\\
	\Psi(\varphi)=C\exp\left[iJ_z\cdot \varphi/h\right]=C\cdot e^{im\varphi},\text{ где } m=\frac{J_z}{h}.
	\end{gather*}	

	Функция $\Psi(\varphi)$ должна быть периодической, т.е. при повороте на угол 2π ничего не меняется. Следовательно:
	\begin{gather*}
	e^{im\varphi}=e^{im(\varphi+2π)}.
	\end{gather*}
	Это равенство справедливо только при $m\in\mathbb Z$ - СЗ оператора $\hat{J}_z.$\\
	
	Перейдём к уравнению $\hat{J}^2\Psi=J^2\Psi$ в сферических координатах:
	\begin{gather*}
	\frac{1}{\sin\theta}\frac{\partial}{\partial\theta}\left(\sin\theta\frac{\partial\Psi}{\partial\theta}\right)+\frac{\partial^2\Psi}{\sin^2\theta\partial\varphi^2}+\lambda\Psi=0,\text{ где }\lambda=\frac{J^2}{h^2}
	\end{gather*}
	Мы пришли к никому не известному уравнению Штурма-Лиувилля, которое решается методом разделения переменных Фурье: $\Psi\left(\theta,\varphi\right)=F(\theta)\Phi(\varphi).$ Конечные решения этого ур-я $\left(\vert\Psi(\theta,\varphi)\vert<\infty\right)$ получаются только при положительных целых чётных λ (СЗ оператора $\hat{J}^2$):
	\begin{gather*}
	\lambda=\frac{J^2}{h^2}=l(l+1),\text{ где }l\in\mathbb{N}\cup0
	\end{gather*}
	СФ этого оператора можно нумеровать квантовым числом $l$, однако каждому СЗ $h^2l(l+1)$ соответствует $2l+1$ разных СФ $\Psi_l$, различающихся проекциями момента импульса на ось $Z.$ Чтобы их отличать, вводят дополнительный нумератор $m : \vert m\vert\le l$
	
	
	Теперь попробуем вычислить постоянную С-Б при помощи формулы Планка:
	\begin{gather*}
	U_\omega(T)=\frac{\hbar\omega^3}{\pi^2c^3}\cdot\frac{1}{e^{\frac{\hbar\omega}{K_BT}}-1}
	\end{gather*}
	Запишем энергетическую светимость:
	\begin{gather*}
	R^\ast(T)
	=
	\int_{0}^{\infty}d\omega r^\ast_\omega(T)
	=
	\int_{0}^{\infty}d\omega\frac{c}{4}U_\omega(T)
	=\\
	=
	\left.
	\int_{0}^{\infty}\frac{\hbar\omega^3}{4\pi^2c^2}\cdot\frac{d\omega}{e^{\frac{\hbar\omega}{K_BT}}-1}
	\right|_{x=\frac{\hbar\omega}{K_BT}}
	=
	\frac{\hbar}{4\pi^2c^2}\left(\frac{K_BT}{\hbar}\right)^4\int_{0}^{\infty}\frac{x^3\cdot dx}{e^x-1}
	\end{gather*}
	Посчитаем интеграл отдельно:
	\begin{gather*}
	\int_{0}^{\infty}\frac{x^3dx}{e^x-1}
	=
	\int_{0}^{\infty}dx\cdot x^3\frac{e^{-x}}{1-e^{-x}}
	=
	\int_{0}^{\infty}dx\cdot x^3\sum_{p=1}^{\infty}e^{-px}
	=\\
	\left.
	\sum_{p=1}^{\infty}\int_{0}^{\infty}dx\cdot x^3e^{-px}
	\right|_{y=-px}
	=
	\sum_{p=1}^{\infty}\left(\frac{1}{p^4}\right)\int_{0}^{\infty}dy\cdot y^3e^{-y}
	=
	\zeta(4)\cdot\Gamma(4)
	=
	\frac{\pi^4}{15},
	\end{gather*}
	где $\zeta(y)$ -- дзета-функция Римана.
	Получаем $\sigma=\frac{\pi^2K_B^4}{60c^2\hbar^3}$, т.к. з-н Стефана-Больцмана: $R^\ast(T)=\sigma T^4$
\end{document}