\documentclass[12pt]{article}

\usepackage{fontspec}
\usepackage{polyglossia}
\usepackage[a4paper,
            total={170mm,255mm},
            left=10mm,
            top=15mm]
            {geometry}
\usepackage{graphicx}

\usepackage{amsmath,
            amsthm,
            amssymb
}

\usepackage{unicode-math,
            tensor
}

\usepackage{wrapfig, 
            hyperref,
            multicol,
            multirow,
            tabularx,
            booktabs,
            makecell,
            subfiles
}

\setdefaultlanguage{russian}
\setotherlanguage{english}
\setkeys{russian}{babelshorthands=true}

\defaultfontfeatures{Ligatures=TeX}
\setmainfont{STIX Two Text}
\setmathfont{STIX Two Math}
\DeclareSymbolFont{letters}{\encodingdefault}{\rmdefault}{m}{it}

\newfontfamily{\cyrillicfont}{STIX Two Text} 
\newfontfamily{\cyrillicfontrm}{STIX Two Text}
\newfontfamily{\cyrillicfonttt}{Courier New}
\newfontfamily{\cyrillicfontsf}{STIX Two Text}

\renewcommand{\thefigure}{\thesection.\arabic{figure}}
\renewcommand{\thetable}{\thesection.\arabic{table}}
\numberwithin{equation}{section}

\renewcommand{\qedsymbol}{$\blacksquare$}
\theoremstyle{definition}
\newtheorem{definition}{Опр.}[section]
\theoremstyle{remark}
\newtheorem{statement}{Утв.}[section]
\theoremstyle{plain}
\newtheorem{theorem}{Теор.}[section]

\addto\captionsrussian{
  \renewcommand{\figurename}{Рис.}
  \renewcommand{\tablename}{Табл.}
  \renewcommand{\proofname}{Док-во}
}

\graphicspath{{./img/}}
\everymath{\displaystyle}

\newcommand{\RNumb}[1]{\uppercase\expandafter{\romannumeral#1\relax}}

\newcommand{\llabel}[1]{\label{\thesubsection:#1}}
\newcommand{\lref}[1]{\ref{\thesubsection:#1}}


\begin{document}

\paragraph{1}
Комбинационное правило Ритберга-Ритца, спектральные серии для атома водорода, постулаты Бора.\\

\textit{Комбинационное правило Ритберга-Ритца:} принцип, гласящий, что все спектральные линии некоторого элемента могут быть представлены через комбинации величин, называемых \textit{термами}.
\begin{gather*}
\forall n_1,n_2\in\mathbb{N} \ n_1 < n_2 \colon \frac{1}{\lambda_{n_1 n_2}} = T_{n_1} - T_{n_2},
\end{gather*}
где $\lambda_{n_1 n_2}$ -- длина волны спектральной линии $n_1 n_2$ ($n_1<n_2$), $T_{n_1}, T_{n_2}$ -- термы, соответствующие $n_1$ и $n_2$.

Покажем как с помощью правила выразить одну спектральную линию через две другие:
\begin{gather*}
\frac{1}{\lambda_{n_1 n_2}}
=
T_{n_1} - T_{n_2}
=
T_{n_1}-T_{n_3} - (T_{n_2}-T_{n_3})
=
\frac{1}{\lambda_{n_1 n_3}} - \frac{1}{\lambda_{n_3 n_2}}
\end{gather*}

Термы для атома водорода вычисляются следующим образом:
$
\forall n\in\mathbb{N}\colon T_n = \frac{R_H}{n^2},
$
где $R_H$ -- постоянная Ритберга для водорода.

Для водорода также выделяют некоторые серии спектральных линий:\\
\begin{table}[h]
\centering
\begin{tabular}{llr}
\toprule
Серия & Характер излучения & $1/\lambda_{n_1 n_2}$ \\
\midrule
Лаймана & Ультрафиолет & $\forall n>1\colon R_{H}\left(1-\frac{1}{n^2}\right)$ \\
& & \\
Бальмера & Видимое & $\forall n>2\colon R_{H}\left(\frac{1}{2^2}-\frac{1}{n^2}\right)$ \\
& & \\
Пашена & Инфракрасное & $\forall n>3\colon R_{H}\left(\frac{1}{3^2}-\frac{1}{n^2}\right)$ \\
& & \\
Брэккета & дальний ИК & $\forall n>4\colon R_{H}\left(\frac{1}{4^2}-\frac{1}{n^2}\right)$ \\
& & \\
Пфунда & дальний ИК & $\forall n>5\colon R_{H}\left(\frac{1}{5^2}-\frac{1}{n^2}\right)$ \\
\bottomrule
\end{tabular}
\end{table}

\textit{Постулаты Бора:}
\begin{enumerate}
\item Атом может длительно пребывать только в стационарных состояниях, характеризующихся определенной энергией $E_1,E_2,...$. В таких состояниях атомы не излучают электромагнитных волн.
\item Излучение света происходит при переходе атома из состояния с большей энергией в состояние с меньшей энергией. Энергия излученного фотона равна разности энергий стационарных состояний.
\begin{gather*}
\hbar\omega_{n_1 \leftarrow n_2} = E_{n_2} - E_{n_1}
\end{gather*}
\item \textit{Правило квантования круговых орбит электрона:} момент импульса электрона, вращающегося на стационарной орбите атома водорода может принимать только дискретные значения
\begin{gather*}
\forall n\in\mathbb{N}\colon L = n\hbar
\end{gather*}
\end{enumerate}

\end{document}