
\documentclass[12pt]{article}

\usepackage{fontspec}
\usepackage{polyglossia}
\usepackage[a4paper,
            total={170mm,255mm},
            left=10mm,
            top=15mm]
            {geometry}
\usepackage{graphicx}

\usepackage{amsmath,
            amsthm,
            amssymb
}

\usepackage{unicode-math,
            tensor
}

\usepackage{wrapfig, 
            hyperref,
            multicol,
            multirow,
            tabularx,
            booktabs,
            makecell,
            subfiles
}

\setdefaultlanguage{russian}
\setotherlanguage{english}
\setkeys{russian}{babelshorthands=true}

\defaultfontfeatures{Ligatures=TeX}
\setmainfont{STIX Two Text}
\setmathfont{STIX Two Math}
\DeclareSymbolFont{letters}{\encodingdefault}{\rmdefault}{m}{it}

\newfontfamily{\cyrillicfont}{STIX Two Text} 
\newfontfamily{\cyrillicfontrm}{STIX Two Text}
\newfontfamily{\cyrillicfonttt}{Courier New}
\newfontfamily{\cyrillicfontsf}{STIX Two Text}

\renewcommand{\thefigure}{\thesection.\arabic{figure}}
\renewcommand{\thetable}{\thesection.\arabic{table}}
\numberwithin{equation}{section}

\renewcommand{\qedsymbol}{$\blacksquare$}
\theoremstyle{definition}
\newtheorem{definition}{Опр.}[section]
\theoremstyle{remark}
\newtheorem{statement}{Утв.}[section]
\theoremstyle{plain}
\newtheorem{theorem}{Теор.}[section]

\addto\captionsrussian{
  \renewcommand{\figurename}{Рис.}
  \renewcommand{\tablename}{Табл.}
  \renewcommand{\proofname}{Док-во}
}

\graphicspath{{./img/}}
\everymath{\displaystyle}

\newcommand{\RNumb}[1]{\uppercase\expandafter{\romannumeral#1\relax}}

\newcommand{\llabel}[1]{\label{\thesubsection:#1}}
\newcommand{\lref}[1]{\ref{\thesubsection:#1}}


\begin{document}

\paragraph{43}
Рассмотрите задачу об электроне в бесконечно глубокой потенциальной яме. Чему равна минимальная кинетическая энергия электрона? Какова вероятность обнаружить электрон в интервале $L/6<x<L/3$ (где $L$ ширина ямы) во втором возбуждённом состоянии?\\

Рассмотрим электрон в потенциальной яме, $0<x<l$.
Движение электрона будет задаваться волновой функцией:
\begin{gather}
 \frac{\hbar^{2}}{2m}\frac{d^{2}\psi}{dx^{2}}+(\varepsilon-V_{0})\psi=0,
\end{gather}
где $k=\frac{\sqrt{2m(\varepsilon-V_{0})}}{\hbar}$ ($V=const$)
Решение данного дифференциального уравнения:

\begin{gather}
\psi=A_{21}e^{9kx}+B_{2}e^{-9kx}.
\end{gather}

Граничные условия внутри потенциальной ямы:

$$
v(x)=
\left\{
\begin{gathered}
\psi(0)=0\\
\psi(l)=0\\
\end{gathered}
\right.
$$

И так решение диффура: 

\begin{gather}
\psi_{n}=2A sin{k_{n}x}
\end{gather}
а $k_{n}=\frac{\pi n}{l}.$

Энергия электрона:

\begin{gather}
T_{n}=\frac{\pi^{2}\hbar^{2}}{2m^{2}l^{2}}{}n^{2}.
\end{gather}

Тогда минимальная энергия электрона:

\begin{gather}
T_{min}=T_{n}=\frac{\pi^{2}\hbar^{2}}{2m^{2}l^{2}}.
\end{gather}
Но стоит вспомнить, что в квантовой механике квадрируемая функция нормирована единицей, откуда $A=\frac{1}{\sqrt{2l}}$

\begin{gather}
\psi_{n}=\sqrt{\dfrac{2}{l}} sin{\frac{\pi n x}{l}}.
\end{gather}

Вероятность обнаружения электрон в состоянии $\psi_{2}$ :

\begin{gather}
P_{2}(L/6<x<L/3)=\int_{l/6}^{l/3}dx|\psi_{2}(x)|^{2}.
\end{gather}



\end{document}