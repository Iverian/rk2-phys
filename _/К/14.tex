\documentclass[12pt]{article}

\usepackage{fontspec}
\usepackage{polyglossia}
\usepackage[a4paper,
            total={170mm,255mm},
            left=10mm,
            top=15mm]
            {geometry}
\usepackage{graphicx}

\usepackage{amsmath,
            amsthm,
            amssymb
}

\usepackage{unicode-math,
            tensor
}

\usepackage{wrapfig, 
            hyperref,
            multicol,
            multirow,
            tabularx,
            booktabs,
            makecell,
            subfiles
}

\setdefaultlanguage{russian}
\setotherlanguage{english}
\setkeys{russian}{babelshorthands=true}

\defaultfontfeatures{Ligatures=TeX}
\setmainfont{STIX Two Text}
\setmathfont{STIX Two Math}
\DeclareSymbolFont{letters}{\encodingdefault}{\rmdefault}{m}{it}

\newfontfamily{\cyrillicfont}{STIX Two Text} 
\newfontfamily{\cyrillicfontrm}{STIX Two Text}
\newfontfamily{\cyrillicfonttt}{Courier New}
\newfontfamily{\cyrillicfontsf}{STIX Two Text}

\renewcommand{\thefigure}{\thesection.\arabic{figure}}
\renewcommand{\thetable}{\thesection.\arabic{table}}
\numberwithin{equation}{section}

\renewcommand{\qedsymbol}{$\blacksquare$}
\theoremstyle{definition}
\newtheorem{definition}{Опр.}[section]
\theoremstyle{remark}
\newtheorem{statement}{Утв.}[section]
\theoremstyle{plain}
\newtheorem{theorem}{Теор.}[section]

\addto\captionsrussian{
  \renewcommand{\figurename}{Рис.}
  \renewcommand{\tablename}{Табл.}
  \renewcommand{\proofname}{Док-во}
}

\graphicspath{{./img/}}
\everymath{\displaystyle}

\newcommand{\RNumb}[1]{\uppercase\expandafter{\romannumeral#1\relax}}

\newcommand{\llabel}[1]{\label{\thesubsection:#1}}
\newcommand{\lref}[1]{\ref{\thesubsection:#1}}


\begin{document}
\paragraph{14}
Постулаты квантовой механики: о квантовых состояниях, о физических величинах, об измерениях, динамический постулат.\\

\textbf{Постулат о квантовых состояниях.}\\
Квантовое состояние полностью задается $\psi$-функцией из пространства волновых функций. $\psi$-функции, отличающиеся только комплексным множителем, задают одно и то же состояние.\\

\textbf{Постулат о физических величинах.}\\
Каждой физической величине ставится в соответствие эрмитов оператор, обладающий полной системой собственных функций.\\

\textbf{Постулат об измерениях.}\\
Пусть измеряется некоторая величина $A$ и ей в соответствие поставлен оператор $\hat{A}: A \rightarrow \hat{A}$. Если оператор эрмитов (т.е. $\hat{A}^{+}=\hat{A}$, то его собственные значения вещественны и $\hat{A}\varphi_n=a_n\varphi_n$.\\

\textbf{Динамический постулат.}\\
Все предсказания, которые могут быть сделаны относительно различных свойств системы в данный момент времени, следуют из значения $\psi$-функции в этот момент времени.\\
\end{document}