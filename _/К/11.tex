\documentclass[12pt]{article}

\usepackage{fontspec}
\usepackage{polyglossia}
\usepackage[a4paper,
            total={170mm,255mm},
            left=10mm,
            top=15mm]
            {geometry}
\usepackage{graphicx}

\usepackage{amsmath,
            amsthm,
            amssymb
}

\usepackage{unicode-math,
            tensor
}

\usepackage{wrapfig, 
            hyperref,
            multicol,
            multirow,
            tabularx,
            booktabs,
            makecell,
            subfiles
}

\setdefaultlanguage{russian}
\setotherlanguage{english}
\setkeys{russian}{babelshorthands=true}

\defaultfontfeatures{Ligatures=TeX}
\setmainfont{STIX Two Text}
\setmathfont{STIX Two Math}
\DeclareSymbolFont{letters}{\encodingdefault}{\rmdefault}{m}{it}

\newfontfamily{\cyrillicfont}{STIX Two Text} 
\newfontfamily{\cyrillicfontrm}{STIX Two Text}
\newfontfamily{\cyrillicfonttt}{Courier New}
\newfontfamily{\cyrillicfontsf}{STIX Two Text}

\renewcommand{\thefigure}{\thesection.\arabic{figure}}
\renewcommand{\thetable}{\thesection.\arabic{table}}
\numberwithin{equation}{section}

\renewcommand{\qedsymbol}{$\blacksquare$}
\theoremstyle{definition}
\newtheorem{definition}{Опр.}[section]
\theoremstyle{remark}
\newtheorem{statement}{Утв.}[section]
\theoremstyle{plain}
\newtheorem{theorem}{Теор.}[section]

\addto\captionsrussian{
  \renewcommand{\figurename}{Рис.}
  \renewcommand{\tablename}{Табл.}
  \renewcommand{\proofname}{Док-во}
}

\graphicspath{{./img/}}
\everymath{\displaystyle}

\newcommand{\RNumb}[1]{\uppercase\expandafter{\romannumeral#1\relax}}

\newcommand{\llabel}[1]{\label{\thesubsection:#1}}
\newcommand{\lref}[1]{\ref{\thesubsection:#1}}


\begin{document}
	\paragraph{11}
	Операторы проекции момента импульса на выделенное направление $\hat{L}_x, \hat{L}_y, \hat{L}_z$ и оператор квадрата момента импульса, запишите и прокомментируйте коммутационные соотношения.
	\\
	Момент импульса:\\
	$$\vec{L} = \vec{r} \times \vec{p} = \left|
	\begin{matrix}
	\vec{l_x} & \vec{l_y} & \vec{l_z} \\
	x & y & z \\
	p_x & p_y & p_z
	\end{matrix}
	\right| =
	\vec{l_x}(yp_z-zp_y)+
	\vec{l_y}(zp_x-xp_z)+
	\vec{l_z}(xp_y-yp_x) 
	=
	L_x\vec{l_x}+L_y\vec{l_y}+L_z\vec{l_z}
	 $$
	 В квантовой механике постулат о физических величинах утверждает, что физической величине соответствует оператор.\\
	 Операторы проекции момента импульса на выделенные направления:\\
	 $\widehat{L_x} = \widehat{y}\widehat{p_z} - \widehat{z}\widehat{p_y} = \frac{\hbar}{i}(y \partial_z - z \partial_y)$.\\
	 $\widehat{L_y} = \widehat{z}\widehat{p_x} - \widehat{x}\widehat{p_z} = \frac{\hbar}{i}(z \partial_x - x \partial_z)$.\\
	 $\widehat{L_z} = \widehat{x}\widehat{p_y} - \widehat{y}\widehat{p_x} = \frac{\hbar}{i}(x \partial_y - y \partial_x)$.\\
	 Установим коммутационные соотношения. Сделаем мы это для того, чтобы понять можно ли одновременно точно померять соответствующие величины и если нет, то записать соотношение неопределенностей этих величин.\\
	 Так, мы можем одновременно точно измерить $p_x$ и $p_y$. $[\widehat{p_x},\widehat{p_y}]\sim[\partial_x,\partial_y] = 0$.\\
	 Мы сделаем проверку коммутатора $[\widehat{L_x},\widehat{L_y}]$ прежде чем писать 0.\\
	 $$[\widehat{L_x},\widehat{L_y}] = [y\widehat{p_z} - z\widehat{p_y},\widehat{L_y}] = [y\widehat{p_z},\widehat{L_y}] - [z\widehat{p_y},\widehat{L_y}]$$
	 $$[y\widehat{p_z},\widehat{L_y}] = y[\widehat{p_z},\widehat{L_y}]+ \underbrace{[y,\widehat{L_y}]\widehat{p_z}}_0$$
	 $$
	 [\widehat{p_z},\widehat{L_y}]
	 =
	 \left([\widehat{p_z},z\widehat{p_x}] = [\widehat{p_z},z]\widehat{p_x}+z[\widehat{p_z},\widehat{p_x}]\right)
	 -
	 \left([\widehat{p_z},x\widehat{p_z}]\right)
	 =
	 (-i\hbar+0)
	 -
	 (0)
	 =
	 -i\hbar
	 $$
	 Обобщим: $[\widehat{L_y},\widehat{p_z}] = i\hbar\widehat{p_x}$
	 Циклические перестановки:\\
	 $[\widehat{L_z},\widehat{p_x}] = i\hbar \widehat{p_y}$\\
	 $[\widehat{L_x},\widehat{p_y}] = i\hbar\widehat{p_z}$\\
	 $$[x\widehat{p_y},\widehat{L_y}] = \widehat{p_y}[z,z\widehat{p_x} - x\widehat{p_z}] = -i\hbar x\widehat{p_y}$$
	 Вывод: $[y\widehat{p_z},\widehat{L_y}] = -i\hbar y\widehat{p_x}$\\
	 Цеклической перестановкой индексов, получим: $[\widehat{L_x},\widehat{L_y}] = i\hbar \widehat{L_z}$\\
	 Тот факт, что компоненты момента импульса не коммутируют озночает , что испуользовать мы будем какую-то одну из компонент в заданный момент времени.\\
	 Теперь, когда мы знаем коммутатор $\widehat{L_x}$ и $\widehat{L_y}$, мы можем выписать соотношение неопределенностей для этих компонент:
	 $$\triangle L_x \triangle L_y \geq \frac{|<[\widehat{L_x},\widehat{L_y}]>|}{2}$$
	 $$\triangle L_x \triangle L_Y \geq \frac{h}{2} <L_z>$$
	 Квадрат момента импульса: $\vec{L}^2 = L_x ^2 + L_y ^2 + L_z ^2$.\\
	 $$ [\widehat{L_z},\widehat{L_x}^2] = [\widehat{L_z},\widehat{L_x}]\widehat{L_x}+\widehat{L_x}[\widehat{L_z},\widehat{L_x}] = [\widehat{L_z},\widehat{L_x}]\widehat{L_x}+ i\hbar \widehat{L_y}$$
	 $$ [\widehat{L_z},\widehat{L_y}^2] = [\widehat{L_z},\widehat{L_y}]\widehat{L_y} + \widehat{L_y}[\widehat{L_z},\widehat{L_y}] = [\widehat{L_z},\widehat{L_y}]\widehat{L_y} + (-i \hbar \widehat{L_x})$$
	 $$[\widehat{L_z},\widehat{L_z}^2] = 0$$
	 Теперь:
	$$ \triangle \widehat{L_z} \triangle(\widehat{L}^2) \geq \underbrace{ \frac{|<[\widehat{L_z},\widehat{L}^2]>|}{2}}_0$$
\end{document}