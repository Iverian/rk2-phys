\documentclass[12pt]{article}

\usepackage{fontspec}
\usepackage{polyglossia}
\usepackage[a4paper,
            total={170mm,255mm},
            left=10mm,
            top=15mm]
            {geometry}
\usepackage{graphicx}

\usepackage{amsmath,
            amsthm,
            amssymb
}

\usepackage{unicode-math,
            tensor
}

\usepackage{wrapfig, 
            hyperref,
            multicol,
            multirow,
            tabularx,
            booktabs,
            makecell,
            subfiles
}

\setdefaultlanguage{russian}
\setotherlanguage{english}
\setkeys{russian}{babelshorthands=true}

\defaultfontfeatures{Ligatures=TeX}
\setmainfont{STIX Two Text}
\setmathfont{STIX Two Math}
\DeclareSymbolFont{letters}{\encodingdefault}{\rmdefault}{m}{it}

\newfontfamily{\cyrillicfont}{STIX Two Text} 
\newfontfamily{\cyrillicfontrm}{STIX Two Text}
\newfontfamily{\cyrillicfonttt}{Courier New}
\newfontfamily{\cyrillicfontsf}{STIX Two Text}

\renewcommand{\thefigure}{\thesection.\arabic{figure}}
\renewcommand{\thetable}{\thesection.\arabic{table}}
\numberwithin{equation}{section}

\renewcommand{\qedsymbol}{$\blacksquare$}
\theoremstyle{definition}
\newtheorem{definition}{Опр.}[section]
\theoremstyle{remark}
\newtheorem{statement}{Утв.}[section]
\theoremstyle{plain}
\newtheorem{theorem}{Теор.}[section]

\addto\captionsrussian{
  \renewcommand{\figurename}{Рис.}
  \renewcommand{\tablename}{Табл.}
  \renewcommand{\proofname}{Док-во}
}

\graphicspath{{./img/}}
\everymath{\displaystyle}

\newcommand{\RNumb}[1]{\uppercase\expandafter{\romannumeral#1\relax}}

\newcommand{\llabel}[1]{\label{\thesubsection:#1}}
\newcommand{\lref}[1]{\ref{\thesubsection:#1}}


\begin{document}

\paragraph{41}
Приведите пример эрмитова оператора, не обладающего полной системой собственных функций.\\
\begin{definition}
	Эрмитов оператор - такой оператор $\hat{A}$ в гильбертовом пространстве, если удовлетворяет равенству $(\hat{A}x,y)=(x,\hat{A}y)$
\end{definition}
Важно знать, что собственные значения эрмитова оператора вещественны.\\
Возьмем оператор $i\frac{d}{dx}$, действующий на квадратично интегрируемые ф-ции $\psi(x)$, определенных на полуоси $(0,+\infty)$. Проверим, что этот оператор эрмитов, если его применять к функциям, обращающимся в 0 при $x=0$:
\begin{flalign}
	\int_{0}^{\infty}\psi^*_1 i \frac{d}{dx}\psi_2 dx=	\int_{0}^{\infty}(i \frac{d}{dx}\psi_1)^* \psi_2 dx=i\psi^*_1*\psi_2\mid^\infty _ 0=0
\end{flalign}
А вот СФ оператор не имеет, потому что единственно возможными СФ являются функции вида $e^ikx$ (убедитесь, решив дифур $i\frac{d}{dx}\phi_k (x)=k\phi_k (x)$), но $\phi_k (x=0) \neq 0$, так что он не обладает полной системой СФ : Вы не выразите $\psi(x)$ через линейную комбинацию $\phi_k (x)$.



\end{document}