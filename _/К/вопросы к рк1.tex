\documentclass[12pt]{article}

\usepackage{fontspec}
\usepackage{polyglossia}
\usepackage{geometry}
\usepackage{graphicx}
\usepackage[final]{pdfpages}
\usepackage{multicol}
\usepackage{enumitem}

\usepackage{amsmath,
            unicode-math,
            tensor
}

\geometry{a4paper,
          total={170mm,255mm},
          left=10mm,
          top=15mm,
}

\setmainlanguage{russian}
\setotherlanguage{english}
\setkeys{russian}{babelshorthands=true}

\defaultfontfeatures{Ligatures=TeX}
\setmainfont{STIX Two Text}
\setmathfont{STIX Two Math}

\newfontfamily{\cyrillicfont}{STIX Two Text} 
\newfontfamily{\cyrillicfontrm}{STIX Two Text}
\newfontfamily{\cyrillicfonttt}{Courier New}
\newfontfamily{\cyrillicfontsf}{STIX Two Text}

\graphicspath{{./img/}}
\everymath{\displaystyle}

\begin{document}

%
\section{РК по физике по теме: <<Квантовая механика>>}
%

\begin{enumerate}

\item\label{_1}
Комбинационное правило Ритберга-Ритца, спектральные серии для атома водорода, постулаты Бора.

\item\label{_2}
Тепловое излучение и люминесценция. Равновесное тепловое излучение: свойства, спектральная плотность энергии, температура.

\item\label{_3}
Покажите, что коммутационные соотношения наблюдаемых операторов не зависят от выбора представления. Приведите примеры.

\item\label{_4}
Энергетическая светимость, испускательная способность, поглощательная способность. Закон Кирхгофа.

\item\label{_5}
Энергетический спектр квантовомеханического гармонического осциллятора.

\item\label{_6}
Докажите, что два наблюдаемых оператора коммутируют тогда и только тогда, когда обладают общей системой собственных функций. Раскройте физическое содержание этого утверждения.

\item\label{_7}
Фотоэффект. Эффект Комптона. Интерференция электронов на двух щелях.

\item\label{_8}
Абсолютно чёрное тело: испускательная способность, энергетическая светимость. Закон Стефана-Больцмана.

\item\label{_9}
Вычислите постоянную Стефана-Больцмана, воспользовавшись формулой Планка.

\item\label{_10}
Квантовомеханический гармонический осциллятор: представление чисел заполнения, энергетический спектр, волновые функции основного и первого возбуждённого состояний.

\item\label{_11}
Операторы проекции момента импульса на выделенное направление $\hat{L}_x, \hat{L}_y, \hat{L}_z$ и оператор квадрата момента импульса, запишите и прокомментируйте коммутационные соотношения.

\item\label{_12}
Составить выражение для величины, имеющей размерность длины, используя скорость света $c$, массу частицы $m$, постоянную Планка $h$. Что это за величина?

\item\label{_13}
Получите и прокомментируйте обобщенное соотношение неопределённостей Хайзенберга величин $A$ и $B$: $\Delta A\Delta B \ge \frac{\left|\langle\hat{A} | \hat{B}\rangle\right|}{2}$.

\item\label{_14}
Постулаты квантовой механики: о квантовых состояниях, о физических величинах, об измерениях, динамический постулат.

\item\label{_15}
Покажите, что оператор $\hat{L}_z = \hat{x}\hat{p}_y-\hat{y}\hat{p}_x$ эрмитов.

\item\label{_16}
Получите и прокомментируйте <<соотношение неопределённостей энергия-время>>.

\item\label{_17}
Квантовомеханическое среднее и его временная эволюция.

\item\label{_18}
Покажите, что два наблюдаемых оператора коммутируют тогда и только тогда, когда имеют общую систему собственных функций. Приведите примеры совместных и несовместных наблюдаемых.

\item\label{_19}
Рассмотрите задачу об одномерном потенциальном барьере бесконечной ширины для случая, когда энергия микрообъекта превышает высоту потенциального порога, найдите коэффициенты отражения и прохождения.

\item\label{_20}
Получите матричные элементы оператора импульса в координатном представлении $\langle x' | \hat{p} | x\rangle$ и оператора координаты в импульсном представлении $\langle p' | \hat{x} | p \rangle$, прокомментируйте результаты.

\item\label{_21}
Классические уравнения Гамильтона и теорема Эренфеста.

\item\label{_22}
Стационарное уравнение Шрёдингера. Рассмотрите задачу о движении электрона в одномерном потенциале, представляющем собой ступеньку бесконечной ширины.

\item\label{_23}
Покажите, что при унитарных преобразованиях векторов состояний квантовомеханические средние не меняются.

\item\label{_24}
Стационарное уравнение Шрёдингера. Электрон в бесконечно глубокой одномерной потенциальной яме.

\item\label{_25}
Ультрафиолетовая катастрофа. Формула Рэлея-Джинса. Формула Планка.

\item\label{_26}
Воспользуйтесь постулатами Бора, чтобы получить комбинационное правило Ритберга-Ритца.

\item\label{_27}
Запишите и прокомментируйте соотношения неопределённостей для $\hat{p}_x$ и $\hat{p}_y$, а также $\hat{L}_x$, $\hat{L}_y$

\item\label{_28}
Обобщённое соотношение неопределённостей Хайзенберга. Примеры использования.

\item\label{_29}
Воспользуйтесь соотношением Вина для спектральной плотности энергии, чтобы получить закон смещения Вина.

\item\label{_30}
Временная эволюция классической величины и временная эволюция квантовомеханического среднего. Интеграл движения в классической и квантовой механике.

\item\label{_31}
Для квантовомеханического гармонического осциллятора, состояние которого задаётся кет-вектором $\left|n\right\rangle$, вычислите $\Delta x\Delta p_x$, прокомментируйте результат.

\item\label{_32}
Спектр операторов $\hat{J}^2$ и $\hat{J}_z$ ($\hat{J}$ -- оператор полного момента импульса).

\item\label{_33}
Покажите ортогональность собственных функций, отвечающих различным собственным значениям эрмитова оператор (считайте, что собственные значения невырождены).

\item\label{_34}
Исходя из обобщённого соотношения неопределённостей Хайзенберга, получите соотношение неопределённостей «время-энергия». Объясните, почему название соотношения не вполне удачно.

\item\label{_35}
Сформулируйте постулат квантовой механики об измерениях. Покажите, что собственные функции эрмитова оператора, отвечающие различным его собственным значениям, ортогональны.

\item\label{_36}
Вычислите коммутаторы $\hat{x}$, $\hat{p}_x$, $\left[\hat{L}_z,\hat{L}^2\right]$. Прокомментируйте результаты.

\item\label{_37}
Получите спектр оператора числа элементарных возбуждений гармонического осциллятора, связь с энергетическим спектром.

\item\label{_38}
Продемонстрируйте сохранение квадрата нормы волновой функции во времени. Вектор плотности тока вероятности. Получите уравнение непрерывности для плотности вероятности.

\item\label{_39}
Тепловое излучение и люминесценция. Законы теплового излучения.

\item\label{_40}
Временное и стационарное уравнения Шрёдингера.

\item\label{_41}
Приведите пример эрмитова оператора, не обладающего полной системой собственных функций.

\item\label{_42}
Сформулируйте теорему Эренфеста; прокомментируйте её на примере электрона в одномерном потенциале.

\item\label{_43}
Рассмотрите задачу об электроне в бесконечно глубокой потенциальной яме. Чему равна минимальная кинетическая энергия электрона? Какова вероятность обнаружить электрон в интервале $\frac{L}{6}\le x\le \frac{L}{3}$ (где $L$ ширина ямы) во втором возбуждённом состоянии?

\item\label{_44}
Принцип дополнительности Бора и соотношение неопределённости Хайзенберга. Постулат квантовой механики об измерениях.

\item\label{_45}
Сформулируйте постулат квантовой механики о физических величинах. Покажите, что все собственные значения эрмитова оператора суть вещественные числа.

\item\label{_46}
Рассмотрите задачу об электроне в трёхмерной бесконечно глубокой потенциальной яме. Какова минимальная кинетическая энергия электрона? Чему равна кратность вырождения энергетического уровня $ 27 \frac{\pi^2\hslash^2}{2m^2}-U_0$ (где $L$ ширина ямы)?

\item\label{_47}
Покажите, что переход от одного представления к другому производится с помощью унитарного преобразования.

\item\label{_48}
Рассмотрите задачу о потенциальном барьере конечной ширины.

\item\label{_49}
Рассмотрите задачу об электроне в бесконечно глубокой потенциальной
яме. Чему равна минимальная кинетическая энергия электрона? Какова вероятность обнаружить электрон в интервале $\frac{L}{6}\le x\le \frac{L}{3}$ (где $L$ ширина ямы) во втором возбуждённом состоянии?

\item\label{_50}
Найдите волновые функции основного и первого возбуждённого состояний квантовомеханического гармонического осциллятора.

\end{enumerate}

\clearpage

%
\section{Таблица соответствия}
%

\begin{multicols}{3}

\begin{tabular}{c|rrr}
\textbf{1}  & \ref{_1} & \ref{_2} & \ref{_3} \\
\textbf{2}  & \ref{_4} & \ref{_5} & \ref{_6} \\
\textbf{3}  & \ref{_7} & \ref{_8} & \ref{_9} \\
\textbf{4}  & \ref{_10} & \ref{_11} & \ref{_12} \\
\textbf{5}  & \ref{_13} & \ref{_14} & \ref{_15} \\
\textbf{6}  & \ref{_16} & \ref{_17} & \ref{_18} \\
\textbf{7}  & \ref{_19} & \ref{_6} & \ref{_20} \\
\textbf{8}  & \ref{_21} & \ref{_22} & \ref{_23} \\
\textbf{9}  & \ref{_24} & \ref{_25} & \ref{_26} \\
\end{tabular}

\begin{tabular}{c|rrr}
\textbf{10} & \ref{_25} & \ref{_10} & \ref{_27} \\
\textbf{11} & \ref{_28} & \ref{_1} & \ref{_29} \\
\textbf{12} & \ref{_8} & \ref{_30} & \ref{_31} \\
\textbf{13} & \ref{_28} & \ref{_32} & \ref{_33} \\
\textbf{14} & \ref{_2} & \ref{_34} & \ref{_35} \\
\textbf{15} & \ref{_30} & \ref{_14} & \ref{_36} \\
\textbf{16} & \ref{_37} & \ref{_6} & \ref{_38} \\
\textbf{17} & \ref{_39} & \ref{_40} & \ref{_41} \\
\textbf{18} & \ref{_14}& \ref{_42} & \ref{_43} \\
\end{tabular}

\begin{tabular}{c|rrr}
\textbf{19} & \ref{_7} & \ref{_44} & \ref{_45} \\
\textbf{20} & \ref{_21} & \ref{_32} & \ref{_46} \\
\textbf{21} & \ref{_24} & \ref{_1} & \ref{_6} \\
\textbf{22} & \ref{_30} & \ref{_32} & \ref{_47} \\
\textbf{23} & \ref{_4} & \ref{_14} & \ref{_27} \\
\textbf{24} & \ref{_48} & \ref{_28} & \ref{_9} \\
\textbf{25} & \ref{_44} & \ref{_25} & \ref{_49} \\
\textbf{26} & \ref{_45} & \ref{_10} & \ref{_50} \\
\textbf{27} & \ref{_6} & \ref{_30} & \ref{_45} \\
\end{tabular}

\end{multicols}


\end{document}