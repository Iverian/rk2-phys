\documentclass[12pt]{article}

\usepackage{fontspec}
\usepackage{polyglossia}
\usepackage[a4paper,
            total={170mm,255mm},
            left=10mm,
            top=15mm]
            {geometry}
\usepackage{graphicx}

\usepackage{amsmath,
            amsthm,
            amssymb
}

\usepackage{unicode-math,
            tensor
}

\usepackage{wrapfig, 
            hyperref,
            multicol,
            multirow,
            tabularx,
            booktabs,
            makecell,
            subfiles
}

\setdefaultlanguage{russian}
\setotherlanguage{english}
\setkeys{russian}{babelshorthands=true}

\defaultfontfeatures{Ligatures=TeX}
\setmainfont{STIX Two Text}
\setmathfont{STIX Two Math}
\DeclareSymbolFont{letters}{\encodingdefault}{\rmdefault}{m}{it}

\newfontfamily{\cyrillicfont}{STIX Two Text} 
\newfontfamily{\cyrillicfontrm}{STIX Two Text}
\newfontfamily{\cyrillicfonttt}{Courier New}
\newfontfamily{\cyrillicfontsf}{STIX Two Text}

\renewcommand{\thefigure}{\thesection.\arabic{figure}}
\renewcommand{\thetable}{\thesection.\arabic{table}}
\numberwithin{equation}{section}

\renewcommand{\qedsymbol}{$\blacksquare$}
\theoremstyle{definition}
\newtheorem{definition}{Опр.}[section]
\theoremstyle{remark}
\newtheorem{statement}{Утв.}[section]
\theoremstyle{plain}
\newtheorem{theorem}{Теор.}[section]

\addto\captionsrussian{
  \renewcommand{\figurename}{Рис.}
  \renewcommand{\tablename}{Табл.}
  \renewcommand{\proofname}{Док-во}
}

\graphicspath{{./img/}}
\everymath{\displaystyle}

\newcommand{\RNumb}[1]{\uppercase\expandafter{\romannumeral#1\relax}}

\newcommand{\llabel}[1]{\label{\thesubsection:#1}}
\newcommand{\lref}[1]{\ref{\thesubsection:#1}}


\begin{document}

\paragraph{3}
Покажите, что коммутационные соотношения наблюдаемых операторов не зависят от выбора представления. Приведите примеры.\\

Рассмотрим два ортонормированных базиса $E=\{\left|j\right\rangle\}$ и $\mathscr{E}=\{\left|\mathscr{j}\right\rangle\}$, один из которых определяет дискретное представление $R$ (индексы принадлежат $\mathbb{N}$ и меняются дискретно), другой $\mathscr{R}$ -- непрерывное (индексы из $\mathbb{R}$, суммирование по индексу заменено на интегрирование по нему же). Свойство ортонормированности запишем так:
\begin{gather*}
    \begin{array}{cc}
        \langle k | j \rangle = \delta_{kj};       & \langle \mathscr{k} | \mathscr{j} \rangle = \delta(\mathscr{k-j});      \\
        \sum_{j}|j\rangle \langle j| = \mathrm{I}; & \int d\mathscr{j} |\mathscr{j}\rangle \langle\mathscr{j}| = \mathrm{I}; \\
    \end{array}
\end{gather*}
где $\mathrm{I}$ -- единичный оператор. Очевидно, что элементы одного базиса могут быть представлены через элементы другого:
\begin{gather*}
    \begin{array}{cc}
        |j\rangle = \int d\mathscr{j} |\mathscr{j}\rangle \langle\mathscr{j}|j\rangle;\qquad &
        |\mathscr{j}\rangle = \sum_{j}|j\rangle \langle j|\mathscr{j}\rangle
    \end{array}
\end{gather*}
Тогда пусть $W=\langle j|\mathscr{j}\rangle$ -- унитарный (оператор, обратный к которому получается взятием комплесного сопряжения от транспонированного оператора, $A=(a_{ij})\colon A^{-1}=(b_{ij})\colon b_{ij}=\bar{a}_{ji}$) оператор перехода $R\rightarrow\mathscr{R}$.\\
Запишем наблюдаемый оператор $\hat{A}$ в двух представлениях:
\begin{gather*}
    \begin{array}{cc}
        A_R = (\langle j'|\hat{A}|j\rangle) & A_\mathscr{R} = (\langle \mathscr{j}'|\hat{A}|\mathscr{j}\rangle)
    \end{array}
\end{gather*}
Через матрицу перехода ($W^{-1}=W^\dagger$ -- эрмитово сопряженный оператор к $W$):
\begin{gather*}
    A_\mathscr{R} = W^\dagger A_R W
\end{gather*}

Запишем коммутатор двух наблюдаемых операторов $\hat{A}$ и $\hat{B}$:
\begin{gather*}
    [\hat{A},\hat{B}] = \hat{A}\hat{B}-\hat{B}\hat{A}
\end{gather*}
Тогда коммутатор в $\mathscr{R}$:
\begin{flalign*}
    &
    [\hat{A},\hat{B}]_\mathscr{R}
    =
    A_\mathscr{R} B_\mathscr{R}-B_\mathscr{R} A_\mathscr{R}
    =
    (W^\dagger A_R W)(W^\dagger B_R W) - (W^\dagger B_R W)(W^\dagger A_R W)
    =\\
    =&
    W^\dagger(A_R B_R - B_R A_R)W
    =\\
    =&
    W^\dagger[\hat{A},\hat{B}]_R W,
\end{flalign*}
получим, что коммутатор преобразуется по линейному закону, т.е. существует оператор $\hat{C}=[\hat{A},\hat{B}]$, матрица которого совпадает с матрицей коммутатора в любом представлении. Осталось лишь показать, что наблюдаемый оператор остается таковым при переходе, т.е. $\hat{A}=\hat{A}^\dagger$ соблюдается в любом представлении:
\begin{gather*}
    A_\mathscr{R}=W^\dagger A_R W
    \Longrightarrow
    A_\mathscr{R}^\dagger = (W^\dagger A_R W)^\dagger = W^\dagger A_R^\dagger (W^\dagger)^\dagger =
    W^\dagger A_R W = A_\mathscr{R}
\end{gather*}

В качестве примера рассмотрим коммутатор $[\hat{x},\hat{p}]$: в координатном представлении:
\begin{gather*}
    [\hat{x},\hat{p}]=i\hbar
\end{gather*}
Вычислим этот коммутатор в импульсном представлении:
\begin{gather*}
    [\hat{x},\hat{p}]\breve{\Psi}(p)
    =
    \left(-\frac{\hbar}{i}\frac{d}{dp}p+\frac{\hbar}{i}p\frac{d}{dp}\right)\breve{\Psi}(p)
    =
    \frac{h}{i}\left(\breve{\Psi}(p)-p\frac{d\breve{\Psi}(p)}{dp}+p\frac{d\breve{\Psi}(p)}{dp}\right)
    =
    i\hbar\breve{\Psi}(p)
\end{gather*}
т.е. при действии на $\breve{\Psi}(p)$ коммутатором, она переходит в $i\hbar\breve{\Psi}(p)$, т.е. сам коммутатор можно записать как $i\hbar$, что совпадает с координатной формой.

\end{document}