\documentclass[12pt]{article}

\usepackage{fontspec}
\usepackage{polyglossia}
\usepackage{geometry}
\usepackage{amsmath,
            amsthm,
            amssymb
}
\usepackage{unicode-math,
            tensor
}
\usepackage{wrapfig, 
            hyperref,
            multicol,
            multirow,
            tabularx,
            booktabs,
            makecell,
            enumitem
}

\geometry{a4paper,
          total={170mm,255mm},
          left=10mm,
          top=15mm,
}

\setdefaultlanguage{russian}
\setotherlanguage{english}
\setkeys{russian}{babelshorthands=true}

\defaultfontfeatures{Ligatures=TeX}
\setmainfont{STIX Two Text}
\setmathfont{STIX Two Math}
\DeclareSymbolFont{letters}{\encodingdefault}{\rmdefault}{m}{it}

\newfontfamily{\cyrillicfont}{STIX Two Text} 
\newfontfamily{\cyrillicfontrm}{STIX Two Text}
\newfontfamily{\cyrillicfonttt}{Courier New}
\newfontfamily{\cyrillicfontsf}{STIX Two Text}

\everymath{\displaystyle}

\newcommand{\eL}[1]{\label{Э-#1}}
\newcommand{\eR}[1]{\ref{Э-#1}}

\newcommand{\oL}[1]{\label{О-#1}}
\newcommand{\oR}[1]{\ref{О-#1}}

\newcommand{\kL}[1]{\label{К-#1}}
\newcommand{\kR}[1]{\ref{К-#1}}

\def\twodigits#1{% 
\ifnum#1<10 0\fi 
\number#1}

\begin{document}

%
\section*{Э Электричество}
%

\begin{enumerate}[label={\textbf{Э-\protect\twodigits{\theenumi}}}]

% 01-1 
\item\eL{1} 
Электрический диполь: электрический дипольный момент, потенциальная энергия диполя в электростатическом поле, момент сил, действующих на диполь в однородном электростатическом поле. Механизмы поляризации диэлектриков.

% 01-3
\item\eL{2}
Рассмотрите электромагнитные волны в пространстве, свободном от зарядов и токов. Покажите, что вектора $\vec{k}$ (волновой вектор), $\vec{E}$, $\vec{B}$ образуют правую тройку в некоторой инерциальной системе отсчёта. Объясните, почему в любой другой инерциальной системе отсчёта этот факт, будучи сформулированным для преобразованных векторов, будет также иметь место.

% 02-1 09-1
\item\eL{3}
Механизмы поляризации диэлектриков. Теорема Гаусса для вектора $\vec{D}$. Условия на границе раздела диэлектриков.

% 03-1
\item\eL{4}
Полевая версия теоремы Нётер. Сохраняющийся нётеровский ток. Канонический и симметризованный тензоры энергии импульса электромагнитного поля. Физический смысл компонент симметризованного тензора энергии-импульса электромагнитного поля.

% 04-1 06-1
\item\eL{5}
Электрический ток. Плотность тока. Закон сохранения электрического заряда. Дифференциальная и интегральная формы закона Ома, границы его области применимости.

% 05-1
\item\eL{6}
4-векторный потенциал и тензор Максвелла. Связь полей $\vec{E}$ и $\vec{B}$ с 4-векторным потенциалом $A^\mu$. Действие для электромагнитного поля.

% 05-3 08-3
\item\eL{7}
Воспользуйтесь полевыми уравнениями Эйлера-Лагранжа чтобы получить уравнения Максвелла <<с источниками>>.

% 07-1
\item\eL{8}
4-векторный потенциал и тензор Максвелла. Связь полей $\vec{E}$ и $\vec{B}$ с 4-векторным потенциалом $A^\mu$. Система уравнений Максвелла-Лоренца.

% 08-1
\item\eL{9}
Закон Кулона, напряжённость поля, силовые линии электростатического поля, электростатическая защита. Работа в электростатическом поле, потенциальность электростатического поля.

% 10-1
\item\eL{10}
Действие для заряженной частицы в электромагнитном поле. 4-сила Лоренца. Релятивистски инвариантное действие для системы <<электромагнитное поле + заряженные частицы>>. Система уравнений Максвелла-Лоренца.

% 12-1
\item\eL{11}
Работа в электростатическом поле, потенциальность электростатического поля. Потенциальная энергия точечного заряда $q$ в поле, создаваемом системой точечных зарядов $Q_j$. Потенциал электростатического поля. Вычисление потенциала по известной напряжённости поля и определение конфигурации поля по заданному потенциалу.

% 13-1
\item\eL{12}
Полевые уравнения Эйлера-Лагранжа. Уравнения Максвелла. Полевая версия теоремы Нётер. Тензор энергии-импульса электромагнитного поля.

% 15-3
\item\eL{13}
Убедитесь в инвариантности тензора Максвелла относительно калибровочных преобразований. Воспользуйтесь уравнениями Максвелла <<с источниками>>, чтобы получить закон сохранения электрического заряда.

% 17-1
\item\eL{14}
Гипотеза молекулярных токов Ампера. Теорема о циркуляции вектора $\vec{H}$. Условия на границе раздела магнетиков.

% 17-3
\item\eL{15}
Вычислите ёмкость сферического конденсатора, представленного двумя концентрическими обкладками, радиусы которых $R_1$ и $R_2$, диэлектрическая проницаемость вещества, заполняющего пространство между ними $\varepsilon$.

% 18-1 26-2
\item\eL{16}
Действие для электромагнитного поля. Преобразования полей при переходе из одной инерциальной системы отсчёта в другую. Релятивистские инварианты электромагнитного поля.

% 20-1
\item\eL{17}
Физический смысл компонент симметризованного тензора энергии-импульса электромагнитного поля. Вектор Пойнтинга и теорема Пойнтинга.

% 21-3
\item\eL{18}
Покажите, что канонический тензор энергии-импульса электромагнитного
поля 
$$
T^{\mu\nu}=F^{\mu\lambda}\partial^\nu A_\lambda-\frac{1}{4}F_{\alpha\beta}F^{\alpha\beta}\eta^{\mu\nu}
$$
не является калибровочно инвариантным в отличие от симметризованного тензора энергии-импульса электромагнитного поля 
$$
\Theta^{\mu\nu} = T^{\mu\nu}-F^{\mu\lambda}\partial_{\lambda}A^{\nu}
$$.

% 23-3
\item\eL{19}
Система уравнений Максвелла-Лоренца, материальные уравнения, условия на границе раздела двух диэлектриков, магнетиков.

% 24-3
\item\eL{20}
Оценить среднюю объёмную плотность электрических зарядов в атмосфере, если известно, что напряженность электрического поля на поверхности Земли составляет примерно $130 \text{В}/\text{м}$, а на высоте $1 \text{км}$ -- примерно $40 \text{В}/\text{м}$.

% 25-1
\item\eL{21}
Сила Ампера. Магнитное поле прямого постоянного тока. Сила взаимодействия двух коллинеарных постоянных токов.

% 27-1
\item\eL{22}
Магнитное поле равномерно медленно движущегося заряда. Закон Био-Савара-Лапласа, границы его области применимости.

% 27-3
\item\eL{23}
Применение теоремы Гаусса к расчету электростатических полей: найдите поле, порождаемое бесконечной равномерно заряженной (поверхностная плотность заряда $\sigma$) плоскостью, бесконечной равномерно заряженной нитью (линейная плотность заряда $\kappa$).

\end{enumerate}

%
\section*{О Оптика}
%

\begin{enumerate}[label={\textbf{О-\protect\twodigits{\theenumi}}}]

% 11-1 24-1
\item\oL{1}
Дифракция света. Принцип Гюйгенса-Френеля. Метод зон Френеля. Классификация дифракционных явлений.

% 14-1 19-1
\item\oL{2}
Временная и пространственная когерентность электромагнитных волн. Длина и радиус когерентности. Связь временной когерентности со степенью монохроматичности.

% 15-1
\item\oL{3}
Принцип Гюйгенса-Френеля. Дифракционный интеграл. Пятно Пуассона-Араго. Дифракционная решётка, её характеристики.

% 16-1 26-1
\item\oL{4}
Электромагнитные волны в пространстве, свободном от зарядов и токов. Интерференция света: определение, общая схема и условия наблюдения интерференции. Интерференция в тонких пленках.

% 19-3
\item\oL{5}
Найти угловое распределение дифракционных минимумов при дифракции на решетке, период которой равен $d$, а ширина щели равна $b$.

% 21-1
\item\oL{6}
Цуг и монохроматическая волна. Длина волны, волновой вектор, волновое число. Волновая поверхность, фронт волны, фазовая скорость. Временная и пространственная когерентность электромагнитных волн.

% 22-1
\item\oL{7}
Диполь Герца. Электромагнитные волны. Поперечность электромагнитных волн.

\end{enumerate}

%
\section*{К Квантовая механика}
%

\begin{enumerate}[label={\textbf{К-\protect\twodigits{\theenumi}}}]

% 01-2
\item\kL{1} 
Тепловое излучение и люминесценция. Равновесное тепловое излучение: свойства, спектральная плотность энергии, температура.

% 02-2
\item\kL{2}
Энергетический спектр квантовомеханического гармонического осциллятора.

% 02-3
\item\kL{3}
Докажите, что два наблюдаемых оператора коммутируют тогда и только тогда, когда обладают общей системой собственных функций. Раскройте физическое содержание этого утверждения.

% 03-2
\item\kL{4}
Абсолютно чёрное тело: испускательная способность, энергетическая светимость. Закон Стефана-Больцмана.

% 03-3
\item\kL{5}
Вычислите постоянную Стефана-Больцмана, воспользовавшись формулой Планка.

% 04-2
\item\kL{6}
Квантовомеханический гармонический осциллятор: представление чисел заполнения, энергетический спектр, волновые функции основного и первого возбуждённого состояний.

% 04-3 20-3
\item\kL{7}
Рассмотрите задачу об электроне в трёхмерной бесконечно глубокой потенциальной яме. Какова минимальная кинетическая энергия электрона? Чему равна кратность вырождения энергетического уровня $27\frac{\pi^2\hbar^2}{2mL^2}+U_0$ (где $L$ ширина ямы)?

% 05-2
\item\kL{8}
Получите и прокомментируйте обобщенное соотношение неопределённостей Хайзенберга величин $A$ и $B$: $\Delta A\Delta B \ge \frac{\left|\left<\hat{A},\hat{B}\right>\right|}{2}$

% 06-2
\item\kL{9}
Квантовомеханическое среднее, его временная эволюция.

% 06-3
\item\kL{10}
Покажите, что два наблюдаемых оператора коммутируют тогда и только тогда, когда имеют общую систему собственных функций. Приведите примеры совместных и несовместных наблюдаемых.

% 07-2
\item\kL{11}
Эксперимент Штерна-Герлаха. Гипотеза спина электрона. Оператор проекции спина на выделенное направление. Эксперименты с поляризованным пучком электронов.

% 07-3 22-3
\item\kL{12}
Получите матричные элементы оператора импульса в координатном представлении $\left<x'|\hat{p}|x\right>$ и оператора координаты в импульсном представлении $\left<p'|\hat{x}|p\right>$, прокомментируйте результаты.

% 08-2
\item\kL{13}
Стационарное уравнение Шрёдингера. Рассмотрите задачу о движении электрона в одномерном потенциале, представляющем собой ступеньку бесконечной ширины.

% 09-2 24-2
\item\kL{14}
Принцип неразличимости частиц одного сорта: математическая формулировка, следствия.

% 09-3
\item\kL{15}
Рассмотрите задачу об одномерном потенциальном барьере бесконечной ширины для случая, когда энергия микрообъекта превышает высоту потенциального порога, найдите коэффициенты отражения и прохождения.

% 10-2
\item\kL{16}
Квантовомеханический гармонический осциллятор: представление чисел заполнения, энергетический спектр, волновые функции основного и первого возбуждённого состояний.

% 10-3 14-3
\item\kL{17}
Определите характер зависимости от температуры электрической восприимчивости диэлектрика, состоящего из полярных молекул.

% 11-2 19-2
\item\kL{18}
Электрон в сферически симметричном потенциале. Главное, орбитальное, магнитное квантовые числа. Опыт Штерна-Герлаха. Гипотеза спина электрона.

% 11-3
\item\kL{19}
Воспользуйтесь соотношением Вина для спектральной плотности энергии, чтобы получить закон смещения Вина.

% 12-2 27-2
\item\kL{20}
Временная эволюция классической величины и временная эволюция квантовомеханического среднего. Интеграл движения в классической и квантовой механике.

% 12-3
\item\kL{21}
Для квантовомеханического гармонического осциллятора, состояние которого задаётся кет-вектором $\left|n\right>$, вычислите $\Delta x\Delta p$, прокомментируйте результат.

% 13-2 20-2 22-2
\item\kL{22}
 Спектр операторов $\hat{J}^2$ и $\hat{J}_z$ ($\hat{J}$ -- оператор полного момента импульса).

% 13-3 25-3
\item\kL{23}
Рассмотрите задачу об электроне в бесконечно глубокой потенциальной яме. Чему равна минимальная кинетическая энергия электрона? Какова вероятность обнаружить электрон в интервале $\frac{L}{6}\le x\le\frac{L}{3}$ (где $L$ ширина ямы) во втором возбуждённом состоянии?

% 14-2
\item\kL{24}
Магнитный момент атома, связанный с орбитальным моментом импульса электрона: квазиклассическое и квантовомеханическое рассмотрения.

% 15-2 23-2
\item\kL{25}
Постулаты квантовой механики: о квантовых состояниях, о физических величинах, об измерениях, динамический постулат.

% 16-2 18-2 
\item\kL{26}
Сформулируйте теорему Эренфеста; прокомментируйте её на примере электрона в одномерном потенциале.

% 16-3
\item\kL{27}
Продемонстрируйте сохранение квадрата нормы волновой функции во времени. Вектор плотности тока вероятности. Получите уравнение непрерывности для плотности вероятности.

% 17-2
\item\kL{28}
Законы теплового излучения. Фотоэффект. Эффект Комптона.

% 18-3
\item\kL{29}
Сформулируйте постулат квантовой механики о физических величинах. Покажите, что все собственные значения эрмитова оператора суть вещественные числа.

% 21-2
\item\kL{30}
<<Старая>> квантовая теория: постулаты Бора и комбинационное правило Ритберга-Ритца.

% 23-3
\item\kL{31}
Запишите и прокомментируйте соотношения неопределённостей для $\hat{x}$ и $\hat{p}$. Убедитесь в эрмитовости операторов $\hat{p}_x$ и $\hat{L}_x$.

% 25-2
\item\kL{32}
Ультрафиолетовая катастрофа. Формула Рэлея-Джинса. Формула Планка.

% 26-3
\item\kL{33}
Найдите волновые функции основного и первого возбуждённого состояний квантовомеханического гармонического осциллятора.


\end{enumerate}

\clearpage

%
\section*{Таблица соответствия}
%

\begin{multicols}{3}

\begin{tabular}{c|rrr}
\textbf{1}  & \eR{1} & \kR{1} & \eR{2} \\
\textbf{2}  & \eR{3} & \kR{2} & \kR{3} \\
\textbf{3}  & \eR{4} & \kR{4} & \kR{5} \\
\textbf{4}  & \eR{5} & \kR{6} & \kR{7} \\
\textbf{5}  & \eR{6} & \kR{8} & \eR{7} \\
\textbf{6}  & \eR{5} & \kR{9} & \kR{10} \\
\textbf{7}  & \eR{8} & \kR{11} & \kR{12} \\
\textbf{8}  & \eR{9} & \kR{13} & \eR{7} \\
\textbf{9}  & \eR{1} & \kR{14} & \kR{15} \\
\end{tabular}

\begin{tabular}{c|rrr}
\textbf{10} & \eR{10} & \kR{16} & \kR{17} \\
\textbf{11} & \oR{1} & \kR{18} & \kR{19} \\
\textbf{12} & \eR{11} & \kR{20} & \kR{21} \\
\textbf{13} & \eR{12} & \kR{22} & \kR{23} \\
\textbf{14} & \oR{2} & \kR{24} & \kR{17} \\
\textbf{15} & \oR{3} & \kR{25} & \eR{13} \\
\textbf{16} & \oR{4} & \kR{26} & \kR{27} \\
\textbf{17} & \eR{14} & \kR{28} & \eR{15} \\
\textbf{18} & \eR{17} & \kR{26} & \kR{29} \\
\end{tabular}

\begin{tabular}{c|rrr}
\textbf{19} & \oR{2} & \kR{18} & \oR{5} \\
\textbf{20} & \eR{17} & \kR{22} & \kR{7} \\
\textbf{21} & \oR{6} & \kR{30} & \eR{18} \\
\textbf{22} & \oR{7} & \kR{22} & \kR{12} \\
\textbf{23} & \eR{19} & \kR{25} & \kR{31} \\
\textbf{24} & \oR{1} & \kR{14} & \eR{20} \\
\textbf{25} & \eR{21} & \kR{32} & \kR{23} \\
\textbf{26} & \oR{4} & \eR{16} & \kR{33} \\
\textbf{27} & \eR{22} & \kR{20} & \eR{23} \\
\end{tabular}

\end{multicols}

\end{document}