\documentclass[__main__.tex]{subfiles}

\begin{document}

\paragraph{Э-04}
Полевая версия теоремы Нётер. Сохраняющийся нётеровский ток. Канонический и симметризованный тензоры энергии импульса электромагнитного поля. Физический смысл компонент симметризованного тензора энергии-импульса электромагнитного поля.\\

\begin{theorem}[Нётер]
    Инвариантность действия относительно некоторой непрерывной группы симметрии приводит к соответствующему закону сохранения.
\end{theorem}

Пусть задано действие
\begin{gather}
    \mathcal{A}=\int\mathcal{L}\left( \mathcal{A}^\alpha (x),\partial_\beta \mathcal{A}^\alpha(x)\right)dx^4,
\end{gather}
тогда для симметрии $x\prime=x+\delta x$ выражения нетеровского тока $\mathcal{J}^\mu$:
\begin{gather}
\mathcal{J}^\mu
=
\left(\mathcal{L}(x)\delta_\nu^\mu-\frac{\partial \mathcal{L}}{\partial (\partial_\mu\mathcal{A}(x))}\partial_\nu\mathcal{A}(x)\right)\delta x^\nu+\frac{\partial\mathcal{L}}{\partial(\partial_\mu\mathcal{A}(x))}\delta\mathcal{A}(x)
\end{gather}

Введем обозначение:
\begin{gather}
    \mathcal{J}^\mu=-\mathcal{F}\indices{^\mu_\nu}\epsilon^\nu,
\end{gather}
где канонический тензор энергии-импульса $\mathcal{F}\indices{^\mu_\nu}$:
\begin{gather}
    \mathcal{F}\indices{^\mu_\nu}\equiv\frac{\partial \mathcal{L}}{\partial (\partial_\mu \mathcal{A})}\partial_\nu\mathcal{A}-\mathcal{L}\delta\indices{^\mu_\nu}.
\end{gather}
Выражения канонического тензора энергии-импульса через тензор электромагнитного поля $F$:
\begin{gather*}
    \underset{emf}{T}^{\mu\nu} = F^{\mu\lambda}\partial^\nu A_\lambda - \frac{1}{4}F_{\alpha\beta}F^{\alpha\beta}\eta^{\mu\nu},
\end{gather*}
а симметризованный тензор энергии-импульса $\underset{emf}{\Theta}^{\mu\nu}$ выражается так:
\begin{gather*}
    \underset{emf}{\Theta}^{\mu\nu}
    =
    F^{\mu\lambda}F\indices{^\nu_\lambda}
    -
    \frac{1}{4}F_{\alpha\beta}F^{\alpha\beta}\eta^{\mu\nu},
\end{gather*}
в матричном виде имеет вид:
\begin{gather*}
\underset{emf}{\Theta^{\mu\nu}} =
\left[
\begin{array}{cccc}
W & S_x & S_y & S_z \\
S_x & \sigma_{xx} & \sigma_{xy} & \sigma_{xz} \\
S_y & \sigma_{yx} & \sigma_{yy} & \sigma_{yz} \\
S_z & \sigma_{zx} & \sigma_{zy} & \sigma_{zz} \\
\end{array}
\right],
\end{gather*}
где $\vec{S}=\{S_x,S_y,S_z\}=\vec{E}\times\vec{B}$ -- вектор Пойнтинга, $W$ -- плотность энергии, $\sigma_{ij}$ -- тензор напряжений.

\end{document}
