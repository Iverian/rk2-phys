\documentclass[__main__.tex]{subfiles}

\begin{document}

\paragraph{Э-04}
Полевая версия теоремы Нётер. Сохраняющийся нётеровский ток. Канонический и симметризованный тензоры энергии импульса электромагнитного поля. Физический смысл компонент симметризованного тензора энергии-импульса электромагнитного поля.\\

\begin{theorem}[Нётер]
    Инвариантность действия относительно некоторой непрерывной группы симметрии приводит к соответствующему закону сохранения.
\end{theorem}

Получим выражение для нётеровского тока.\\
Пусть задано действие
$$
    \mathcal{A}=\int\mathcal{L}\left( \mathcal{A}^\alpha (x),\partial_\beta \mathcal{A}^\alpha(x)\right)dx^4,\\
$$
где $dx^4=dt\cdot dx^3,\ \partial_\beta=\frac{\partial}{\partial x^\beta}.$ Тогда, рассмотрев малые вариации $x\prime=x+\delta x, \mathcal{A}(x)\rightarrow\mathcal{A}\prime(x)$ получим вaриацию функционала действия:
\begin{flalign*}
    0=&
    \delta \mathcal{A}\equiv
    \int_{\Omega\prime}\prime\mathcal{L}\left(\mathcal{A}^{\alpha\prime}(x\prime),\partial_\beta\prime \mathcal{A}^{\alpha\prime}(x\prime)\right)d(x')^4-
    \int_\Omega\mathcal{L}\left( \mathcal{A}^\alpha (x),\partial_\beta \mathcal{A}^\alpha(x)\right)dx^4
    =\\
    =&
    \int_{\Omega\prime}\prime\left(\mathcal{L}(x)+\delta\mathcal{L}(x)\right)dx^4-
    \int_\Omega\mathcal{L}\left(x\right)dx^4=
    \int_\Omega\mathcal{L}(x)\partial_\mu\delta x^\mu dx^4+
    \int_\Omega\delta\mathcal{L}(x)dx^4
    =\\
    =&
    \int_\Omega\partial_\mu\left(\mathcal{L}(x)\delta x^\mu\right)dx^4+
    \int_\Omega\delta\mathcal{L}(x)dx^4=
    \left[\delta\mathcal{L}(x)=\tilde{\delta}\mathcal{L}(x)+\partial_\mu\left(\mathcal{L}(x)\delta x^\mu\right)\right]
    =\\
    =&
    \int_\Omega\tilde{\delta}\mathcal{L}(x)+2\int_\Omega\partial_\mu\left(\mathcal{L}(x)\delta x^\mu\right)
\end{flalign*}

Посмотрим на первое подынтегральное слагаемое:
\begin{flalign*}
    &
    \tilde{\delta}\mathcal{L}(x)=
    \frac{\partial\mathcal{L}}{\partial\mathcal{A}}\tilde{\delta}\mathcal{A}(x)+
    \frac{\partial\mathcal{L}}{\partial(\partial_\mu\mathcal{A})}\tilde{\delta}(\partial_\mu\mathcal{A}(x))
    =\\
    =&
    \frac{\partial \mathcal{L}}{\partial \mathcal{A}}\tilde{\delta}\mathcal{A}(x)+
    \partial_\mu\left(\frac{\partial \mathcal{L}}{\partial (\partial_\mu\mathcal{A})}\tilde{\delta}\mathcal{A}(x)\right)-
    \partial_\mu\left(\frac{\partial \mathcal{L}}{\partial (\partial_\mu\mathcal{A})}\right)\tilde{\delta}\mathcal{A}(x)
    =\\
    =&
    0+ \partial_\mu\left(\frac{\partial \mathcal{L}}{\partial (\partial_\mu\mathcal{A})}\tilde{\delta}\mathcal{A}(x)\right)
    \Rightarrow\\
    \Rightarrow&
    \int_\Omega\partial_\mu\left(\mathcal{L}(x)\delta x^\mu+\frac{\partial \mathcal{L}}{\partial (\partial_\mu\mathcal{A})}\tilde{\delta}\mathcal{A}(x)\right)dx^4
    =
    \left[\tilde{\delta}\mathcal{A}(x)=\delta\mathcal{A}(x)-\partial_\nu\mathcal{A}\delta x^\nu\right]
    =\\
    =&
    \int_\Omega d^4x\partial_\mu\mathcal{J}^\mu
    \Rightarrow
    \partial_\mu\mathcal{J}^\mu=0,
\end{flalign*}
где $\mathcal{J}^\mu=\left(\mathcal{L}(x)\delta_\nu^\mu-\frac{\partial \mathcal{L}}{\partial (\partial_\mu\mathcal{A}(x))}\partial_\nu\mathcal{A}(x)\right)\delta x^\nu+\frac{\partial\mathcal{L}}{\partial(\partial_\mu\mathcal{A}(x))}\delta\mathcal{A}(x)$ - нётеровский ток\\

Из его же лекций, в которых ничего нет и ничего не понятно следует, что
$$
    \mathcal{J}^\mu=-\mathcal{F}_\nu^\mu\epsilon^\nu,
$$
где канонический тензор энергии-импульса
$$
    \mathcal{F}_\nu^\mu\equiv\frac{\partial \mathcal{L}}{\partial (\partial_\mu \mathcal{A})}\partial_\nu\mathcal{A}-\mathcal{L}\delta_\nu^\mu.
$$

Канонический тензор энергии-импульса имеет вид:
\begin{gather*}
    \underset{emf}{T}^{\mu\nu} = F^{\mu\lambda}\partial^\nu A_\lambda - \frac{1}{4}F_{\alpha\beta}F^{\alpha\beta}\eta^{\mu\nu}
\end{gather*}
\begin{gather*}
    \underset{emf}{T^{\mu\nu}}
    -
    F^{\mu\lambda}\partial_\lambda A^\nu
    =
    F^{\mu\lambda}(\partial^\nu A_\lambda
    -
    \partial_\lambda A^\nu)
    -
    \frac{1}{4}F_{\alpha\beta}F^{\alpha\beta}\eta^{\mu\nu}
    =
    F^{\mu\lambda}F\indices{^\nu_\lambda}
    -
    \frac{1}{4}F_{\alpha\beta}F^{\alpha\beta}\eta^{\mu\nu}
    \equiv
    \underset{emf}{\Theta}^{\mu\nu}
\end{gather*}
$\underset{emf}{\Theta}^{\mu\nu}$ - симметризованый тензор энергии-импульса электромагнитного поля. В матричном виде имеет вид:
\begin{gather*}
\underset{emf}{\Theta^{\mu\nu}} =
\left[
\begin{array}{cccc}
W & S_x & S_y & S_z \\
S_x & \sigma_{xx} & \sigma_{xy} & \sigma_{xz} \\
S_y & \sigma_{yx} & \sigma_{yy} & \sigma_{yz} \\
S_z & \sigma_{zx} & \sigma_{zy} & \sigma_{zz} \\
\end{array}
\right],
\end{gather*}
где $\vec{S}=\{S_x,S_y,S_z\}=\vec{E}\times\vec{B}$ -- вектор Пойнтинга, $W$ -- плотность энергии, $\sigma_{ij}$ -- тензор напряжений.

\end{document}
