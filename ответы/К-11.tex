\documentclass[__main__.tex]{subfiles}

\begin{document}

\paragraph{К-11} Эксперимент Штерна-Герлаха. Гипотеза спина электрона. Оператор проекции спина на выделенное направление. Эксперименты с поляризованным пучком электронов.\\

\textbf{Эксперимент Штерна-Герлаха.} Узкий пучок атомов пропускать через неоднородное магнитное поле, то на магнитные моменты в предположении, что направление поля (по оси я) совпадает с его градиентом, будет действовать сила $F_z = p_{m_z}\frac{\partial B}{\partial z}$. Направление оси z выбирается перпендикулярным пучку атомов, и атомы отклоняются от первоначального направления движения. Если магнитный момент может принимать любые значения, то на экране должна наблюдаться одна широкая непрерывная полоса. Однако в опыте наблюдалось расщепление пучка с образованием на экране узких линий, подтверждающее пространственное квантование. \\

Неожиданный результат появлялся получался в том случае, когда атомы заведомо находились в основном состоянии, в котором магнитный момент равен нулю, а следовательно, расщепления пучка быть не должно. Однако пучок расщеплялся на два пучка, как если бы проекции магнитного момента на направление поля были равны $\pm \mu_{Б}$. \\

\textbf{Гипотеза спина} Электрон сам по себе (независимо от орбитального движения) обладает собственным моментом импульса, называемый спином. \\

\textbf{Оператор проекции спина на выделенное направление} Пусть $\vec{n}$ - единичный вектор, указывающий направление. Тогда оператор $\hat{S}\vec{n} = \hat{S_x} n_x + \hat{S_y} n_y + \hat{S_z} n_z$ называется оператором проекции спина на направление $\vec{n}$\\

\textbf{Эксперименты с поляризованным пучком электронов} \\ Прибор Штерна-Герлаха ориентируется так, чтобы градиент магнитного поля был направлен оси по Z. Затем через магнитное поле пропускается пучок электронов. В месте, откуда выходят электроны в состоянии спин-вниз. В результате остается пучок электронов с положительной ориентации по z-проекции спина (т.е получаем поляризованный пучок). Затем на пути этого пучка ставим второй прибор Ш-Г так, чтобы направление неоднородности магн. поля z в нем образует угол $\nu$. 

Получаем, что если $\vec{n}$ сонаправлен оси Oz, то вероятность того, что все электроны выйдут из второго прибора в состоянии спин-вверх равна 1. Если $\vec{n}$ направлен вниз (против оси Oz), то вероятность найти на выходе из второго прибора электроны в состоянии спин-вниз - нулевая. Если $\nu = \frac{\pi}{2}$, то пучок, входящий в прибор №2 расщепляется на два пучка, и мы получаем, что вероятность найти электрон в состоянии "спин-вниз" (или "спин-вверх") равна $\frac{1}{2}$.

\end{document}