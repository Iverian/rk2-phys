\documentclass[__main__.tex]{subfiles}

\begin{document}
\paragraph{К-13}
Стационарное уравнение Шредингера. Рассмотрите задачу о движении электрона в одномерном потенциале, представляющем собой ступеньку бесконечной ширины.

\begin{definition}
	Стционарное уравнение Шредингера: 
	\begin{gather}
	\llabel{shred}
	\hat{H}\psi(\vec{r})=\varepsilon\psi(\vec{r}), 
	\end{gather}
	где $\hat{H}=\frac{\hat{p}}{2m}+U(t,\vec{r})$ - гамильтониан или оператор полной энергии, $\psi(\vec{r})$ - волновая функция, $\varepsilon$ - некая константа,  отвечающая собственным значениям оператора полной энергии.
\end{definition}

Рассмотрим потенциальный порог бесконечной ширины.


$$
v(x)=
\left\{
\begin{gathered}
0\hfill ,x<0 (I - ground)\\
V_{0}\hfill, x\ge 0 (II - ground)\\
\end{gathered}
\right.
$$

В классической механике: если $\varepsilon<V_{0}$, (U- потенциальная энергия частицы) то движение возможно только при x<0, если $\varepsilon>V_{0}$, то движение возможно всюду, но $T_{II}<T_{I}$, T- энергия.

\textbf{1. Рассмотрим  $\varepsilon>V_{0}$ }

1.1 Область I (x<0)

\begin{gather}
\llabel{movi1}
\frac{\hbar^{2}}{2m}\frac{d^{2}\psi}{dx^{2}}+\varepsilon\psi=0
\end{gather}

\begin{gather}
\llabel{movi2}
\frac{d^{2}\psi}{dx^{2}}+k_{1}^{2}\psi=0,
\end{gather}
где $k_{1}=\frac{\sqrt{2m\varepsilon}}{\hbar}$.

Тогда волновая функция движения электрона в первой области:

\begin{gather}
\llabel{movi3}
\psi_{I}=A_{1}e^{ik_{1}x}+B_{1}e^{-ik_{1}x}.
\end{gather}

1.2 Область II(x>0)


\begin{gather}
\llabel{movi5}
\frac{\hbar^{2}}{2m}\frac{d^{2}\psi}{dx^{2}}+(\varepsilon-V_{0})\psi=0
\end{gather}

\begin{gather}
\llabel{movi6}
\frac{d^{2}\psi}{dx^{2}}+k_{2}^{2}\psi=0,
\end{gather}
где $k_{2}=\frac{\sqrt{2m(\varepsilon-V_{0})}}{\hbar}$.

Тогда волновая функция движения электрона во второй  области:

\begin{gather}
\llabel{movi7}
\psi_{II}=A_{21}e^{ik_{2}x}+B_{2}e^{-ik_{2}x}.
\end{gather}

Если электрон движется слева направо, а источника электрона на $+\infty$ нет, то $B_{1}=0, B_{2}=0$. Если справа налево, то  $A_{1}=0, A_{2}=0$

\textbf{1. Рассмотрим  $\varepsilon<V_{0}$ }
1.1. Область I (x<0)

Также как и в пункте 1.


1.2 Область II(x>0)

Тоже самое , только $(\varepsilon-V_{0})<0$ и значит  $k_{2}=\frac{\sqrt{2m(V_{0}-\varepsilon)}}{\hbar}$.
\end{document}