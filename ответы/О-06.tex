\documentclass[12pt]{article}

\usepackage[a4paper,
            total={170mm,255mm},
            left=10mm,
            top=15mm]
            {geometry}

\usepackage{fontspec,
            polyglossia,
            graphicx}

\usepackage{amsmath,
            amsthm,
            amssymb}

\usepackage{unicode-math,
            tensor}

\usepackage{wrapfig,
            hyperref,
            multicol,
            multirow,
            tabularx,
            booktabs,
            subfiles}

\setdefaultlanguage{russian}
\setotherlanguage{english}
\setkeys{russian}{babelshorthands=true}

\defaultfontfeatures{Ligatures=TeX}
\setmainfont{STIX Two Text}
\setmathfont{STIX Two Math}
\DeclareSymbolFont{letters}{\encodingdefault}{\rmdefault}{m}{it}

\newfontfamily{\cyrillicfont}{STIX Two Text} 
\newfontfamily{\cyrillicfontrm}{STIX Two Text}
\newfontfamily{\cyrillicfonttt}{Courier New}
\newfontfamily{\cyrillicfontsf}{STIX Two Text}

\renewcommand{\thefigure}{\thesection.\arabic{figure}}
\renewcommand{\thetable}{\thesection.\arabic{table}}
\numberwithin{equation}{section}

\renewcommand{\qedsymbol}{$\blacksquare$}
\theoremstyle{definition}
\newtheorem{definition}{Опр.}[section]
\theoremstyle{remark}
\newtheorem{statement}{Утв.}[section]
\theoremstyle{plain}
\newtheorem{theorem}{Теор.}[section]

\graphicspath{{./img/}}
\everymath{\displaystyle}

\newcommand{\llabel}[1]{\label{\thesubsection:#1}}
\newcommand{\lref}[1]{\ref{\thesubsection:#1}}


\begin{document}
\paragraph{o-06}
Цуг и монохроматическая волна. Длина волны, волновой вектор, волновое число. Волновая поверхность, фронт волны, фазовая скорость. Кинематические уравнения плоской и сферической монохроматических волн.

\begin{definition}
Цуг волн - определенная совокупность волн, обладающих разными частотами, которые описывают обладающую волновыми свойствами формацию, в общем случае ограниченную во времени и пространстве.
\end{definition}
\begin{definition}
Монохроматическая волна - модель в физике, удобная для теоретического описания явления волновой природы, означающая, что в спектр волны в ходит всего одна составляющая по частоте 
\end{definition}
\begin{definition}
Длина волны - расстояние между двумя ближайшими друг к другу точками в пространстве, в которых колебания происходит в одинаковой фазе
\end{definition}
\begin{definition}
Волновой вектор - вектор $\vec k$, определяющий направление распространения пронстранственного периода плоской монохроматической волны.
\end{definition}
\begin{definition}
Волновое число - модуль волнового вектора $k = \frac{2\pi}{\lambda}$
\end{definition}
\begin{definition}
Волновая поверхность - геометрическое место точек, испытывающих возмущение обобщенной координаты в одинаковой фазе
\end{definition}
\begin{definition}
Фронт волны - волновая поверхность, отделяющая часть пространства, в которой колебания происходят, от той части, где еще нет колебаний
\end{definition}
\begin{definition}
Фазовая скорость - скорость перемещения точки, обладающей постоянной фазой колебательного движения в пространстве, вдоль заданного направления ($v = \frac{dx}{dt} = \frac{\omega}{k}$)
\end{definition}
Уравнение плоской монохроматической волны:
$$s(x, t) = a\cos(\omega t - kx - \varphi_0)$$
Уравнение сферической монохроматической волны:
$$s(\vec r, t) = \frac{a}{r}\cos(\omega t - kr - \varphi_0)$$

\end{document} 
