\documentclass[__main__.tex]{subfiles}


\begin{document}
	
	\paragraph{K-32}Ультрафиолетовая катастрофа. Формула Рэлея-Джинса. Формула Планка.\\
	
	\begin{definition}
		Закон Рэлея — Джинса — закон излучения для спектральной плотности мощности излучения (спектральной плотности энергетической светимости) $u(\omega,T)$   и для испускательной способности $f(\omega,T)$ абсолютно чёрного тела, который получили Рэлей и Джинс в рамках классической статистики (теорема о равнораспределении энергии по степеням свободы и представление об электромагнитном поле как о бесконечномерной динамической системе).	
	\end{definition}
	Вывод основывается на законе о равнораспределении энергии по степеням свободы: на каждое электромагнитное колебание приходится в среднем энергия, складываемая из двух частей $kT$ . Одну половинку вносит электрическая составляющая волны, а вторую — магнитная. Само по себе, равновесное излучение в полости, можно представить как систему стоячих волн. Количество стоячих волн в трехмерном пространстве дается выражением:
	\begin{flalign}
		dn_\omega = \frac{\omega^2 \cdot  d\omega}{2\cdot \pi^2 \cdot  v^3}
	\end{flalign}
	
	В нашем случае скорость $v$ следует положить равной $c$, более того, в одном направлении могут двигаться две электромагнитные волны с одной частотой, но со взаимно перпендикулярными поляризациями, тогда (0.1) вдобавок следует помножить на два
	\begin{flalign}
		\llabel{_25:dn}
		dn_\omega = \frac{\omega^2 \cdot  d\omega}{\pi^2 \cdot  c^3}
	\end{flalign}
	Рэлей и Джинс каждому колебанию приписали энергию $\overline{\varepsilon}= k\cdot T$  . Помножив (\lref{_25:dn}) на $ \overline{\varepsilon}$, получим плотность энергии, которая приходится на интервал частот $d\omega$ :
	\begin{flalign}
		\llabel{_25:0}
		u(\omega,T)d\omega = \overline{\varepsilon}dn_\omega = kT\frac{\omega^2}{\pi^2\cdot c^3}d\omega
	\end{flalign}
	тогда:
	\begin{flalign}
		\llabel{_25:1}
		u(\omega,T)=kT\cdot \frac{\omega^2}{\pi^2\cdot c^3}
	\end{flalign}
	Зная связь испускательной способности абсолютно чёрного тела $f (\omega,T)$ с равновесной плотностью энергии теплового излучения $f(\omega , T ) =\frac{c}{4} u(\omega, T )$, для $f ( \omega , T )$ находим
	\begin{flalign}
		\llabel{_25:2}
		f(\omega,T)=kT\frac{\omega^2}{4\cdot \pi^2\cdot c^2}
	\end{flalign}
	(\lref{_25:1}) и (\lref{_25:2}) называют формулой Рэлея - Джинса\\
	\textbf{Ультрафиолетовая катастрофа}\\
	Формулы (\lref{_25:1}) и (\lref{_25:2}) удовлетворительно согласуются с экспериментальными данными лишь для больших длин волн, на более коротких волнах согласие с экспериментом резко расходится. Более того, интегрирование (\lref{_25:0}) по $\omega$ в пределах от 0 до $\infty$ для равновесной плотности энергии $u (T)$  дает бесконечно большое значение. Этот результат, получивший название ультрафиолетовой катастрофы, очевидно, входит в противоречие с экспериментом: равновесие между излучением и излучающим телом должно устанавливаться при конечных значениях $u ( T )$ . Логично предположение, что несогласие с экспериментом вызвано некими закономерностями, которые несовместимы с классической физикой. Эти закономерности были определены Максом Планком\\
	\textbf{Формула Планка}\\
	\begin{definition}
		Формула Планка — выражение для спектральной плотности мощности излучения (спектральной плотности энергетической светимости) абсолютно чёрного тела, которое было получено Максом Планком для плотности энергии излучения $u(\omega,T)$ 
	\end{definition}
	\begin{flalign}
		u(\omega,T)=\frac{\omega^2}{\pi^2 \cdot  c^3}\cdot \frac{\hbar\omega}{e^\frac{\hbar\omega}{kT}-1}
	\end{flalign}
\end{document}