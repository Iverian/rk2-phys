\documentclass[__main__.tex]{subfiles}

\begin{document}
\paragraph{Э-05}
Электрический ток. Плотность тока.Закон сохранения электрического заряда. Дифференциальная и интегральная формы закона Ома, границы его области применимости.\\

\begin{definition}
	Электрический ток — это упорядоченное движение заряженных частиц в проводнике. 
\end{definition}

\begin{definition}
	Вектор плотности тока $\vec{j}$ - это физическая величина, харрактеризующая электрический ток, численно равная силе тока $di$, через расположенную в данной точке перпендикулярную к направлению движения носителей площадку $dS$, отнесенной к величине этой площадке:
	$$\vec{j}=\frac{di}{dS}$$
\end{definition}
Ландау говорит, что эта формула лучше:
$$
\vec{j}=σ\cdot\vec{E},
$$
$σ$- коэффициент электропроводности\ проводимость тела

\textbf{4-вектор плотности тока}\\
\begin{gather*}
	\rho(t, \vec x) = \sum_a q_a \delta(\vec x - \vec x_a(t))\\
	\vec j(t, \vec x) = \sum_a q_a \vec v_a \delta(\vec x - \vec x_a(t)),\\
\end{gather*}
где $a$ - индекс частицы.
\begin{statement}
	$j^\mu_{(x)}\equiv \left(\rho(t, \vec x), \vec j(t, \vec x) \right)$ есть 4-вектор.
\end{statement}
\begin{proof}
	\begin{flalign}
	\begin{split}
		&
		j^\mu_{(x)}
		=
		\sum_a q_a \frac{dx_a^\mu}{dt} \delta(\vec x - \vec x_a(t))
		=\\
		=&
		\sum_a q_a \frac{dx_a^\mu}{dr_a} \frac{dr_a}{dt}\delta(\vec x - \vec x_a(t))
		=
		\sum_a q_a \int dr \frac{dx_a^\mu}{dr}\frac{\delta (r - r_a)}{\left(\frac{dt}{dr}\right)_{r=r_a}}\delta(\vec x - \vec x_a(r))
		=\\
		=&
		\sum_a q_a \int dx_a^\mu \delta(x - x_a(r))
	\end{split}
	\end{flalign}
	$x^\mu_a$ - 4-вектор, $\delta(x - x_a(r))$ - Лоренцев скаляр $\rightarrow j^\mu_{(x)}$ - 4-х вектор.\\
\end{proof}

Из уравнений Максвелла <<с источниками>> :
\begin{gather*}
	\partial^\beta F_{\beta\alpha} = -j_\alpha \longrightarrow \partial^\alpha\partial^\beta F_{\beta\alpha} = -\partial^\alpha j_\alpha\\
	\partial^\alpha\partial^\beta F_{\beta\alpha} = \partial^\beta\partial^\alpha F_{\alpha\beta} = \partial^\alpha\partial^\beta F_{\alpha\beta} = - \partial^\alpha\partial^\beta F_{\beta\alpha} \longrightarrow \partial^\alpha\partial^\beta F_{\beta\alpha} = 0
\end{gather*}
Получаем закон сохранения электрического заряда:
$$\partial^\alpha j_\alpha = 0,$$
или:
$$\partial_t\rho + \nabla \cdot \vec j = 0$$

\begin{definition}
	Закон Ома в дифференциальной форме:
	$$\vec{j}=σ*\vec{E}$$,
	\\$σ$- коэффициент электропроводности\ проводимость тела
\end{definition}
\begin{definition}
	Закон Ома в интегральной форме:
	$$ρ\vec{j}=\vec{E}$$
	$$ρ\vec{j}dS\vec{n}=\vec{E}\vec{n}dS$$
	$$ρI=ES$$
	$$I\frac{ρdl}{S}=Edl$$
	$$I\int(\frac{ρdl}{S})=\int(\vec{E},\vec{dl})$$
\end{definition}
Границы применимости закона Ома\\

При некоторых значениях напряженности электрического поля, созданного в газах, перемещающаяся заряженная частица может приобрести такую энергию, которой достаточно для того, чтобы вызвать вторичную ионизацию молекул. Число носителей зарядов при этом возрастает, удельная электропроводность изменяется. Вследствие этого пропорциональность между плотностью тока и напряженностью электрического поля нарушается. Отклонение от пропорциональности наблюдается и при искровом разряде в газах. Оба эти случая означают явное нарушение закона Ома.

Не подчиняется закону Ома и ток в электронных лампах, ток через контакт между двумя полупроводниками или полупроводником и металлом. Катастрофическим нарушением закона Ома является ток в сверхпроводниках.

Однако для металлов ни при каких условиях не удалось заметить отклонений от пропорциональности между плотностью тока и напряженностью электрического поля.
\end{document}