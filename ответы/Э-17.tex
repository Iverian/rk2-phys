\documentclass[12pt]{article}

\usepackage[a4paper,
            total={170mm,255mm},
            left=10mm,
            top=15mm]
            {geometry}

\usepackage{fontspec,
            polyglossia,
            graphicx}

\usepackage{amsmath,
            amsthm,
            amssymb}

\usepackage{unicode-math,
            tensor}

\usepackage{wrapfig,
            hyperref,
            multicol,
            multirow,
            tabularx,
            booktabs,
            subfiles}

\setdefaultlanguage{russian}
\setotherlanguage{english}
\setkeys{russian}{babelshorthands=true}

\defaultfontfeatures{Ligatures=TeX}
\setmainfont{STIX Two Text}
\setmathfont{STIX Two Math}
\DeclareSymbolFont{letters}{\encodingdefault}{\rmdefault}{m}{it}

\newfontfamily{\cyrillicfont}{STIX Two Text} 
\newfontfamily{\cyrillicfontrm}{STIX Two Text}
\newfontfamily{\cyrillicfonttt}{Courier New}
\newfontfamily{\cyrillicfontsf}{STIX Two Text}

\renewcommand{\thefigure}{\thesection.\arabic{figure}}
\renewcommand{\thetable}{\thesection.\arabic{table}}
\numberwithin{equation}{section}

\renewcommand{\qedsymbol}{$\blacksquare$}
\theoremstyle{definition}
\newtheorem{definition}{Опр.}[section]
\theoremstyle{remark}
\newtheorem{statement}{Утв.}[section]
\theoremstyle{plain}
\newtheorem{theorem}{Теор.}[section]

\graphicspath{{./img/}}
\everymath{\displaystyle}

\newcommand{\llabel}[1]{\label{\thesubsection:#1}}
\newcommand{\lref}[1]{\ref{\thesubsection:#1}}


\begin{document}
\paragraph{Э-17}
Физический смысл компонент симметризованного тензора энергии-импульса электромагнитного поля. Вектор Пойнтинга и теорема Пойнтинга.\\

\begin{definition}
	Вектор Пойтинга: $\vec{S} = [\vec{E} \vec{B}]$ 
\end{definition}

\begin{theorem}
	Теорема Пойтинга:\\
	Если коротко, то теорема описывает закон сохранения энергии электромагнитного поля
	\begin{gather*}
		\frac{\delta\varepsilon}{\delta t}+\nabla\vec{S} = -\vec{\gamma}\cdot\vec{E}
	\end{gather*}
	где $\varepsilon = \frac{E^2+B^2}{2}$
\end{theorem}
Канонический тензор энергии-импульса имеет вид:
 \begin{gather*}
\underset{emf}{T}^{\mu\nu} = F^{\mu\lambda}\partial^\nu A_\lambda - \frac{1}{4}F_{\alpha\beta}F^{\alpha\beta}\eta^{\mu\nu}
\end{gather*}
\begin{gather*}
\underset{emf}{T^{\mu\nu}} - F^{\mu\lambda}\partial_\lambda A^\nu = F^{\mu\lambda}(\partial^\nu A_x - \partial_\lambda A^\nu) - \frac{1}{4}F_{\alpha\beta}F^{\alpha\beta}\eta^{\mu\nu} = F^{\mu\lambda}F^\nu_\lambda - \frac{1}{4}F_{\alpha\beta}F^{\alpha\beta}\eta^{\mu\nu}\equiv \underset{emf}{\Theta}^{\mu\nu}
\end{gather*}
 $\underset{emf}{\Theta}^{\mu\nu}$ - симметризованый тензор энергии-импульса электромагнитного поля.\\
Выразим компоненты тензора $T^{\mu\nu}$ через напряжённости электрического и магнитного полей. C помощью значений 
$F_{\mu\nu}= 
\begin{bmatrix}
	0 & E_x & E_y & E_z\\
-E_x & 0 & -H_z & H_y\\
-E_y & H_z & 0 & -H_x\\
-E_z & -H_y & H_x & 0
\end{bmatrix},
\quad F^{\mu\nu} = 
\begin{bmatrix}
0 & -E_x & -E_y & -E_z\\
E_x & 0 & -H_z & H_y\\
E_y & H_z & 0 & -H_x\\
E_z & -H_y & H_x & 0
\end{bmatrix}
$
легко убедиться в том, что $T^{00}$ совпадает, как и следовало, с плотностью энергии $W = \frac{E^2 + H^2}{8\pi}$, а компоненты $cT^{0\alpha}$ - с компонентами вектора Пойнтинга $S = \frac{c}{4\pi}[EH]$. Пространственные же компоненты $T^{\alpha\beta}$ образуют трёхмерный тензор с составляющими
\begin{gather*}
	\sigma_{xx} = \frac{1}{8\pi}(E^2_y + E^2_z - E^2_x + H^2_y + H^2_z - H^2_x),\\
	\sigma_{xy} = -\frac{1}{4\pi}(E_xE_y + H_xH_y)
\end{gather*}
и т. д., или
\begin{gather*}
	\sigma_{\alpha\beta} = \frac{1}{4\pi}\{-E_\alpha E_\beta - H_\alpha H_\alpha + \frac{1}{2}\delta_{\alpha\beta}(E^2 + H^2)\}.
\end{gather*}
 Этот трёхмерный тензор называют максвелловским тензором напряжений.
 
\end{document}