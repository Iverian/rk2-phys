\documentclass[__main__.tex]{subfiles}

\begin{document}
\paragraph{Билет 22}
Магнитный момент атома, свзанный с орбитальным моментом импульса электрона: квазиклассическое и квантомеханическое рассмотрения.

1) \textbf{Квазиклассическое рассмотрение}

Рассмотрим квазиклассическую модель атома: тяжелое ядро, вокруг ядра вращающийся со скоростью v электрон, радиус-вектор которого заметает за период T площадку S. 

Круговой ток:
\begin{gather}
\label{K24/1}
I=\frac{-e}{T} 
\end{gather}

Магнитный момент тока:
\begin{gather}
\label{K24/2}
 \mu=\frac{e}{T}S=\frac{ev}{2\pi r}\pi r^{2}= \frac{evr}{2} 
\end{gather}

Магнитный момент электрона:
\begin{gather}
\label{K24/3}
 L= r v m_{e} 
\end{gather}

Так что момент имульса тока можно связть с моментом импульса электрона:
\begin{gather}
\label{K24/4}
 \vec{\mu}= -\frac{e}{2m_{e}}\vec{L}= -\frac{e\hbar}{2} m_{e}[=\mu_{b}]\frac{\vec{L}}{\hbar} 
\end{gather}
 где $\mu_{b}$ - магнетон Бора
 
 2) \textbf{Квантомеханическое рассмотрение}
 Магнитный момент атома обусловленный током:
 \begin{gather}
 \label{K24/5}
  \ dI=-e\vec{j}d\vec{A}
 \end{gather}
 где $\vec{j}$ - плотность тока, A - элемент площадки сферы.
 
 Расмотрим проекцию на ось Z момента импульса электрона:
 
 \begin{gather}
 \label{K24/6}
 \mu_{z}=\int dI S= -\frac{e\hbar m}{2m_e} \int |\psi|^2 2 \pi r s v dA 
 \end{gather}
  где  $\int |\psi|^2 2 \pi r s v dA=1, \hbar m = \bar{L}$- собственные значения оператора $\hat{L}$,  m - масса атома, тогда  по условию нормировки
   можно записать:
   \begin{gather}
  \label{K24/7}
 \mu_{z} = -\mu_{b}m 
  \end{gather}
  
  Тоже самое для проекции на ось OX, OY,  момента импульса электрона, аналогичен квазиклассическому состоянию  :
  
  \begin{gather}
  \label{K24/8}
  \hat{\mu}=-\mu_{b}\frac{\hat{L}}{\hbar}
  \end{gather}


\end{document}