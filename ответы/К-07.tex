\documentclass[__main__.tex]{subfiles}

\begin{document}
\paragraph{K-07}
Рассмотрите задачу об электроне в трёхмерной бесконечно глубокой потенциальной яме. Какова минимальная кинетическая энергия электрона? Чему равна кратность вырождения энергетического уровня $ 27 \frac{\pi^2\hslash^2}{2m^2}+U_0$ (где $L$ ширина ямы)?.\\

Обобщим задачу об электроне в одномерной бесконечно глубокой потенциальной яме:\\

$$\psi_{{n_1},{n_2},{n_3}}=\psi_{n_1}(x)\psi_{n_2}(y)\psi_{n_3}(z) = \sqrt{\frac{8}{l_1 l_2 l_3}}sin\left(\frac{\pi n_1 x}{l_1}\right)sin\left(\frac{\pi n_2 y}{l_2}\right)sin\left(\frac{\pi n_3 z}{l_3}\right)$$

Тогда кинетическая энергия электрона равна:
$$T_{n_1 n_2 n_3} = \frac{\pi^2\hbar^2}{2m}\left(\frac{{n_1}^2}{{l_1}^2}+\frac{{n_2}^2}{{l_2}^2}+\frac{{n_3}^2}{{l_3}^2}\right)$$

Так как $l_1 = l_2 = l_3 = L$, то: 

$$T_{n_1 n_2 n_3} =  \frac{\pi^2\hbar^2}{2m L^2}\left({n_1}^2+{n_2}^2+{n_3}^2\right)$$

Cоответственно, минимальная кинетическая энергия электрона равна: $T_{min} =   \frac{3\pi^2\hbar^2}{2m L^2}$\\

\begin {definition}
Количество наборов квантовых чисел $n_1$, $n_2$, $n_3$ соотвествующих одной и той же энергии, называется кратностью этого энергетического уровня.
\end {definition}

Для энергетического уровня $ 27 \frac{\pi^2\hslash^2}{2m^2}+U_0$, то есть для случая ${n_1}^2+{n_2}^2+{n_3}^2 = 27$ , будут соответствовать следующие наборы  $n_1$, $n_2$, $n_3$:

\begin {table}[h]
\centering
\begin {tabular}{llr}
\toprule
$n_1$ & $n_2$ & $n_3$ \\
\midrule
5 & 1 & 1 \\
1 & 5 & 1 \\
1 & 1 & 5 \\
3 & 3 & 3 \\
\bottomrule
\end {tabular}
\end {table}

Таким образом кратность равна 4.

\end{document}