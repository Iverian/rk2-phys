\documentclass[__main__.tex]{subfiles}

\begin{document}
\paragraph{Э-19}
	Покажите, что канонический тензор энергии-импульса электромагнитного
поля 
$$
T^{\mu\nu}=F^{\mu\lambda}\partial^\nu A_\lambda-\frac{1}{4}F_{\alpha\beta}F^{\alpha\beta}\eta^{\mu\nu}
$$
не является калибровочно инвариантным в отличие от симметризованного тензора энергии-импульса электромагнитного поля 
$$
\Theta^{\mu\nu} = T^{\mu\nu}-F^{\mu\lambda}\partial_{\lambda}A^{\nu}
$$.
\textbf{Калибровочные преобразования}\\
$$\mathcal A_\mu \longrightarrow \mathcal A'_\mu \equiv \mathcal A_\mu +\partial_\mu f(x)$$
Матерь божья, тут дичь. Если вы читаете это, то пререпишите ниженаписанное ниже, запишите определение этих преобразований, и попытайтесь их применить...\\
Помните, что по временной компоненте производная отрицательная, к примеру:\\ 
\textbf{Инвариантность тензора Максвелла относительно калибровочных преобразований(это не писать, это пример)}\\
$$F_{\mu\nu} \longrightarrow F'_{\mu\nu} \equiv \partial_\mu \mathcal A'_\nu - \partial_\nu \mathcal A'_\mu = \partial_\mu \mathcal A_\nu + \partial_\mu \partial_\nu f(x) - \partial_\nu \mathcal A_\mu - \partial_\nu\partial_\mu f(x) = F_{\mu\nu}$$
\textbf{Связь полей $\vec E$ и $\vec B$ с 4-векторным потенциалом $A^\mu$}\\
\begin{flalign*}
\left. \begin{matrix}
-E_x = F_{01} = \partial_0\mathcal A_1 - \partial_1 \mathcal A_0\\
-E_y = F_{02} = \partial_0\mathcal A_2 - \partial_2 \mathcal A_0\\
-E_z = F_{03} = \partial_0\mathcal A_3 - \partial_3 \mathcal A_0\\
\end{matrix}\right\}& \qquad \vec E = -\nabla\phi - \partial_t \vec \mathcal A\\
\left. \begin{matrix}
B_x = F_{23} = \partial_2\mathcal A_3 - \partial_3 \mathcal A_2\\
-B_y = F_{13} = \partial_1\mathcal A_3 - \partial_3 \mathcal A_1\\
B_z = F_{12} = \partial_1\mathcal A_2 - \partial_2 \mathcal A_1\\
\end{matrix}\right\}& \qquad \vec B = \nabla\times \vec \mathcal A
\end{flalign*}
\begin{gather*}
p_\alpha = (- \mathcal E, \vec p)
\frac{dp_\alpha}{dr} = qF_{\alpha\beta}\frac{dx^\beta}{dr};\\
dr = \sqrt{1-v^2}dt = \frac{dt}{\gamma} \rightarrow \gamma\frac{dp_\alpha}{dt} = qF_{\alpha\beta}\gamma\frac{dx^\beta}{dt} \rightarrow \frac{dp_\alpha}{dt} = qF_{\alpha\beta}\frac{dx^\beta}{dt};\\
\frac{d}{dt}\left(\begin{matrix}-\mathcal E\\ p_x\\p_y\\p_z\end{matrix}\right) = 
q\left(\begin{matrix}
0&-E_x&-E_y&-E_z\\
E_x&0&B_z&-B_y\\
E_y&-B_z&0&B_x\\
E_z&B_y&-B_x&0
\end{matrix}\right)\left(\begin{matrix}1\\v_x\\v_y\\v_z\end{matrix}\right)
\end{gather*}
или:
\begin{gather*}
\frac{d\mathcal E}{dt} = q\vec E\cdot \vec v\\
\frac{d\vec p}{dt} = q\vec E + q\vec v \times \vec B (\text{Сила Лоренца})
\end{gather*}

\end{document}