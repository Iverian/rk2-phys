\documentclass[__main__.tex]{subfiles}

\begin{document}
\paragraph{Э-19}
Система уравнений Максвелла-Лоренца, материальные уравнения, условия на границе раздела двух диэлектриков, магнетиков\\

Вывод для Гениев!\\
Формулы для пацанов!\\
\begin{gather*}
	\oint Edl = \frac{1}{c}\frac{\partial}{\partial t}\int Hdf,\\
	\oint H df = 0.\\
	div E = 4\pi\rho.\\
	rot H = \frac{1}{c}\frac{\partial E}{\partial t} + \frac{4\pi}{c}j.\\
\end{gather*}

 \textbf{Материальные уравнения}

\begin{gather*}
	\mathbf {D} =\varepsilon _{0}\varepsilon \mathbf {E} =\varepsilon _{0}(1+\chi _{e})\mathbf {E} \\
	\mathbf {B} =\mu _{0}\mu \mathbf {H} =\mu _{0}(1+\chi _{m})\mathbf {H} \\
\end{gather*}
$\varepsilon-$ относительная диэлектрическая проницательность\\
$\mu-$ относительная магнитная проницаемость\\
$\chi _{e}-$ диэлектрическая восприимчивость\\
$\chi _{m}-$ магнитная восприимчивость

 \textbf{Граничные уравнения}\\

 \begin{gather*}
	(\mathbf {E} _{1}-\mathbf {E} _{2})\times \mathbf {n} _{1-2}=\mathbf {0} \\
{\displaystyle (\mathbf {H} _{1}-\mathbf {H} _{2})\times \mathbf {n} _{1-2}=\mathbf {j} _{s}} 
\end{gather*}
$\mathbf {n} _{1-2}$ - единичный вектор нормали к поверхности, направленный из среды 1 в среду 2 и имеющий размерность, обратную длине\\ 
$\mathbf {{j} _{s}}$ - плотность поверхностных свободных токов вдоль границы\\

\textbf{Вывод Системы уравнений Максвелла-Лоренца в вакууме(для лохов)}\\

\begin{gather*}
	H = rot A, \quad E = -\frac{1}{c}\frac{\partial A}{\partial t} - grad \varphi,
\end{gather*}
где $H$ - напряжённость магнитного поля, $E$ - напряжённость электрического поля,
$A$ - векторный потенциал поля,
$\varphi$ - скалярный потенциал.

Определим $rot E$:
\begin{gather*}
	rot E = -\frac{1}{c}\frac{\partial}{\partial t}rot A - rot grad \varphi.
\end{gather*}

Ротор градиента равен нулю.
\begin{gather}
	\label{e-02-rotE}
	rot E = -\frac{1}{c}\frac{\partial H}{\partial t}.
\end{gather}

Возьмём дивергенцию от обеих частей уравнения $rot A = H$ (дивергенция ротора равна нулю):
\begin{gather}
	\label{e-02-divH}
	div H = 0
\end{gather}

Уравнения $\ref{e-02-rotE}$ и  $\ref{e-02-divH}$ составляют первую пару уравнений Максвелла.\\
В интегральной форме:\\
\begin{gather*}
	\oint Edl = \frac{1}{c}\frac{\partial}{\partial t}\int Hdf,\\
	\oint H df = 0.
\end{gather*}

Для вывода второй пары уравнений Максвелла нам нужно знать, что действие:
\begin{gather}
	\label{e-02-S}
	S = -\sum\int mcds - \frac{1}{c^2}\int A_ij^id\Omega - \frac{1}{16\pi c}\int F_{ik}F^{ik}d\Omega.
\end{gather}

При нахождении уравнений поля их принципа наименьшего действия мы должны считать заданным движения зарядов и должны варьировать только потенциалы поля(играющие здесь роль "координат" системы); при нахождении уравнений движения мы, наоборот, считали поле заданным и варьировали траекторию частицы.
Поэтому вариация первого члена в $\ref{e-02-S}$ равна теперь нулю, а во втором не должен варьироваться ток $j^i$. Таким образом, 
\begin{gather*}
	\delta S = -\frac{1}{c}\big[\frac{1}{c}j^i\delta A_i + \frac{1}{8\pi}F^{ik}\delta F_{ik}\big]d\Omega = 0
\end{gather*}
(при варьировании во втором члене учтено, что $F^{ik}\delta F_{ik} \equiv F_{ik}\delta F^{ik}$). Подставляя
\begin{gather*}
	F_{ik} = \frac{\partial A_k}{\partial x^i} - \frac{\partial A_i}{\partial x^k},
\end{gather*}
имеем:
\begin{gather*}
	\delta S = -\frac{1}{c}\int \big\{\frac{1}{c}j^i\delta A_i + \frac{1}{8\pi}F^{ik}\frac{\partial}{\partial x^i}\delta A_k - \frac{1}{8\pi}F^{ik}\frac{\partial}{\partial x^k}\delta A_i\big\}d\Omega.
\end{gather*}
Во втором члене меняем местами индексы $i$ и $k$, по которым производится суммирование, и, кроме того, заменяем $F_{ki}$ на $-F_{ik}$.\\
Тогда мы получим:
\begin{gather*}
	\delta S = -\frac{1}{c}\int\big\{\frac{1}{c}j^i\delta A_i - \frac{1}{4\pi}F^{ik}\frac{\partial}{\partial x^k}\delta A_i\big\}d\Omega.
\end{gather*}
Второй из этих интегралов берём по частям, т.е. применяем теорему Гаусса:
\begin{gather}
	\label{e-02-deltaS}
	\delta S = -\frac{1}{c}\int\big\{\frac{1}{c}j^i + \frac{1}{4\pi}\frac{\partial F^{ik}}{\partial x^k}\big\}\delta A_id\Omega - \frac{1}{4\pi c}\int F^{ik}\delta A_idS_k\big|.
\end{gather}
Во втором члене  мы должны взять его значение на пределах интегрирования. Пределами интегрирования по координатам является бесконечность, где поле исчезает. На пределах же интегрирования по времени, т.е. в заданные начальный и конечный моменты времени, вариация потенциалов равна нулю, так как по смыслу принципа наименьшего действия потенциалы в эти моменты заданы. Таким образом, второй член в $\ref{e-02-deltaS}$ равен нулю, и мы находим:
\begin{gather*}
	\int(\frac{1}{c}j^i + \frac{1}{4\pi}\frac{\partial F^{ik}}{\partial x^k})\delta A_id\Omega = 0.
\end{gather*}
Ввиду того, что по смыслу принципа наименьшего действия вариации $\delta A_i$ произвольны, нулю должен равняться коэффициент при $\delta A_i$, т.е.
\begin{gather}
	\label{e-02-deltaF}
	\frac{\partial F^{ik}}{\partial x^k} = -\frac{4\pi}{c}j^i.
\end{gather}
Перепишем эти четыре ($i = 0, 1, 2, 3$) уравнения в трёхмерной форме. При $i=1$ имеем:
\begin{gather*}
	\frac{1}{c}\frac{\partial F^{10}}{\partial t} + \frac{\partial F^{11}}{\partial x} + \frac{\partial F^{12}}{\partial y} + \frac{\partial F^{13}}{\partial z} = -\frac{4\pi}{c}j^1.
\end{gather*}
Подставляя значения составляющих тензора $F^{ik}$, находим:
\begin{gather*}
	\frac{1}{c}\frac{\partial E_x}{\partial t} - \frac{\partial H_z}{\partial y} + \frac{\partial H_y}{\partial z} = -\frac{4\pi}{c}j_x.
\end{gather*}
Вместе с двумя следующими $(i = 2, 3)$ уравнениями они могут быть записаны как одно векторное:
\begin{gather}
	\label{e-02-rotH}
	rot H = \frac{1}{c}\frac{\partial E}{\partial t} + \frac{4\pi}{c}j.
\end{gather}
Наконец, уравнение с $i=0$ даёт:
\begin{gather}
\label{e-02-divE}
	div E = 4\pi\rho.
\end{gather}
Уравнения $\ref{e-02-rotH}$ и $\ref{e-02-divE}$ и составляют вторую пару уравнения Максвелла.\\
В интегральной форме:\\
\begin{gather*}
	\oint Hdl = \frac{1}{c}\frac{\partial}{\partial t}\int Edf + \frac{4\pi}{c}\int jdf,\\
	\oint Edf = 4\pi\int \rho dV.
\end{gather*}
 Уравнения Максвелла являются основными уравнениями электродинамики.\\

\end{document}