\documentclass[__main__.tex]{subfiles}

\begin{document}
\paragraph{K-18}
Электрон в сферическо симметричном потенциале. Главное, орбитальое, магнитное квантовые числа. Опыт Штерна-Герлаха. Гипотеза спина электрона.

Получив решение уравнения для энергии электрона, можно дать определния квантовым числам:
\begin{gather*}
	E_n  = -\frac{Z^2}{n^2}\frac{k^2me^4}{2\hbar^2},
\end{gather*}
где n - главное квантовое число, принадлежащее натуральным, характеризующее уровень энергии электрона; c ростом n возрастоает радиус орбиты и энергия электрона.
m - магнитное квантовое число,характеризующее ориентацию в пространстве момента импульса, изменяющееся в пределах [-\eta,\eta], где
 \eta - орбитальное квантовое число, определяющее амплитуду волновой функции электрона, т.е. за форму электронного облака. Меняется от [0,n-1], каждому значению сопоставлена буква s,p,d,f,g...
 \\
 Опыт Ш-Г заключается в следующем:
 "Направим пучок атомов в область неоднородного магнитного поля. Под действием силы они отклонятся от первоначального направления, величина отклонения тем больше, чем больше модуль магнитного момента, вверх или вниз зависит от знака момента. Согласно квантовой механике магнитный момент квантуется, потому исходный пучок обязан расщипиться на число пучков разрешённых проекций магнитного момента на ось z. При фиксированном орбитальном числе, количество возможных значений магнитного числа нечётно(2\eta +1),  поэтому если мы возьмём пучок атомов водорода, где все числа нули, то на выходе пучок теоретически не должен расщепиться. НО он расщепляется на 2 пучка. Отсюда следует ГИПОТЕЗА о наличии у электрона собственного момента импульса, или спина, и полный момент импульса определяется как сумма орбитального момента и спина."
\end{document}