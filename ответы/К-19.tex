\documentclass[__main__.tex]{subfiles}

\begin{document}

\paragraph{К-19}
Воспользуйтесь соотношением Вина для спектральной плоскости энергии, чтобы получить закон смещения Вина.
\begin{definition}
	Соотношением Вина для спектральной плотности энергии называется выражение: 
	\begin{gather}
	\llabel{29/1}
          u_{\nu}=C_{1}\nu^{3}e^{\frac{-C_{2}\nu}{T}},
	\end{gather}
где $C_{1}$ и $C_{2}$ - неопределенные константы, $u_{\nu}$ - плотность энергии излучения, $T$ - температура излучающего тела, $\nu$ - частота излучения
\end{definition}

\begin{definition}
	Абсолютным черным телом называется физическое тело, котороые при любой температуре поглощает все падающее на него электромагнитное излучение во всех диапазонах.
\end{definition}
Но Вин нашел зависимость : $\nu_{max} = f(T)$, где $\nu_{max}$ - частота соответствующая максимальному значению $u_{\nu}$ абсолютно черного тела.

Закон смещения Вина получим, найдя максимум функции (\lref{29/1}), то есть производную по $\nu$ и приравняем ее к нулю.

$$
   \frac {\partial u_{\nu}}{\partial \nu}= C_{1}3\nu_{max}^{2}e^{\frac{-C_{2}\nu_{max}}{T}}+
   C_{1}\nu_{max}^{3}e^{\frac{-C_{2}\nu_{max}}{T}}(-\frac{C_{2}}{T}),
$$
тогда
$$
    \frac{\nu_{max}}{T}=\frac{3}{C_{2}}=const.
$$
Тогда закон смещения Вина:
\begin{gather}
    \frac{\nu_{max}}{T}=const.
    \llabel{goof}
\end{gather}
Но чаще закон смещения Вина записывают в виде: $\lambda_{max}=\frac{b}{T}$, где постоянная Вина $b=2,9\cdot10^{-3}$ м$\cdot$К. 
\end{document}