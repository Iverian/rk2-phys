\documentclass[__main__.tex]{subfiles}

\begin{document}

\paragraph{Э-20}
Оценить среднюю объемную плотность электрических зарядов в атмосфере($\rho$), если известно, что напряженность электрического поля на поверхности Земли составляет примерно 130 В/м, а на высоте 1 км - примерно 40 В/м.\\

\textbf{Решение:}

Напряженность электрического поля на поверхности Земли:
\begin{equation}
    E_{0}=\frac{Q_{3}}{4\pi\epsilon_{0} R^2_{3}}
\end{equation}
где $Q_{3}$ - суммарный заряд Земли, $R_{3}$ - радиус Земли.

На высоте $h$ напряженность эл. поля складывается из напряженности поля заряда Земли и из напряженности поля, созданного зарядом $Q_{а}$ - верхних слоев атмосферы с радиусом  $R_{3}+h$.

\begin{equation}
    E_{h}=\frac{Q_{3}+Q_{а}}{4\pi \epsilon_{0} (R_{3}+h)^{2}},
\end{equation}

Так как:
\begin{equation}
    Q_{a}=(\frac{4}{3}\pi (R_{3}+h)^{3}-\frac{4}{3}\pi (R_{3})^{3})\rho,
\end{equation}
то:
\begin{equation}
    E_{h}=\frac{Q_{3}}{4\pi \epsilon_{0} (R_{3}+h)^{2}}+\frac{\rho}{3\epsilon_{0}}(R_{3}+h-\frac{R_{3}^{3}}{(R_{3}+h)^{2}})
\end{equation}
Так как радиус земли много больше $h$, то запишем $R_{3}=R_{3}+h$, тогда:
\begin{equation}
    E_{h}=\frac{Q_{3}}{4\pi \epsilon_{0} (R_{3}+h)^{2}}+\frac{\rho h}{3\epsilon_{0}}=E_{0}+\frac{\rho h}{3\epsilon_{0}}.
\end{equation}

тогда в итоге средняя объемная плотность эл.зарядов в атмосфере:
\begin{equation}
    \rho=\frac{3\epsilon_{0}(E_{h}-E_{0})}{h}=\frac{3 8.85*10^{-12*}(40-130)}{1000}=-2.38*10^{-12}[Кл * м^{-3}].
\end{equation}

\end{document}