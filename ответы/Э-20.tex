\documentclass[__main__.tex]{subfiles}

\begin{document}
Оценить среднюю обьемную плотность электрических зрядов в атмосфере($\rho$), если известно, что напряженность эектрического поля на поверхности Земли составляет примерно 130 В/м, а на высоте 1 км - примерно 40 В/м.

\textbf{Решение}

Напряженность электрического поля на поверхности Земли:
\begin{equation}
\label{Э-20\1}
E_{0}=\frac{Q_{3}}{4\pi \epsilon_{0} R^2_{3}}
\end{equation}
 где $Q_{3}$ - суммарный заряд Земли, $R_{3}$ - радиус Земли. 
 
 На высоте h напряженность эл. поля складывается из напряженности поля заряда Земли и из напряженности поля, созданнового зарядом $Q_{а}$ - верхних слоев атмосферы с радиусом  $R_{3}+h$.
 
 \begin{equation}
 \label{Э-20\2}
 E_{h}=\frac{Q_{3}+Q_{а}}{4\pi \epsilon_{0} (R_{3}+h)^{2}},
 \end{equation}
 
 Так как:
  \begin{equation}
 \label{Э-20\3}
 Q_{a}=(\frac{4}{3}\pi (R_{3}+h)^{3}-\frac{4}{3}\pi (R_{3})^{3})\rho,
 \end{equation}
 то:
  \begin{equation}
 \label{Э-20\4}
 E_{h}=\frac{Q_{3}}{4\pi \epsilon_{0} (R_{3}+h)^{2}}+\frac{\rho}{3\epsilon_{0}}(R_{3}+h-\frac{R_{3}^{3}}{(R_{3}+h)^{2}})
 \end{equation}
 Так как радиус земли много больше h, то запишем $R_{3}=R_{3}+h$, тогда:
  \begin{equation}
 \label{Э-20\5}
 E_{h}=\frac{Q_{3}}{4\pi \epsilon_{0} (R_{3}+h)^{2}}+\frac{\rho h}{3\epsilon_{0}}=E_{0}+\frac{\rho h}{3\epsilon_{0}}.
 \end{equation}
 
 тогда в итоге средняя обьемная плотность эл.зарядов в атмосфере:
  \begin{equation}
 \label{Э-20\6}
\rho=\frac{3\epsilon_{0}(E_{h}-E_{0})}{h}=\frac{3 8.85*10^{-12*}(40-130)}{1000}=-2.38*10^{-12} [Кл * м^{-3}].
 \end{equation}
 
\end{document}