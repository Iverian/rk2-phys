\documentclass[__main__.tex]{subfiles}

\begin{document}

\paragraph{K-25}
Постулаты квантовой механики: о квантовых состояниях, о физических величинах, об измерениях, динамический постулат.\\

\textbf{Постулат о квантовых состояниях.}\\
Квантовое состояние полностью задается $\psi$-функцией из пространства волновых функций. $\psi$-функции, отличающиеся только комплексным множителем, задают одно и то же состояние.\\

\textbf{Постулат о физических величинах.}\\
Каждой физической величине ставится в соответствие эрмитов оператор, обладающий полной системой собственных функций.\\

\textbf{Постулат об измерениях.}\\
Пусть измеряется некоторая величина $A$ и ей в соответствие поставлен оператор $\hat{A}: A \rightarrow \hat{A}$. Если оператор эрмитов (т.е. $\hat{A}^{+}=\hat{A}$, то его собственные значения вещественны и $\hat{A}\varphi_n=a_n\varphi_n$.\\

\textbf{Динамический постулат.}\\
Все предсказания, которые могут быть сделаны относительно различных свойств системы в данный момент времени, следуют из значения $\psi$-функции в этот момент времени.\\

\end{document}