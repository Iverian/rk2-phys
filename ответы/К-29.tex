\documentclass[__main__.tex]{subfiles}

\begin{document}
\paragraph{К-29}
Сформулируйте постулат квантовой механики о физических величинах. Покажите, что все собственные значения эрмитова оператора суть вещественные числа.\

\textbf{Постулат квантовой механики о физических величинах}\\
Каждой физической величине ставится в соответствие эрмитов оператор обладающий полной системой собственных функций.
\begin{definition}
	$A$-эрмитов, если $(Ax,y)=(x,Ay)$	
\end{definition}

\begin{statement}
	Все собственные значения эрмитова оператора $\hat{A}$ - вещественные числа.
\end{statement}
\begin{proof}
	Пусть $\psi_a$ - собственная функция оператора $\hat{A}$, тогда при воздействии оператора $\hat{A}$ на эту функцию получим эту же функцию домноженную на какое-то число $a$:
	\begin{gather*}
		\hat{A}\psi_a = a\psi_a
	\end{gather*}
	Необходимо доказать, что $a$ - вещественное. Для этого воспользуемся утверждением того, что если $\hat{A}$ - эрмитов оператор, то скалярное произведение $(\psi_a,\hat{A}\psi_a)$ - вещественное.
	\begin{gather*}
		(\psi_a,\hat{A}\psi_a) = (\psi_a,a\psi_a) = a(\psi_a,\psi_a)
	\end{gather*}
	Ввиду того, скалярное произведение по определению $(\psi_a,\psi_a) = \vert\vert\psi_a\vert\vert^2$ - вещественное число, то "$a$" ничего не остается как быть только вещественным т.к левая часть уравнения по утверждению выше  - вещественная.
\end{proof}
\end{document}