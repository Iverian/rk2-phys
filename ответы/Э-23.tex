\documentclass[__main__.tex]{subfiles}

\begin{document}
\paragraph{Э-23}
Применение теоремы Гаусса к расчету электростатическиъ полей: найдите поле, порождаемое бесконечной равномерно заряженной плоскость. (поверхностная плотность заряда $\sigma$), бесконечной равномеромерно заярежнной нитью (линейная плотность заряда $\kappa$)\\

Бесконечная плоскость заряжена с постоянной поверхностной плотностью $\sigma$ ($\sigma=dQ/dS$ — заряд, приходящийся на единицу поверхности). Линии напряженности перпендикулярны рассматриваемой плоскости и направлены от нее в обе стороны. В качестве замкнутой поверхности мысленно построим цилиндр, основания которого параллельны заряженной плоскости, а ось перпендикулярна ей. Так как образующие цилиндра параллельны линиям напряженности (соsa=0), то поток вектора напряженности сквозь боковую поверхность цилиндра равен нулю, а полный поток сквозь цилиндр равен сумме потоков сквозь его основания (площади оснований равны и для основания Еn совпадает с Е), т. е. равен 2ES. Заряд, заключенный внутри построенной цилиндрической поверхности, равен sS. Согласно теореме Гаусса,  $2ES=sS/e0$`, откуда 
$$E = \frac{\sigma}{2e_0}$$
Следует, что поле равномерно заряженной плоскости однородно.


Поле равномерно заряженного бесконечного цилиндра (нити). Бесконечный цилиндр радиуса R заряжен равномерно с линейной плотностью  $\kappa$ ( $\kappa=\frac{dQ}{dV}$   –  заряд, приходящийся на единицу длины). Из соображений симметрии следует, что линии напряженности будут направлены по радиусам круговых сечений цилиндра с одинаковой густотой во все стороны относительно оси цилиндра. В качестве замкнутой поверхности мысленно построим коаксиальный с заряженным цилиндр радиуса r и высотой l. Поток вектора Е сквозь торцы коаксиального цилиндра равен нулю  (торцы параллельны линиям напряженности), а сквозь боковую поверхность равен $2prlЕ$. По теореме Гаусса, при r>R $2prlЕ = \frac{\kappa*l}{e_0}$, откуда
$$E=\frac{1}{2\pi e_0}\frac{\kappa}{r} (r >= R)$$
Если $r<R$, то замкнутая поверхность зарядов внутри не содержит, поэтому в этой области $E=0$. Таким образом, напряженность поля вне равномерно заряженного бесконечного цилиндра определя­ется выражением, внутри же его поле отсутствует.
\end{document}