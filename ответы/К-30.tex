\documentclass[__main__.tex]{subfiles}

\begin{document}
\paragraph{K-30}
Старая квантовая механика - постулаты Бора и правило Ритберца-Ритца	.\\

\textit{Комбинационное правило Ритберга-Ритца:} принцип, гласящий, что все спектральные линии некоторого элемента могут быть представлены через комбинации величин, называемых \textit{термами}.
\begin{gather*}
	\forall n_1,n_2\in\mathbb{N} \ n_1 < n_2 \colon \frac{1}{\lambda_{n_1 n_2}} = T_{n_1} - T_{n_2},
\end{gather*}
где $\lambda_{n_1 n_2}$ -- длина волны спектральной линии $n_1 n_2$ ($n_1<n_2$), $T_{n_1}, T_{n_2}$ -- термы, соответствующие $n_1$ и $n_2$.

Покажем как с помощью правила выразить одну спектральную линию через две другие:
\begin{gather*}
	\frac{1}{\lambda_{n_1 n_2}}
	=
	T_{n_1} - T_{n_2}
	=
	T_{n_1}-T_{n_3} - (T_{n_2}-T_{n_3})
	=
	\frac{1}{\lambda_{n_1 n_3}} - \frac{1}{\lambda_{n_3 n_2}}
\end{gather*}

Термы для атома водорода вычисляются следующим образом:
$
\forall n\in\mathbb{N}\colon T_n = \frac{R_H}{n^2},
$
где $R_H$ -- постоянная Ритберга для водорода.


\textit{Постулаты Бора:}
\begin{enumerate}
	\item Атом может длительно пребывать только в стационарных состояниях, характеризующихся определенной энергией $E_1,E_2,...$. В таких состояниях атомы не излучают электромагнитных волн.
	\item Излучение света происходит при переходе атома из состояния с большей энергией в состояние с меньшей энергией. Энергия излученного фотона равна разности энергий стационарных состояний.
	\begin{gather*}
		\hbar\omega_{n_1 \leftarrow n_2} = E_{n_2} - E_{n_1}
	\end{gather*}
	\item \textit{Правило квантования круговых орбит электрона:} момент импульса электрона, вращающегося на стационарной орбите атома водорода может принимать только дискретные значения
	\begin{gather*}
		\forall n\in\mathbb{N}\colon L = n\hbar
	\end{gather*}
\end{enumerate}
\end{document}