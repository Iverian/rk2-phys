\documentclass[__main__.tex]{subfiles}

\begin{document}
\paragraph{20}
Временная эволюция классической величины и временная эволюция квантовомеханического среднего. Интеграл движения в классической и квантовой механике.\\

Будем иметь дело только с временной функцией, нормированной на единицу, т.е.
$\|\psi(t_{0}, \vec{r})\|^{2}=1
\Rightarrow\langle A \rangle =(\psi, \hat{A}\psi)$

Исследуем как квантомеханическое среднее эволюционирует по времени:

\begin{flalign*}
	\begin{split}
		\frac{d}{dt} \langle A(t) \rangle
		&=
		\left.
		\frac{d}{dt}(\psi(t,\vec{r}),\hat{A}\psi(t,\vec{r}))=(\partial_{t}\psi,\hat{A}\psi)+(\psi,(\partial_{t}\hat{A})\psi)+(\psi,(\hat{A})\partial_{t}\psi)
		\right|_{\partial_{t}\psi=\frac{1}{i\hbar}\hat{H}\psi}
		=\\
		&=(\psi,(\partial_{t}\hat{A})\psi)+\frac{1}{i\hbar}\hat{H}\psi((\psi , \hat{A}\hat{H}\psi)-(\hat{H}\psi, \hat{A}\psi)).
	\end{split}
\end{flalign*}
При этом $(\hat{H}\psi, \hat{A}\psi)=(\psi,\hat{H} \hat{A}\psi)$, не забываем, что $\hat{H}$ -- эрмитов оператор, можем переставлять в скобочках. Также $(\psi,(\hat{A}\hat{H}-\hat{H}\hat{A})\psi)=(\psi,[\hat{A},\hat{H}]\psi).$ $\langle A H \rangle = (\psi,[\hat{A},\hat{H}]\psi).$
Тогда  

\begin{gather}
	\llabel{_30:final}
	\frac{d}{dt}\langle A(t) \rangle = \langle \partial_{t}\hat{A} \rangle + \frac{1}{i\hbar}\langle [\hat{A},\hat{H}] \rangle
\end{gather}
В (\lref{_30:final}) собственно показана эволюция квантомеханического среднего. 

Если нет явной зависимости физической величины $A$ от времени, то :

$$
\frac{d}{dt}\langle A(t) \rangle =\frac{1}{i\hbar}\langle [\hat{A},\hat{H}] \rangle
$$

Если теперь выясняется, что $ [\hat{A},\hat{H}]=0$, то интеграл движения предтавляется в виде:

\begin{gather*}
	\frac{d}{dt}\langle A(t) \rangle=0.
\end{gather*}

\end{document}