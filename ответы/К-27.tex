\documentclass[__main__.tex]{subfiles}

\begin{document}
\paragraph{К-27}
Продемонстрируйте сохранение квадрата нормы волновой функции во времени. Вектор плотности тока вероятности. Получите уравнение непрерывности для плотности вероятности.	\newline

У нас имеется уравнение Шрёдингера: $\hat\mathcal{H}\;\Psi(t,\vec r)=i\hbar\frac{\partial}{\partial t}\Psi(t,\vec r).$ Мы делаем допущение, что потенциальная энергия в операторе Гамильтона $\hat{\mathcal H}$ не зависит явно от времени, и получаем возможность использовать метод разделения переменных Фурье для нахождения функции $\Psi$. В состоянии типа стационарного :
\begin{gather*}
\Psi(t,\vec r)=e^{-\frac{i}{\hbar}\varepsilon t}\cdot \psi(\vec r)
\end{gather*}
квадрат нормы пси-функции не зависит от времени: $\Vert\Psi\Vert^2=1\cdot\Vert\psi(\vec r)\Vert^2.$
\begin{gather*}
\partial_t\Vert\Psi\Vert^2=\partial_t\left(\Psi,\Psi\right)=\left(\partial_t\Psi,\Psi\right)+(\Psi,\partial_t\Psi)=\left[\partial_t\Psi=\frac{1}{i\hbar}\hat\mathcal{H}\Psi\right]=\\
=\left(\frac{1}{i\hbar}\hat\mathcal{H}\Psi,\Psi\right)+\left(\Psi,\frac{1}{i\hbar}\hat\mathcal{H}\Psi\right)=\frac{1}{i\hbar}\left[-(\Psi,\hat{\mathcal H}\Psi)+(\hat{\mathcal H}\Psi,\Psi)\right]=\\
=\frac{1}{i\hbar}\left[-(\hat{\mathcal H}^*\Psi,\Psi)+(\hat{\mathcal H}\Psi,\Psi)\right]=\left[\hat{\mathcal H}^*=\hat{\mathcal H}\right]=0
\end{gather*}
\par
Получим вектор плотности тока вероятности $j$:
\begin{gather*}
\partial_t\;\Psi^*\Psi=\partial_t\Psi^*\;\Psi+\Psi^*\partial_t\Psi=-\frac{1}{i\hbar}\hat{\mathcal H}\Psi^*\;\Psi+\Psi^*\frac{1}{i\hbar}\hat{\mathcal H}\Psi=\left[\hat{\mathcal H}=-\frac{\hbar^2}{2m}\Delta+U(\vec r)\right]=\\
=\frac{\hbar}{2im}\left(\Delta\Psi^*\;\Psi-\Psi^*\Delta\Psi\right)=-\nabla Re\left(\Psi^*\frac{\hbar}{im}\nabla\Psi\right)=-\nabla\vec j(t,\vec r)
\end{gather*}
Обозначим $\Psi^*(t,\vec r)\cdot\Psi(t,\vec r)=p(t,\vec r)$ - плотность вероятности, тогда получим ур-е непрерывности:
\begin{gather*}
\partial_t p(t,\vec r)+\nabla\cdot\vec j(t,\vec r)=0.
\end{gather*}
\end{document}