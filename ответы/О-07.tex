\documentclass[__main__.tex]{subfiles}

\begin{document}

\paragraph{О-07} Диполь Герца. Электромагнитные волны. Поперечность электромагнитных волн.\\

\textbf{Диполь Герца}. Электрический диполь, дипольный момент $\vec{p}(t)$ которого гармонически изменяется со временем. В радиотехнике диполь Герца эквивалентен небольшой антенне, размер которой много меньше длины волны. Является простейшей системой для получения электромагнитных волн.\\

\textbf{Электромагнитные волны} - распространяющееся в пространстве возмущения (изменение состояния) электромагнитного поля. Основными характеристиками принято считать частоту, длину волны и поляризацию. \\

\textbf{Поперечность электромагнитных волн} Электромагнитные волны поперечны - векторы $\vec{E}$ (напряженность эл. поля) и $\vec{B}$ (напр. магн. поля) перпендикулярны друг другу и лежат в плоскости, перпендикулярной распространению волны. \\

\end{document}