\documentclass[__main__.tex]{subfiles}

\begin{document}
\paragraph{K-31} Запишите и прокомментируйте соотношения неопределённостей для $\hat{x}$ и $\hat{p}$. Убедитесь в эрмитовости операторов $\hat{p}_x$ и $\hat{L}_x$.\\
Для $\hat{p}_x$  соотношениe неопределенности (которые я нашел в лекциях, да и в гугле) это соотношения неопределенности Гейзенберга, а именно:
$$\Delta x \Delta p_x\ge \hbar$$
Эти славные отношения по факту определяют границу классической механики...\\
Задачу можно рассматривать в классической механике, если неравенства Гейзенберга никак не влияют на результат\\
А если неравенства на результат влияют, то это уже квантмех.\\
Есть такая формулировка: чем точнее измеряется одна характеристика частицы, тем менее точно можно измерить вторую.\\

\textbf{Эрмитовость оператора $\hat{p}_x$}
$$ I_1=\int_{-\inf}^{\inf} \psi_1^*  (\frac{\hbar}{i} \frac{d}{dx}) \psi_2 dx = (\frac{\hbar}{i}) \int_{-\inf}^{\inf} \psi_1^* d\psi_2 $$
$$ I_2=\int_{-\inf}^{\inf}(\frac{\hbar}{i} \frac{d\psi_1^*}{dx})    \psi_2 dx =- (\frac{\hbar}{i}) \int_{-\inf}^{\inf} d\psi_1^*\psi_2 $$
В результате:
$$ I_1-I_2 =\frac{\hbar}{i}\int_{-\inf}^{\inf} d\psi_1^*\psi_2 = 0$$
Волновые функции квадратично интегрируемы и равны нулю на бесконечности, поэтому оператор импульса эрмитов.\\

\textbf{Эрмитовость оператора $\hat{L}_x$}
Разность эрмитовых операторов будет эрмитовым оператором, так что по сути задача сводится к доказательству эрмитовости операторов $\hat{y}\hat{p}_z$ и $\hat{z}\hat{p}_y$. \\
Произведение эрмитовых операторов ($\hat{z}, \hat{y}, \hat{p}_z, \hat{p}_y$ являются эрмитовыми) эрмитово только для коммутирующих множителей. Найдем коммутатор $[\hat{y}, \hat{p}_z]$:
$$
[\hat{y}, \hat{p}_z]=\hat{y}\hat{p}_z-\hat{p}_z\hat{y}=\frac{\hbar}{i}\left(y\frac{\partial}{\partial z}-\frac{\partial}{\partial z} y\right) =0
$$ 
Абсолютно аналогично доказывается, что $[\hat{z}, \hat{p}_y]=0$. Следовательно, операторы $\hat{y}\hat{p}_z$ и $\hat{z}\hat{p}_y$ эрмитовы и $\hat{L}_x$ эрмитов.
\end{document}