\documentclass[__main__.tex]{subfiles}

\begin{document}
\paragraph{Э-22}
Магнитное поле равномерно медленно движущегося заряда. Закон Био-Савара-Лапласа, границы его области применимости.\\

И как всегда, из многочисленных опытов было получено, что точечный заряд $q$,
двигаясь равномерно с небольшой скоростью $\mathcal v$,
порождает магнитное поле $B:$
\begin{gather*}
	B = \frac{q}{C\;r^3}\left[\mathcal{v}r\right],
\end{gather*}
где $r$ - вектор от заряда к наблюдаемой точке, С - некая константа.
Старая, умная советская книга говорит, что:
\begin{gather*}
	dB = \frac{1}{C}\frac{\left[j\;r\right]}{r^3}dV, \;\;V-volume
\end{gather*}
а для линейного элемента тока:
\begin{gather*}
	dB = \frac{\mathcal I}{C}\frac{\left[dl\;r\right]}{r^3},\;\;\mathcal{I}-current.
\end{gather*}
Эти формулы выражают закон БСЛ. Однако в дифф. форме этот закон не поддаётся опытной проверке, потому используют интегральную форму ( просто поставь крючок после = ).
Эти выражения применимы ЛИШЬ ДЛЯ ПОСТОЯННЫХ ТОКОВ, которые всегда замкнуты.
\end{document}