\documentclass[__main__.tex]{subfiles}

\begin{document}
\paragraph{К-14}
Принцип неразличимости частиц одного сорта: математическая формулировка, следствия.


Поскольку в квантовой механике частицы не имеют траектории, то при перекрытии волновых функции частиц одного сорта в какой-либо области пространства нельзя, обнаружив частицу, достоверно узнать с какой именно частицей мы имеем дело. Если даже в какой либо момент времени нам удалось локализовать электрон, то в последующий момент его координата станет неопределенной и можно говорить только о вероятности нахождения электрона в какой либо области пространства. Таким образом, если после процесса локализации и нумерации электронов в некоторый момент времени мы найдем электрон в данной области пространства, то нельзя точно указать, какой именно из электронов мы обнаружили (поскольку существует вероятность обнаружить любой из них). Это означает, что плотность вероятности  нахождения частицы в заданной точке пространства не должна зависеть от перестановки одинаковых частиц местами, т. е.:

\begin{gather*}
	|Y(x_1,x_2)|^2 = |Y(x_2,x_1)|^2
\end{gather*}

Это свойство получило название принципа неразличимости тождественных частиц: в квантовой механике тождественные частицы (имеющие одинаковую массу, заряд, спин и т.д.) принципиально неразличимы. Принцип неразличимости тождественных частиц является чрезвычайно важным, поскольку	из него следует вывод, что системы квантовых частиц необходимо изучать только в совокупности, а не индивидуально.

Данному условию будут удовлетворять две волновые функции:\\\\
$Y(x_1,x_2) = Y(x_2,x_1)$ - симметричная\\\\
$Y(x_1,x_2) = -Y(x_1,x_2)$ - антисимметричная

\end{document}